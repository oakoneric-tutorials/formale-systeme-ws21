\documentclass[ngerman, a4paper, 11pt]{scrartcl}

\usepackage[top=2.5cm,bottom=2.5cm,left=2.5cm,right=2.5cm]{geometry}
\usepackage{parskip}
\usepackage[onehalfspacing]{setspace} % increase row-space

\usepackage[ngerman]{babel}
\usepackage[utf8]{inputenc}
\usepackage{lmodern}
\usepackage[scale=0.8]{opensans}
\newcommand*{\osfamily}{\fontfamily{fos}\selectfont}
\DeclareTextFontCommand{\textos}{\osfamily}
\usepackage{bold-extra}

\usepackage{amsmath, amsthm, amssymb}

\usepackage[normalem]{ulem}
\usepackage[autostyle=true,english=british]{csquotes}
\usepackage{enumerate}
\usepackage[inline]{enumitem} % customize label
\usepackage{graphicx}
\usepackage[table,dvipsnames]{tudscrcolor} % package xcolor loaded automatically
\usepackage[font=small,labelfont=bf]{caption} % captions of non-floated figures

\usepackage{tikz}
\usetikzlibrary{automata, positioning, arrows, arrows.meta}

\usepackage[
	type={CC},
	modifier={by-nc-sa},
	version={4.0},
]{doclicense}

\usepackage[unicode,bookmarks=true]{hyperref}
\hypersetup{
	colorlinks,
	urlcolor=cdblue,
	pdfborder={0 0 0}			% no boxed around links
	pdfborderstyle={/S/U/W 1},	% underlining insteas of boxes
	linkbordercolor=cdblue,
	urlbordercolor=cdblue
}

\newcommand*\ruleline[1]{\par\noindent\raisebox{.8ex}{\makebox[\linewidth]{\hrulefill\hspace{1ex}\raisebox{-.8ex}{#1}\hspace{1ex}\hrulefill}}}

\newcommand{\defeq}{\mathrel{:=}}
\newcommand{\tuple}[1]{\langle #1 \rangle}
\newcommand{\menge}[1]{\left\{ #1 \right\}}
\newcommand{\quer}[1]{\overline{#1}}

\newtheoremstyle{bsp}
{6pt}		% Space above
{6pt}		% Space below
{\itshape}	% Body font
{}			% Indent amount
{\bfseries}	% Theorem head font
{.}			% Punctuation after theorem headi
{.5em}		% Space after theorem heading
{}			% Theorem head spec (can be left empty, meaning ‘normal’)
\theoremstyle{bsp}
\newtheorem*{beispiel}{Beispiel}
\theoremstyle{definition}
\newtheorem*{aufgabe}{Aufgabe}

\begin{document}

\begin{center}
	{\bfseries \sffamily \huge Aufgabe 2} 
	
	\ruleline{\sffamily \Large Übungsblatt 3}
	
	{\scshape Eric Kunze --- \today}
\end{center}
\medskip

{ \footnotesize \doclicenseThis }

\begin{center}	
	\slshape Keine Garantie auf Vollständigkeit und/oder Korrektheit!
\end{center}

\bigskip

\begin{aufgabe}
	Sei $L$ eine reguläre Sprache über einem mindestens zweielementigen Alphabet $\Sigma$.
	Zeigen Sie, dass die folgenden Sprachen regulär sind.	
\end{aufgabe} 

\begin{enumerate}[label=(\alph*)]
	\item $L_1 =  \{ x \in L : \text{es gibt kein } y \in \Sigma^+ \text{, so dass } xy \in L \}$
	\item $L_2 = \{ x \in L : \text{kein echtes Präfix von $x$ liegt in $L$} \}$
\end{enumerate}

Ausgangspunkt ist also in beiden Fällen eine reguläre Sprache $L$, für die wir einen DFA $\mathcal{M} = \tuple{Q, \Sigma, \delta, q_0, F}$ finden können, sodass $L = L(\mathcal{M})$ (d.h. $\mathcal{M}$ erkennt genau die Sprache $L$). Diesen Automaten haben wir gegeben und können ihn nun so verändern, dass er die Sprachen $L_1$ bzw. $L_2$ erkennt.

\section*{Teil (a): es gibt keine Verlängerung}

In $L_1$ sollen nur noch die Wörter $x$ aus $L$ enthalten sein, die sich nicht in $L$ verlängern lassen. Es soll also kein echtes Wort $y \in \Sigma^+$ geben, sodass das verlängerte Wort $xy$ in $L$ läge.
Wann kann eine solche Verlängerung auftreten? Wir müssten in $L$ sowohl das kurze Wort als auch das verlängerte Wort erkennen. Das können wir uns beispielhaft wie folgt vorstellen.

\begin{beispiel}
	Der Automat $\mathcal{M}$ für $L$ sei wie folgt gegeben:
	\begin{center}
		\begin{tikzpicture} [->, >=stealth', initial text=, auto, node
			distance=17mm, bend angle=20, semithick]
			\node[state,initial]   (q0)               {$q_0$};
			\node[state]           (q1) [right of=q0] {$q_1$};
			\node[state,accepting] (q2) [right of=q1] {$q_2$};
			\node[state]           (q3) [right of=q2] {$q_3$};
			\node[state,accepting] (q4) [right of=q3] {$q_4$};
			
			\path[->] (q0) edge node {$a$} (q1)
			(q1) edge node {$a$} (q2)
			(q2) edge node {$b$} (q3)
			(q3) edge node {$b$} (q4);
		\end{tikzpicture}
	\end{center}
	Damit ist $L = \menge{aa, aabb}$. Für das Wort $x = aa$ existiert folglich eine Verlängerung $y = bb \in \Sigma^+$, sodass $xy = aabb \in L$ gilt. Für $x = aabb$ existiert eine solche Verlängerung jedoch nicht. Somit gilt für dieses Beispiel $L_1 = \menge{aabb}$.
\end{beispiel}

Anhand des Beispiels erkennen wir relativ gut, dass wir die Akzeptanz von Wörtern \enquote{auf dem Weg} unterbinden müssen. Im Beispiel muss also die Akzeptanz in $q_2$ verboten werden, d.h. $q_2$ darf kein Finalzustand mehr sein, weil wir ausgehend von $q_2$ noch einmal in einen Finalzustand gelangen können.

Deswegen definieren wir uns die Menge
\begin{equation*}
	F' \defeq \menge{q \in F : \text{ex. kein } y \in \Sigma^\ast: \delta(q, y) \in F} .
\end{equation*}
Diese Menge besteht aus allen Finalzuständen $q \in F$ des Automaten $\mathcal{M}$ für $L$, für die wir kein Wort $y \in \Sigma^+$ finden können, sodass der Wortübergang $\delta(q, y)$ wieder in einem Finalzustand landet.

\begin{beispiel}
	Wir wollen diese Definition anhand des obigen Beispiels nachvollziehen. Es gilt auf jeden Fall $F' \subseteq F$, d.h. wir müssen nur Finalzustände des Originalautomaten betrachten. Für $q_2$ existiert $y = bb \in \Sigma^\ast$, sodass $\delta(q_2, y) = \delta(q_2, bb) = q_4 \in F$. Damit ist also $q_2 \notin F'$. Für $q_4$ finden wir aber offensichtlich keinen Wortübergang, der noch einmal in einem Finalzustand landet (da es überhaupt gar keinen Übergang mehr gibt), d.h. $q_4 \in F'$. Zusammengefasst ist also $F' = \menge{q_4}$.
\end{beispiel}

Ersetzen wir nun die Menge der Finalzustände $F$ im Originalautomaten durch $F'$, dann fallen alle unterwegs akzeptierenden Finalzustände weg und wir erreichen genau unser Ziel. Setze also
\begin{equation*}
	M' \defeq \tuple{Q, \Sigma, \delta, q_0, F'}
\end{equation*}
und übernehmen dabei alle anderen Komponenten (Zustandsmenge, Alphabet, Übergangsfunktion, Startzustand) des Originalautomaten.

\begin{beispiel}
	Der Automat $\mathcal{M}'$ für obigen Beispiel sieht dann wie folgt aus:
	\begin{center}
		\begin{tikzpicture} [->, >=stealth', initial text=, auto, node
			distance=17mm, bend angle=20, semithick]
			\node[state,initial]   (q0)               {$q_0$};
			\node[state]           (q1) [right of=q0] {$q_1$};
			\node[state]           (q2) [right of=q1] {$q_2$};
			\node[state]           (q3) [right of=q2] {$q_3$};
			\node[state,accepting] (q4) [right of=q3] {$q_4$};
			
			\path[->] (q0) edge node {$a$} (q1)
			(q1) edge node {$a$} (q2)
			(q2) edge node {$b$} (q3)
			(q3) edge node {$b$} (q4);
		\end{tikzpicture}
	\end{center}
	Man beachte, dass sich wirklich nur die Menge der Finalzustände geändert hat. Die erkannte Sprache ist nun $L(\mathcal{M}') = \menge{aabb}$, was genau $L_1$ für das Beispiel entspricht.
\end{beispiel}

Nun haben wir also einen DFA $\mathcal{M}'$ konstruiert, von dem wir hoffen, dass er genau $L_1$ erkennt. Um das zu verifizieren, müssen wir noch $L(\mathcal{M}') = L_1$ zeigen. 
\begin{align*}
	x \in L(\mathcal{M}') \quad 
	&\Leftrightarrow \quad \delta(q_0, x) \in F' \tag{Def. Akzeptanz}\\
	&\Leftrightarrow \quad \delta(q_0, x) \in F \text{ und ex. kein } y \in \Sigma^+: \delta(q_0, xy) \in F  \tag{Def. $F'$} \\
	&\Leftrightarrow \quad x \in L_1 \tag{Def. von $L_1$}\\
\end{align*}

Unsere Konstruktion ist folglich korrekt.

\pagebreak

\section*{Teil (b): keinen Präfix}

Die Sprache $L_2$ soll genau die Wörter von $L$ enthalten, die keinen echten\footnote{im folgenden steht Präfix immer für ein \textit{echtes} Präfix} Präfix in $L$ haben. Es soll also das lange Wort $xy$ in $L$ liegen, nicht jedoch der Präfix $x$.

\begin{beispiel}
	Wir arbeiten wieder mit unserem Beispielautomaten $\mathcal{M}$.
	\begin{center}
		\begin{tikzpicture} [->, >=stealth', initial text=, auto, node
			distance=17mm, bend angle=20, semithick]
			\node[state,initial]   (q0)               {$q_0$};
			\node[state]           (q1) [right of=q0] {$q_1$};
			\node[state,accepting] (q2) [right of=q1] {$q_2$};
			\node[state]           (q3) [right of=q2] {$q_3$};
			\node[state,accepting] (q4) [right of=q3] {$q_4$};
			
			\path[->] (q0) edge node {$a$} (q1)
			(q1) edge node {$a$} (q2)
			(q2) edge node {$b$} (q3)
			(q3) edge node {$b$} (q4);
		\end{tikzpicture}
	\end{center}	
	Nach wie vor gilt $L = L(\mathcal{M}) = \menge{aa, aabb}$. Für $xy = aa$ finden wir keinen Präfix, der auch in $L$ liegt. Es gäbe nur die Möglichkeiten $x = \epsilon \notin L$ und $x = a \notin L$. Das ergibt $aa \in L_2$. Für $xy = aabb$ finden wir jedoch einen solchen Präfix, nämlich mit $x = aa \in L$. Somit ist $aabb \notin L_2$
\end{beispiel}

Am Beispiel machen wir uns wieder schnell klar, was wir verbieten müssen: erreichen wir einmal einen Finalzustand, dürfen wir nicht mehr weitergehen, da sonst ein längeres Wort enstehen kann, dessen Präfix wir mit Erreichen des ersten Finalzustands auch akzeptieren.
Im Beispielautomaten darf vom Finalzustand $q_2$ kein Übergang mehr erfolgen, weil wir sonst noch in den Finalzustand $q_4$ gelangen können. Dort wird das Wort $aabb$ akzeptiert, bis $q_2$ haben wir aber schon dessen Präfix $aa$ akzeptiert und das soll in $L_2$ nicht möglich sein. 
Wir definieren also für alle Symbole $a \in \Sigma$ die Übergänge 
\begin{equation*}
	\delta''(q, a) \defeq \begin{cases}
		q_\bot &\text{falls } q \in F \\
		\delta(q, a) &\text{sonst}
	\end{cases}
\end{equation*}
Dabei soll $q_\bot \notin Q$ ein neu eingeführter Zustand sein, in dem wir anhalten ohne zu akzeptieren (es gibt keinen ausgehenden Übergang, aber $q_\bot$ ist auch kein Finalzustand).

Damit können wir nun einen DFA
\begin{equation*}
	\mathcal{M}'' \defeq \tuple{Q \cup \menge{q_\bot}, \Sigma, \delta'', q_0, F}
\end{equation*}
definieren, der obige Übergangsfunktion $\delta''$ verwendet sowie den neu eingeführten Fangzustand $q_\bot$.

\begin{beispiel}
	Für unser Beispiel fügen wir den Zustand $q_\bot$ und Übergänge von den Finalzuständen $q_2$ und $q_4$ mit den Symbolen $a$ und $b$ aus $\Sigma$ zum Fangzustand. Der Übergang $q_2 \overset{b}{\to} q_3$ wird dabei überschrieben durch den Übergang $q_2 \overset{a,b}{\to} q_\bot$.
	\begin{center}
		\begin{tikzpicture} [->, >=stealth', initial text=, auto, node
			distance=17mm, bend angle=20, semithick]
			\node[state,initial]   (q0)               {$q_0$};
			\node[state]           (q1) [right of=q0] {$q_1$};
			\node[state,accepting] (q2) [right of=q1] {$q_2$};
			\node[state]           (q3) [right of=q2] {$q_3$};
			\node[state,accepting] (q4) [right of=q3] {$q_4$};
			\node[state]           (qb) [below of=q3] {$q_\bot$};
			
			\path[->] (q0) edge node {$a$} (q1)
			(q1) edge node {$a$}   (q2)
			(q2) edge node[below left] {$a,b$} (qb)
			(q3) edge node {$b$}   (q4)
			(q4) edge node {$a,b$} (qb);
		\end{tikzpicture}
	\end{center}	
\end{beispiel}

Nun bleibt auch hier zu zeigen, dass unsere Konstruktion korrekt ist, d.h. dass $L(\mathcal{M}'') = L_2$ gilt.
\begin{align*}
	x \in L(\mathcal{M}'') \quad 
	&\Leftrightarrow \quad \delta''(q_0, x) \in F \tag{Def. Akzeptanz}\\
	&\Leftrightarrow \quad \delta(q_0, x) \in F \text{ und ex. kein echtes Präfix } \quer{x} \text{ von } x \text{ mit } \delta(q_0, \quer{x}) \in F \tag{Def. $\delta''$} \\
	&\Leftrightarrow \quad x \in L_2 \tag{Def. von $L_2$}\\
\end{align*}


\section*{Erkennt $\mathcal{M}'$ auch $L_2$?}

Wir verwenden wieder unser Beispiel und wollen $L_2 = \menge{aa}$ mit dem Automat $\mathcal{M}'$ erkennen.
\begin{center}
	\begin{tikzpicture} [->, >=stealth', initial text=, auto, node
		distance=17mm, bend angle=20, semithick]
		\node[state,initial]   (q0)               {$q_0$};
		\node[state]           (q1) [right of=q0] {$q_1$};
		\node[state]           (q2) [right of=q1] {$q_2$};
		\node[state]           (q3) [right of=q2] {$q_3$};
		\node[state,accepting] (q4) [right of=q3] {$q_4$};
		
		\path[->] (q0) edge node {$a$} (q1)
		(q1) edge node {$a$} (q2)
		(q2) edge node {$b$} (q3)
		(q3) edge node {$b$} (q4);
	\end{tikzpicture}
\end{center}
Wir können zwar die Übergänge $q_0 \overset{a}{\longrightarrow} q_1 \overset{a}{\longrightarrow} q_2$ durchführen und das Wort $aa$ zwischenzeitlich lesen, jedoch ist nach unserer Konstruktion $q_2$ kein Finalzustand mehr, sodass wir an dieser Stelle nicht akzeptieren und weitere Übergänge durchführen müssten. Dadurch verlängert sich das Wort $aa$ zwangsweise zu $aabb$. Es gilt also $aa \notin L(\mathcal{M}')$ und somit schon $L_2 \neq L(\mathcal{M}')$. $L_2$ wird folglich nicht von $\mathcal{M}'$ erkannt.

\section*{Erkennt $\mathcal{M}''$ auch $L_1$?}

Wieder versuchen wir anhand des Beispiels die Sprache $L_1 = \menge{aabb}$ mit dem Automaten $\mathcal{M}''$ zu erkennen.
\begin{center}
	\begin{tikzpicture} [->, >=stealth', initial text=, auto, node
		distance=17mm, bend angle=20, semithick]
		\node[state,initial]   (q0)               {$q_0$};
		\node[state]           (q1) [right of=q0] {$q_1$};
		\node[state,accepting] (q2) [right of=q1] {$q_2$};
		\node[state]           (q3) [right of=q2] {$q_3$};
		\node[state,accepting] (q4) [right of=q3] {$q_4$};
		\node[state]           (qb) [below of=q3] {$q_\bot$};
		
		\path[->] (q0) edge node {$a$} (q1)
		(q1) edge node {$a$}   (q2)
		(q2) edge node[below left] {$a,b$} (qb)
		(q3) edge node {$b$}   (q4)
		(q4) edge node {$a,b$} (qb);
	\end{tikzpicture}
\end{center}	
Wir starten und können zumindest den Teil $aa$ lesen und würden diesen auch schon akzeptieren. Es ist also $aa \in L(\mathcal{M}'')$ aber offensichtlich $aa \notin L_1$. Somit muss schon $L_1 \neq L(\mathcal{M}'')$ gelten, d.h. $L_1$ wird nicht von $\mathcal{M}''$ erkannt.

\end{document}
