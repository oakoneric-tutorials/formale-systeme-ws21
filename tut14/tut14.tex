\documentclass{beamer}
\usepackage{../tut-slides}
\usepackage{../mathoperators}
\usepackage{../fs}

\usepackage{csquotes}

\usepackage{amsmath,amssymb}
%\usepackage{enumerate}
\usepackage[normalem]{ulem}
\newcommand{\labelitemi}{\raisebox{1pt}{\scalebox{.9}{$\blacktriangleright$}}}
\newcommand{\labelitemii}{$\vartriangleright$}
\newcommand{\labelitemiii}{--}

\usepackage{booktabs}
\usepackage{tabularx}

\newcommand{\tuple}[1]{\langle{#1}\rangle}
\newcommand{\simquot}[1]{#1/_{\!\!{\sim}}}

\usepackage{pifont}

\newcommand{\cmark}{\textcolor{cddarkgreen}{\ding{51}}}%
\newcommand{\xmark}{\textcolor{darkred}{\ding{55}}}%

\newcommand{\solutionmarkB}{%
	\begin{tikzpicture}[remember picture, overlay]
		\node [
			fill=none,  % Farbe des Randstreifens
			text=cdorange,  % Textfarbe
			font=\fosfamily\bfseries\large,  % Einstellungen für die Schrift
			inner xsep=0,       % Abstand des Textes von unten
			% maximale Textbreite = Papierhöhe - 2*Abstand des Textes von unten:
			%			text width={\dimexpr\paperheight-2\footskip\relax},
			align=center,
			%			minimum height=8mm,% Breite des Randstreifens
			anchor=south west,
			rotate=90
		] at ($(current page.north west)+(+8mm,-20mm)$)
		{Lösung};
		\draw[draw, line width=2pt, color=cdorange] ($(current page.north west)+(+1mm,0)$) -- ($(current page.south west)+(+1mm,0)$);
	\end{tikzpicture}%
}
\newcommand{\solutionmarkFT}{\begin{tikzpicture}[remember picture, overlay]
	\node [
	fill=none,  % Farbe des Randstreifens
	text=cdorange,  % Textfarbe
	font=\fosfamily\bfseries\large,  % Einstellungen für die Schrift
	inner xsep=0,       % Abstand des Textes von unten
	% maximale Textbreite = Papierhöhe - 2*Abstand des Textes von unten:
	%			text width={\dimexpr\paperheight-2\footskip\relax},
	align=center,
	%			minimum height=8mm,% Breite des Randstreifens
	anchor=south west,
	rotate=90
	] at ($(current page.north west)+(+8mm,-27mm)$)
	{Lösung};
	\draw[draw, line width=2pt, color=cdorange] ($(current page.north west)+(+1mm,0)$) -- ($(current page.south west)+(+1mm,0)$);
	\end{tikzpicture}%
}


%%%%%%%%%%%%%%%%%%%%%%%%%%%%%%%%%%%%%%%%%%%%%%%%%%%%%%%%%%%%%%%%%%%%%%%
\newcommand{\ghost}[1]{\raisebox{0pt}[0pt][0pt]{\makebox[0pt][l]{#1}}}
\newcommand{\blue}[1]{\textcolor{darkblue}{#1}}
\newcommand{\purple}[1]{\textcolor{cdpurple}{#1}}
\newcommand{\green}[1]{\textcolor{cddarkgreen}{#1}}
\newcommand{\orange}[1]{\textcolor{cdorange}{#1}}
\newcommand{\indigo}[1]{\textcolor{cdindigo}{#1}}

\usepackage{ragged2e}

\newcommand{\cellblue}{\cellcolor{cdblue!20}}
\newcommand{\cellred}{\cellcolor{cdpurple!20}}

\undef\Var
\DeclareMathOperator{\Var}{Var}


\begin{document}	
	\title{Formale Systeme}
	\subtitle{Übung 14}
	\author{Eric Kunze}
	\email{eric.kunze@tu-dresden.de}
	\city{TU Dresden}
	\date{\formatdate{4}{2}{2022}}
%	\institute{Lehrstuhl für Grundlagen der Programmierung}
	\titlegraphic{\includegraphics[width=2cm]{../TUD-white.pdf}}

	\maketitle


	\begin{frame} \frametitle{Übungsblatt 14}
		\tableofcontents
	\end{frame}

	
	\section{Aufgabe 1 \\ \itshape Wiederholung}
	
	
	\begin{frame} \frametitle{Aufgabe 1}
		\footnotesize
		Stimmen die folgenden Aussagen? Begründen Sie.
		\begin{enumerate}[a)]
			\item Wenn $\Gamma \models \psi$ und $\psi$ eine Tautologie ist, dann ist $\Gamma$ auch allgemeingültig.
			
			\only<2->{%
				\begin{itshape}
					\textbf{Falsch} --- Gegenbeispiel: Aus Falschem folgt Beliebiges. 
					\begin{equation*}
						\psi = (p \lor \lnot p) \text{ ist tautologisch.} \qquad \qquad 
						\Gamma = (p \land \lnot p) \text{ ist unerfüllbar.} 
					\end{equation*}
					Dennoch gilt $\Gamma \models \psi$.
				\end{itshape}}
			\item Eine Formel $\phi$ ist eine Hornformel, wenn sie in NNF ist und aus Konjunktionen innerhalb von Disjunktionen von Literalen besteht, von denen jede höchstens ein positives Literal enthält.
			
			\only<3->{%
				\begin{itshape}
					\textbf{Falsch} --- Konjunktionen innerhalb von Disjunktionen $\leadsto$ disjunktive Normalform
				\end{itshape}}
			\item Für $K_1 = \menge{a, b, c}$ und $K_2 = \menge{\lnot a, \lnot b}$ ist $\menge{c}$ keine Resolvente.
			
			\only<4->{%
				\begin{itshape}
					\textbf{Richtig} --- es können nur die Klauseln $\menge{b, c, \lnot b}$ und $\menge{a, c, \lnot a}$ entstehen.
				\end{itshape}}
			\item Aus $\models \phi$ folgt, dass $\phi$ allgemeingültig ist.
			
			\only<5>{%
				\begin{itshape}
					\textbf{Richtig} --- per Definition (siehe VL 22, Folie 17)
			\end{itshape}}
		\end{enumerate}
	\end{frame}

	\section{Aufgabe 2 \\ \itshape Logisches Schließen}

	\begin{frame} \frametitle{Aufgabe 2}
		\justifying \footnotesize
		Welche der folgenden Formeln sind erfüllbar? 
		Welche der Formeln gehören zu dem Formeltyp, für den Erfüllbarkeit in polynomieller Zeit lösbar ist?
		\begin{gather}
			\bigg( \Big( \big( (p_1 \land p_2) \lor p_3 \big) \land \big( \lnot p_1 \lor \lnot p_3 \big) \Big) \lor \Big( \lnot p_2 \land \lnot p_4 \Big) \bigg)			
			\\
			\bigg( \lnot \Big( \lnot p_1 \land \lnot \big(p_2 \land (\lnot p_1 \to p_2 ) \big) \Big) \land \lnot p_2 \bigg)
			\\
			\left.\begin{array}{rcl}
				\Big(  \quad
				(\lnot p_1 \lor \lnot p_2 \lor \lnot p_3 \lor p_4) & \land & \\
				(\lnot p_5 \lor \lnot p_6) & \land & \\
				(\lnot p_7 \lor \lnot p_2 \lor p_6) & \land & \\
				(\lnot p_6 \lor \lnot p_2) & \land & \\
				(\lnot p_6 \lor \lnot p_3 \lor p_2) & \land & \\
				(\lnot p_3 \lor \lnot p_4 \lor p_5) & \land & \\
				(\lnot p_1 \lor p_7) & \land & \\
				(\lnot p_1 \lor \lnot p_7 \lor p_4) & \land & \\				
				p_3 &\land& \\
				p_1 && \Big)
			\end{array} \right] \\			
			\Big( (p_1 \lor \lnot p_2) \land (p_2 \lor \lnot p_3) \land (p_3 \lor \lnot p_1) \land (\lnot p_2 \lor \lnot p_3) \land p_1 \Big)
		\end{gather}
	\end{frame}
	
	\begin{frame}
		\footnotesize
		\begin{enumerate}[\bfseries (a)]
			\item erfüllbar, z.B. via Transformation in DNF und Entscheidung, ob Monome gegensätzliche Literale enthalten \\
			erfüllende Wertzuweisung: $w(p_2) = 0$ und $w(p_4) = 0$ reicht
			\item erfüllbar mit $w(p_1) = 1$ und $w(p_2) = 0$
		\end{enumerate}
		Sowohl (a) als auch (b) sind keine Horn-Formel und daher ist ihre Erfüllbarkeit nicht in $\Slang P$ entscheidbar.
		\begin{enumerate}[\bfseries (a)]
			\setcounter{enumi}{2}
			\item 
			\begin{minipage}[t]{\dimexpr0.4\linewidth-\fboxrule-\fboxsep}
				Horn-Regelmenge:
				\begin{align*}
					p_1 \land p_2 \land p_3 &\to p_4 \\
					p_5 \land p_6 &\to \bot \\
					p_7 \land p_2 &\to p_6 \\
					p_6 \land p_2 &\to \bot \\
					p_6 \land p_3 &\to p_2& \\
					p_3 \land p_4 &\to p_5 \\
					p_1 &\to p_7 \\
					p_1 \land p_7 &\to p_4 \\				
					\top &\to p_3 \\
					\top &\to p_1
				\end{align*}
			\end{minipage}
			\begin{minipage}[t]{\dimexpr0.6\linewidth-\fboxrule-\fboxsep}
				Hyperresolution:
				\begin{align*}
					V_0 &= \menge{p_3, p_1} \\
					V_1 &= V_0 \cup \menge{p_7} \\
					V_2 &= V_1 \cup \menge{p_4} \\
					V_3 &= V_2 \cup \menge{p_5} = V_4 = V
				\end{align*}
				Es gibt keine Regel $q_0 \land \dots \land q_m \to \bot$ mit $q_0, \dots, q_m \in V$. Daher ist die Formel erfüllbar.
			\end{minipage}
		\end{enumerate}
	\solutionmarkB
	\end{frame}

	\begin{frame}
		\footnotesize
		\begin{enumerate}[\bfseries (a)]
			\setcounter{enumi}{3}
			\item 
			\begin{minipage}[t]{\dimexpr0.4\linewidth-\fboxrule-\fboxsep}
				Horn-Regelmenge:
				\begin{align*}
				p_2 &\to p_1 \\
				p_3 &\to p_2 \\
				p_1 &\to p_3 \\
				p_2 \land p_3 &\to \bot \\
				\top &\to p_1
				\end{align*}
			\end{minipage}
			\begin{minipage}[t]{\dimexpr0.6\linewidth-\fboxrule-\fboxsep}
				Hyperresolution:
				\begin{align*}
				V_0 &= \menge{p_1} \\
				V_1 &= V_0 \cup \menge{p_3} \\
				V_2 &= V_1 \cup \menge{p_2} = V_3 = V
				\end{align*}
				Da $(p_2 \land p_3 \to \bot)$ eine Horn-Regel mit $p_2, p_3 \in V$ ist, ist die Formel unerfüllbar.
			\end{minipage}
		\end{enumerate}
		Da (c) und (d) jeweils Horn-Formeln sind, kann deren Erfüllbarkeit in polynomieller Zeit entschieden werden.
	\solutionmarkB
	\end{frame}

	\section{Aufgabe 3 \\ \itshape Ziffernfolgen in $\pi$ finden}
	
	\begin{frame} \frametitle{Aufgabe 3}
		\justifying \small
		Begründen Sie die Semientscheidbarkeit des folgenden Problems:
		\begin{itemize}
			\item Gegeben ist eine Zahlenfolge $s = s_1 s_2 \dots s_n \in \menge{0, 1, \dots , 9}^n$ ($n \ge 1$).
			\item Gefragt: Kommt in dem Nachkommateil der Dezimaldarstellung von $\pi$ die Sequenz $s$ vor?
		\end{itemize}

		Hinweis: \\ \itshape
		Sie dürfen als bekannt voraussetzen, dass es beliebig genaue Näherungs\-verfahren für $\pi$ gibt. Skizzieren Sie die Arbeitsweise eines Semientscheidungsverfahrens für das genannte Problem unter Verwendung eines Algorithmus  $\operatorname{Pi-N"aherungsverfahren}(k)$, das als Eingabe eine natürliche Zahl $k \ge 1$ hat und als Ausgabe die $k$ ersten Ziffern des Nachkommateils der Dezimaldarstellung von $\pi$ zurückgibt.
	\end{frame}

	\begin{frame}
		Eingabe: $s = s_1 s_2 \dots s_n \in \menge{0, 1, \dots, 9}^n$ mit $n \ge 1$
		
		FOR $k = n, n+1, \dots$: \\
		\hspace{2em} Berechne $a_1 \dots a_k = \operatorname{Pi-N"aherungsverfahren}(k)$ \\
		\hspace{2em} Falls $a_{k-n+1} \dots a_k = s_1 \dots s_n$: \hspace{1em} akzeptiere \\
		\hspace{2em} Sonst: wähle nächstes $k$ \\
		
		\solutionmarkB
	\end{frame}

	\section{Aufgabe 4 \\ \itshape Mengen-Constraint-Systeme}
	
	\begin{frame} \frametitle{Aufgabe 4}
		\footnotesize \justifying
		Gegeben sei eine endliche Menge $E$ von Elementen und eine Menge $V$ von Variablen. Ein
		Mengen-Constraintsystem $C$ über $E$ und $V$ ist eine endliche Menge von Constraints der Form:
		\begin{equation*}
			a \in X, a \notin X, a \in X \cup Y, \text{ oder } X \subseteq Y \cup Z
		\end{equation*}
		für $a \in E$ und $X, Y, Z \in V$. Eine Lösung $L$ eines Mengen-Constraintsystems $C$ über $E$ und $V$
		ist eine Abbildung $L : V \to 2^E$, so dass
		\begin{align*}
			\text{für alle Ausdrücke der Form } (a \in X) \in C \text{ gilt}: \quad &a \in L(X), \\
			\text{für alle Ausdrücke der Form } (a \notin X) \in C \text{ gilt}: \quad &a \notin L(X), \\
			\text{für alle Ausdrücke der Form } (a \in X \cup Y ) \in C \text{ gilt}: \quad &a \in L(X) \cup L(Y ), \\
			\text{für alle Ausdrücke der Form } (X \subseteq Y \cup Z) \in C \text{ gilt}: \quad &L(X) \subseteq L(Y ) \cup L(Z).
		\end{align*}
		
		\begin{enumerate}[\bfseries a)]
			\item Hat das folgende Mengen-Constraintsystem eine Lösung?
			\begin{align*}
				V &= \menge{M_1, M_2, M_3, M_4} \qquad \und \qquad
				E = \menge{a, b, c, d} \\
				C &= \menge{
					\begin{array}{l}
						M_2 \subseteq M_1 \cup M_3, M_4 \subseteq M_3 \cup M_2, \\
						a \in M_1, a \notin M_3, b \in M_4, b \in M_1, b \notin M_3, \\
						c \in M_4, c \notin M_1, c \notin M_3, d \in M_4, d \notin M_1, d \notin M_2
					\end{array} }
			\end{align*}
		\end{enumerate}
	\end{frame}

	\begin{frame}
		\justifying
		\textbf{Teil (a)}
		\begin{itemize}
			\item Aus $M_4 \subseteq M_3 \cup M_2$ wissen wir, dass $c \in L(M_2)$, da $c \in L(M_4)$ und $c \notin L(M_3)$ gelten muss.
			\item Daraus folgernd muss wegen $M_2 \subseteq M_1 \cup M_3$ auch $c \in L(M_1)$ oder $c \in L(M_3)$ gelten (da $c \in L(M_4)$ gilt).
		\end{itemize}
	
		Die Forderung $c \in L(M_1)$ kann aber wegen des Constraints $c \notin M_1$ nicht erfüllt werden. Analog kann $c \in L(M_3)$ durch $c \notin M_3$ nicht erfüllt werden. Damit ist $M_2 \subseteq M_1 \cup M_3$ nicht haltbar und das System kann keine Lösung besitzen.
		\solutionmarkB
	\end{frame}

	\begin{frame} \frametitle{Aufgabe 4}
		\footnotesize \justifying
		Gegeben sei eine endliche Menge $E$ von Elementen und eine Menge $V$ von Variablen. Ein
		Mengen-Constraintsystem $C$ über $E$ und $V$ ist eine endliche Menge von Constraints der Form:
		\begin{equation*}
			a \in X, a \notin X, a \in X \cup Y, \text{ oder } X \subseteq Y \cup Z
		\end{equation*}
		für $a \in E$ und $X, Y, Z \in V$. Eine Lösung $L$ eines Mengen-Constraintsystems $C$ über $E$ und $V$
		ist eine Abbildung $L : V \to 2^E$, so dass
		\begin{align*}
			\text{für alle Ausdrücke der Form } (a \in X) \in C \text{ gilt}: \quad &a \in L(X), \\
			\text{für alle Ausdrücke der Form } (a \notin X) \in C \text{ gilt}: \quad &a \notin L(X), \\
			\text{für alle Ausdrücke der Form } (a \in X \cup Y ) \in C \text{ gilt}: \quad &a \in L(X) \cup L(Y ), \\
			\text{für alle Ausdrücke der Form } (X \subseteq Y \cup Z) \in C \text{ gilt}: \quad &L(X) \subseteq L(Y ) \cup L(Z).
		\end{align*}
		
		\begin{enumerate}[\bfseries a)]
			\setcounter{enumi}{1}
			\item Geben Sie ein allgemeines Verfahren an, das Lösungen eines Mengen-Constraintsystems
			in polynomieller Zeit entscheidet.
			
			\itshape Hinweis: Übersetzen Sie $C$ in eine endliche Menge von Hornformeln – entscheidend ist die
			Kodierung der Mengenzugehörigkeit von Elementen.
		\end{enumerate}
	\end{frame}

	\begin{frame}
		\footnotesize \justifying
		\textbf{Teil (b):} Wir wollen das Problem auf $\Slang{Horn-SAT} \in \Slang P$ reduzieren, d.h. eine Horn-Formel (bzw. eine Horn-Regelmenge) finden, deren Erfüllbarkeit die Lösbarkeit eines Mengen-Constraint-Systems zeigt.
		
		Gegeben sei also eine Mengen-Constraint-System $C$ über Elementen $E$ und Variablen $V$.
		
		Wir verwenden die aussagenlogischen Variablen
		\begin{align*}
			\mathcal{P} &= \menge{p_{a, X} : (a \notin X) \in C}
		 \intertext{sowie die Regelmenge}
			\Gamma &= \menge{(\top \to p_{a,X}) : (a \notin X) \in C} \\
			&\quad\cup \menge{(p_{a,X} \to \bot) : (a \in X) \in C} \\
			&\quad\cup \menge{(p_{a,X} \land p_{a, Y} \to \bot) : (a \in X \cup Y) \in C} \\
			&\quad\cup \bigcup_{a \in E} \menge{((p_{a,Y} \land p_{a, Z}) \to p_{a, X}) : (X \subseteq Y \cup Z) \in C}. 
		\end{align*}
		Dann ist das Mengen-Constraint-System $C$ genau dann erfüllbar, wenn $\Gamma$ erfüllbar ist. Die Erfüllbarkeit der Horn-Regelmenge kann gemäß Vorlesung in polynomieller Zeit entschieden werden.
		\solutionmarkB
	\end{frame}

\end{document}