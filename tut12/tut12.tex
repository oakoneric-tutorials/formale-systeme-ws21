\documentclass{beamer}
\usepackage{../tut-slides}
\usepackage{../mathoperators}
\usepackage{../fs}

\usepackage{csquotes}

\usepackage{amsmath,amssymb}
%\usepackage{enumerate}
\usepackage[normalem]{ulem}
\newcommand{\labelitemi}{\raisebox{1pt}{\scalebox{.9}{$\blacktriangleright$}}}
\newcommand{\labelitemii}{$\vartriangleright$}
\newcommand{\labelitemiii}{--}

\usepackage{booktabs}
\usepackage{tabularx}

\newcommand{\tuple}[1]{\langle{#1}\rangle}
\newcommand{\simquot}[1]{#1/_{\!\!{\sim}}}

\usepackage{pifont}

\newcommand{\cmark}{\textcolor{cddarkgreen}{\ding{51}}}%
\newcommand{\xmark}{\textcolor{darkred}{\ding{55}}}%

\newcommand{\solutionmarkB}{%
	\begin{tikzpicture}[remember picture, overlay]
		\node [
			fill=none,  % Farbe des Randstreifens
			text=cdorange,  % Textfarbe
			font=\fosfamily\bfseries\large,  % Einstellungen für die Schrift
			inner xsep=0,       % Abstand des Textes von unten
			% maximale Textbreite = Papierhöhe - 2*Abstand des Textes von unten:
			%			text width={\dimexpr\paperheight-2\footskip\relax},
			align=center,
			%			minimum height=8mm,% Breite des Randstreifens
			anchor=south west,
			rotate=90
		] at ($(current page.north west)+(+8mm,-20mm)$)
		{Lösung};
		\draw[draw, line width=2pt, color=cdorange] ($(current page.north west)+(+1mm,0)$) -- ($(current page.south west)+(+1mm,0)$);
	\end{tikzpicture}%
}
\newcommand{\solutionmarkFT}{\begin{tikzpicture}[remember picture, overlay]
	\node [
	fill=none,  % Farbe des Randstreifens
	text=cdorange,  % Textfarbe
	font=\fosfamily\bfseries\large,  % Einstellungen für die Schrift
	inner xsep=0,       % Abstand des Textes von unten
	% maximale Textbreite = Papierhöhe - 2*Abstand des Textes von unten:
	%			text width={\dimexpr\paperheight-2\footskip\relax},
	align=center,
	%			minimum height=8mm,% Breite des Randstreifens
	anchor=south west,
	rotate=90
	] at ($(current page.north west)+(+8mm,-27mm)$)
	{Lösung};
	\draw[draw, line width=2pt, color=cdorange] ($(current page.north west)+(+1mm,0)$) -- ($(current page.south west)+(+1mm,0)$);
	\end{tikzpicture}%
}


%%%%%%%%%%%%%%%%%%%%%%%%%%%%%%%%%%%%%%%%%%%%%%%%%%%%%%%%%%%%%%%%%%%%%%%
\newcommand{\ghost}[1]{\raisebox{0pt}[0pt][0pt]{\makebox[0pt][l]{#1}}}
\newcommand{\blue}[1]{\textcolor{darkblue}{#1}}
\newcommand{\purple}[1]{\textcolor{cdpurple}{#1}}
\newcommand{\green}[1]{\textcolor{cddarkgreen}{#1}}

\usepackage{ragged2e}
\undef\Var
\DeclareMathOperator{\Var}{Var}
\DeclareMathOperator{\Sub}{Sub}


\begin{document}	
	\title{Formale Systeme}
	\subtitle{Übung 12}
	\author{Eric Kunze}
	\email{eric.kunze@tu-dresden.de}
	\city{TU Dresden}
	\date{\formatdate{21}{1}{2022}}
%	\institute{Lehrstuhl für Grundlagen der Programmierung}
	\titlegraphic{\includegraphics[width=2cm]{../TUD-white.pdf}}

	\maketitle


	\begin{frame} \frametitle{Übungsblatt 12}
		\tableofcontents
	\end{frame}

	\begin{frame}\frametitle{Logische Äquivalenz}
		\small
		Formeln sind äquivalent, wenn sie die gleiche Semantik haben:
		
		\defbox{Zwei Formeln $F$ und $G$ sind \redalert{semantisch äquivalent}, in Symbolen \redalert{$F\equiv G$},
			wenn sie genau die selben Modelle haben, d.h. wenn\\[1ex]
			
			\narrowcentering{für alle Wertzuweisungen $w$ gilt: $w(F)=w(G)$}}
	\end{frame}

	\begin{frame}\frametitle{Junktoren äquivalent ausdrücken}
		\footnotesize
		\begin{align*}
			F\to G & \equiv \neg F\vee G \equiv \neg(F\wedge\neg G) \\
			F\leftrightarrow G & \equiv (F\to G)\wedge(G\to F)\equiv (F\wedge G)\vee(\neg F\wedge \neg G) \\
			F\wedge G & \equiv \neg(\neg F\vee\neg G) \quad \text{\textcolor{devilscss}{(De Morgansches Gesetz)}} \\
			F\vee G & \equiv \neg(\neg F\wedge\neg G) \quad \text{\textcolor{devilscss}{(De Morgansches Gesetz)}}
		\end{align*}\pause
		
		\theobox{Satz: Sei $F$ eine beliebige aussagenlogische Formel.
			\begin{itemize}
				\item Es gibt eine zu $F$ äquivalente Formel, die nur die Junktoren $\wedge$ und $\neg$ enthält.
				\item Es gibt eine zu $F$ äquivalente Formel, die nur die Junktoren $\vee$ und $\neg$ enthält.
		\end{itemize}}		
	\end{frame}

	
	\begin{frame}\frametitle{Nützliche Äquivalenzen}
		\small
		\begin{align*}
			\begin{split}
				F\wedge G & \equiv G\wedge F\\
				F\vee G & \equiv G\vee F
			\end{split}
			& \text{\textcolor{devilscss}{Kommutativität}} \\[1ex]
			%
			\begin{split}
				(F\wedge G)\wedge H & \equiv F\wedge (G\wedge H)\\
				(F\vee G)\vee H & \equiv F\vee (G\vee H)
			\end{split}
			& \text{\textcolor{devilscss}{Assoziativität}} \\[1ex]
			%
			\begin{split}
				F\wedge (G\vee H) & \equiv (F\wedge G) \vee (F\wedge H)\\
				F\vee (G\wedge H) & \equiv (F\vee G)\wedge (F\vee H)
			\end{split}
			& \text{\textcolor{devilscss}{Distributivität}} \\[1ex]
			%
			\begin{split}
				F\wedge F & \equiv F\\
				F\vee F & \equiv F
			\end{split}
			& \text{\textcolor{devilscss}{Idempotenz}} \\[1ex]
			%
			\begin{split}
				F\wedge (F\vee G) & \equiv F\\
				F\vee (F\wedge G) & \equiv F
			\end{split}
			& \text{\textcolor{devilscss}{Absorption}}
		\end{align*}
		
	\end{frame}
	
	\begin{frame}\frametitle{Nützliche Äquivalenzen}
		\small
		\begin{align*}
			\neg\neg F &\equiv F
			& \text{\textcolor{devilscss}{doppelte Negation}}\\[1ex]
			%
			\begin{split}
				\neg(F\wedge G) & \equiv (\neg F\vee \neg G)\\
				\neg(F\vee G) & \equiv (\neg F\wedge \neg G)
			\end{split}
			& \text{\textcolor{devilscss}{De Morgansche Gesetze}} \\[1ex]
			%
			\begin{split}
				F\wedge \top & \equiv F\\
				F\vee \top & \equiv \top
			\end{split}
			& \text{\textcolor{devilscss}{Gesetze mit $\top$}}\\[1ex]
			%
			\begin{split}
				F\wedge \bot & \equiv \bot\\
				F\vee \bot & \equiv F
			\end{split}
			& \text{\textcolor{devilscss}{Gesetze mit $\bot$}}\\[1ex]
			%
			\begin{split}
				\neg\top & \equiv \bot\\
				\neg\bot & \equiv \top
			\end{split}
			& 
		\end{align*}
		
		Alle diese Äquivalenzen können leicht mit Wahrheitswertetabellen überprüft werden.
	\end{frame}


	\section{Aufgabe 1 \\ \itshape weitere Äquivalenzen}
	
	\begin{frame} \frametitle{Aufgabe 1}
		\small
		Zeigen Sie die Gültigkeit der folgenden Äquivalenzen:
		\begin{enumerate}[a)]
			\item Distributivitätsregel:
			\begin{equation*}
				(\phi \lor (\psi \land \pi)) \equiv ((\phi \lor \psi) \land (\phi \lor \pi))
			\end{equation*}
			\item Absorptionsregel:
			\begin{equation*}
				(\phi \land (\phi \lor \psi)) \equiv \phi
			\end{equation*}
		\end{enumerate}
	\end{frame}
	

	\section{Aufgabe 2 \\ \itshape mehr Äquivalenzen}

	\begin{frame} \frametitle{Aufgabe 2}
		\small
		Prüfen Sie, ob die folgenden Äquivalenzen gelten.
		\begin{enumerate}[(a)]
			\item $(((a \to b) \land (\lnot a \to (b \land c))) \land ((\lnot b \lor c) \to d))$ \\
			$\equiv$ \\
			$((\lnot (a \leftrightarrow b) \land (a \lor c)) \land  \lnot ((b \lor d) \to (c \land \lnot d)))$
			\item $(((b \land \ell) \to m) \land ((a \land b) \to \ell) \land a \land b)$ \\
			$\equiv$ \\
			$((\lnot b \land \ell \land \lnot a \land b) \lor (\lnot \ell \land \ell \land \lnot a \land b) \lor (m \land \ell \land a \land b))$
		\end{enumerate}
	\end{frame}

%	\begin{frame}
%		\newcommand{\cellblue}{\cellcolor{cdblue!20}}
%		\newcommand{\cellred}{\cellcolor{cdpurple!20}}
%		\small \vspace{1em}
%		\textbf{(a)} $\Big\{ (\neg a \lor b), (\neg b \lor c), (b \land c) \Big\} \models ((a \leftrightarrow b) \lor c)$
%		\begin{center}
%			\begin{tabular}{c|c|c||c|c|c||c|c}
%				\hline
%				$a$ & $b$ & $c$ & $\lnot a \lor b$ & $\lnot b \lor c$ & $b \land c$ & $a \leftrightarrow b$ & $(a \leftrightarrow b) \lor c$ \\
%				\hline
%				0 & 0 & 0 & 1 & 1 & 0 & 1 & \cellred 1 \\
%				0 & 0 & 1 & 1 & 1 & 0 & 1 &  \cellred1 \\
%				0 & 1 & 0 & 1 & 0 & 0 & 0 & 0 \\
%				0 & 1 & 1 & \cellblue 1 & \cellblue 1 & \cellblue 1 & 0 & \cellred 1 \\
%				1 & 0 & 0 & 0 & 1 & 0 & 0 & 0 \\
%				1 & 0 & 1 & 0 & 1 & 0 & 0 & \cellred 1 \\
%				1 & 1 & 0 & 1 & 0 & 0 & 1 & \cellred 1 \\
%				1 & 1 & 1 & \cellblue 1 & \cellblue 1 & \cellblue 1 & 1 & \cellred 1 \\ \hline
%			\end{tabular}
%		\end{center}
%	
%		Jedes Modell von \colorbox{cdblue!20}{$\Big\{ (\neg a \lor b), (\neg b \lor c), (b \land c) \Big\}$} ist auch Modell von \colorbox{cdpurple!20}{$((a \leftrightarrow b) \lor c)$}. Damit gilt die Aussage.
%		\solutionmarkB
%	\end{frame}
%
%	\begin{frame}
%		\small \vspace{1em}
%		\textbf{(b)} $\Big\{ (a \rightarrow b), (c \lor a), (a \rightarrow \neg b), \neg c \Big\} \models a$
%		\begin{center}
%			\begin{tabular}{c|c|c||c|c|c|c}
%				\hline
%				$a$ & $b$ & $c$ & $a \rightarrow b$ & $c \lor a$ & $a \rightarrow \lnot b$ & $\lnot c$ \\
%				\hline
%				0 & 0 & 0 & 1 & 0 & 1 & 1 \\
%				0 & 0 & 1 & 1 & 1 & 1 & 0 \\
%				0 & 1 & 0 & 1 & 0 & 1 & 1 \\
%				0 & 1 & 1 & 1 & 1 & 1 & 0 \\
%				1 & 0 & 0 & 0 & 1 & 1 & 1 \\
%				1 & 0 & 1 & 0 & 1 & 1 & 0 \\
%				1 & 1 & 0 & 1 & 1 & 0 & 1 \\
%				1 & 1 & 1 & 1 & 1 & 0 & 0 \\ \hline
%			\end{tabular}
%		\end{center}
%	
%		\justifying
%		Die Menge $\Big\{ (a \rightarrow b), (c \lor a), (a \rightarrow \neg b), \neg c \Big\}$ hat keine Modelle, d.h. es folgt Beliebiges. Insbesondere ist also jedes (nicht existente) Modell von $\Big\{ (a \rightarrow b), (c \lor a), (a \rightarrow \neg b), \neg c \Big\}$ auch Modell der Formel $a$. Damit gilt die Aussage.
%		
%		\solutionmarkB
%	\end{frame}
%
%	\begin{frame}
%		\small \vspace{1em}
%		\textbf{(c)} $\Big\{ (a \land \neg b) \lor (\neg a \land b), (\neg c \land b), \neg (\neg a \lor b) \Big\} \models \neg(a \lor b)$
%		\begin{center}
%			\begin{tabular}{c|c|c||c|c|c||c}
%				\hline
%				$a$ & $b$ & $c$ & $(a \land \lnot b) \lor (\lnot a \land b)$ & $\lnot c \land b$ & $\lnot (\lnot a \lor b)$ & $\lnot (a \lor b)$ \\
%				\hline
%				0 & 0 & 0 & 0 & 0 & 0 & 1 \\
%				0 & 0 & 1 & 0 & 0 & 0 & 1 \\
%				0 & 1 & 0 & 1 & 1 & 0 & 0 \\
%				0 & 1 & 1 & 1 & 0 & 0 & 0 \\
%				1 & 0 & 0 & 1 & 0 & 1 & 0 \\
%				1 & 0 & 1 & 1 & 0 & 1 & 0 \\
%				1 & 1 & 0 & 0 & 1 & 0 & 0 \\
%				1 & 1 & 1 & 0 & 0 & 0 & 0 \\ \hline
%			\end{tabular}
%		\end{center}
%		
%		\justifying
%		Die Menge $\Big\{ (a \land \neg b) \lor (\neg a \land b), (\neg c \land b), \neg (\neg a \lor b) \Big\}$ hat keine Modelle, d.h. es folgt Beliebiges. Insbesondere ist also jedes (nicht existente) Modell von $\Big\{ (a \land \neg b) \lor (\neg a \land b), (\neg c \land b), \neg (\neg a \lor b) \Big\}$ auch Modell der Formel $\neg(a \lor b)$. Damit gilt die Aussage.
%		
%		\solutionmarkB
%	\end{frame}

	\section{Aufgabe 3 \\ \itshape Normalformen}
	
	\begin{frame}\frametitle{Negationsnormalform}
		\small
		\defbox{Eine Formel $F$ ist in \redalert{Negationsnormalform} (\alert{NNF}) wenn
			\begin{enumerate}[(a)]
				\item sie nur die Junktoren $\wedge$, $\vee$ und $\neg$ enthält und
				\item der Junktor $\neg$ nur direkt vor Atomen vorkommt (d.h. nur in Teilformeln der Form $\neg p$ mit $p\in\Slang{P}$).
			\end{enumerate}
		}\medskip
		
		Formeln, die negierte oder nichtnegierte Atome sind, nennt man \redalert{Literale}.
		In NNF darf Negation also nur in Literalen auftauchen.
		\pause
		
		\examplebox{Beispiele:
			\begin{itemize}
				\item $(\neg p\wedge q)\vee (p\wedge\neg q)$ ist in NNF
				\item $(b\wedge b)\vee\neg(b\wedge b)$ ist nicht in NNF
				\item $q\vee \neg\neg p$ ist nicht in NNF
				\item $p\leftrightarrow p$ ist nicht in NNF
		\end{itemize}}
		
	\end{frame}

	\begin{frame}\frametitle{Konjunktive und Disjunktive Normalform}
		\footnotesize
		
		\defbox{Eine Formel $F$ ist in \redalert{konjunktiver Normalform} (\alert{KNF}) wenn sie 
			eine Konjunktion von Disjunktionen von Literalen ist, d.h. wenn sie die Form hat:\\[1ex]
			%
			\narrowcentering{$(L_{1,1}\vee \ldots\vee L_{1,m_1})\wedge(L_{2,1}\vee \ldots\vee L_{2,m_2})\wedge\ldots
				\wedge(L_{n,1}\vee \ldots\vee L_{n,m_n}) $}\\[1ex]
			% \narrowcentering{$\bigwedge_{i=1}^n \bigvee_{j=1}^{m_i} L_{i,j}$}\\[1ex]
			%
			wobei die Formeln $L_{i,j}$ Literale sind. Eine Disjunktion von Literalen heißt \redalert{Klausel}.
		}\medskip
		
		\defbox{Eine Formel $F$ ist in \redalert{disjunktiver Normalform} (\alert{DNF}) wenn sie 
			eine Disjunktion von Konjunktionen von Literalen ist, d.h. wenn sie die Form hat:\\[1ex]
			%
			\narrowcentering{$(L_{1,1}\wedge \ldots\wedge L_{1,m_1})\vee(L_{2,1}\wedge \ldots\wedge L_{2,m_2})\vee\ldots
				\vee(L_{n,1}\wedge \ldots\wedge L_{n,m_n}) $}\\[1ex]
			% \narrowcentering{$\bigwedge_{i=1}^n \bigvee_{j=1}^{m_i} L_{i,j}$}\\[1ex]
			%
			wobei die Formeln $L_{i,j}$ Literale sind. Eine Konjunktion von Literalen heißt \redalert{Monom}.
		}
	\end{frame}

	\newcommand{\colA}[2]{\textcolor<#2>{darkred}{#1}}
	\newcommand{\colB}[2]{\textcolor<#2>{darkblue}{#1}}
	\newcommand{\colC}[2]{\textcolor<#2>{darkgreen}{#1}}
	\begin{frame}\frametitle{KNF und DNF bilden}
		\footnotesize
		Man kann KNF und DNF bilden, indem man die NNF erzeugt und anschließend Distributivgesetze anwendet
		$\leadsto$ oft direkter\bigskip
		
		\textbf{Konjunktive Normalform}
		\medskip
		
		Distributivgesetz: $\colA{F}{2-}\vee(\colB{G}{2-}\wedge \colC{H}{2-}) \equiv (\colA{F}{2-}\vee \colB{G}{2-})\wedge (\colA{F}{2-}\vee \colC{H}{2-})$\pause
		\medskip
		
		\examplebox{Beispiel:\\
			$\begin{array}{r@{{}\equiv{}}l}
				\colA{(p\wedge\neg q)}{3}\vee(\colB{\neg p}{3}\wedge\colC{q}{3}) \pause& (\colA{(\colB{p}{4}\wedge\colC{\neg q}{4})}{3}\vee\colA{\colB{\neg p}{3}}{4})\wedge (\colA{(p\wedge\neg q)}{3}\vee \colC{q}{3})) \pause\\
				& (\colB{p}{4}\vee\colA{\neg p}{4})\wedge(\colC{\neg q}{4}\vee\colA{\neg p}{4})\wedge ((\colB{p}{5}\wedge\colC{\neg q}{5})\vee \colA{q}{5}))\pause\\
				& (p\vee\neg p)\wedge(\neg q\vee\neg p)\wedge (\colB{p}{5}\vee \colA{q}{5})\wedge(\colC{\neg q}{5}\vee \colA{q}{5})\pause
			\end{array}$\\[1ex]
			(Man könnte die wahren Klauseln $(p\vee\neg p)$ und $(\neg q\vee q)$ streichen.)
		}\smallskip
		
		\visible<6->{
			\textbf{Disjunktive Normalform}
			\medskip
			
			Distributivgesetz: $F\wedge(G\vee H) \equiv (F\wedge G)\vee (F\wedge H)$ ~~~(analog)
		}
	\end{frame}
	
	\begin{frame} \frametitle{Aufgabe 3}
		\small
		Transformieren Sie die Formel
		\begin{equation*}
			\phi = \Big( \big( \lnot (a \leftrightarrow b) \lor \lnot (c \land a) \big) \lor \lnot \big( c \to b \big) \Big)
		\end{equation*}
	in
	\begin{enumerate}[a)]
		\item Negationsnormalform
		\item konkjunktive Normalform
		\item disjunktive Normalform
	\end{enumerate}
	\end{frame}

	
	\section{Aufgabe 4 \\ \itshape Resolution}
	
	\begin{frame} \frametitle{Aufgabe 4}
		\small
		Prüfen Sie die folgende Formel mittels Resolutionsverfahren auf Erfüllbarkeit:
		\begin{enumerate}[a)]
			\item $b \land (a \lor b) \land (\lnot b \lor c) \land (\lnot b \lor \lnot c) \land (\lnot a \lor c)$
			\item $\lnot \Big( c \to \big( (\lnot a \land b \land c) \lor (a \land \lnot b) \big) \Big)$
		\end{enumerate}
	\end{frame}
	
	

\end{document}