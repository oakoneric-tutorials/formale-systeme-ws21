\documentclass{beamer}
\usepackage{../tut-slides}
\usepackage{../mathoperators}
\usepackage{../fs}

\usepackage{csquotes}

\usepackage{amsmath,amssymb}
%\usepackage{enumerate}
\usepackage[normalem]{ulem}
\newcommand{\labelitemi}{\raisebox{1pt}{\scalebox{.9}{$\blacktriangleright$}}}
\newcommand{\labelitemii}{$\vartriangleright$}
\newcommand{\labelitemiii}{--}

\usepackage{booktabs}
\usepackage{tabularx}

\newcommand{\tuple}[1]{\langle{#1}\rangle}
\newcommand{\simquot}[1]{#1/_{\!\!{\sim}}}

\usepackage{pifont}

\newcommand{\cmark}{\textcolor{cddarkgreen}{\ding{51}}}%
\newcommand{\xmark}{\textcolor{darkred}{\ding{55}}}%

\newcommand{\solutionmarkB}{%
	\begin{tikzpicture}[remember picture, overlay]
		\node [
			fill=none,  % Farbe des Randstreifens
			text=cdorange,  % Textfarbe
			font=\fosfamily\bfseries\large,  % Einstellungen für die Schrift
			inner xsep=0,       % Abstand des Textes von unten
			% maximale Textbreite = Papierhöhe - 2*Abstand des Textes von unten:
			%			text width={\dimexpr\paperheight-2\footskip\relax},
			align=center,
			%			minimum height=8mm,% Breite des Randstreifens
			anchor=south west,
			rotate=90
		] at ($(current page.north west)+(+8mm,-20mm)$)
		{Lösung};
		\draw[draw, line width=2pt, color=cdorange] ($(current page.north west)+(+1mm,0)$) -- ($(current page.south west)+(+1mm,0)$);
	\end{tikzpicture}%
}
\newcommand{\solutionmarkFT}{\begin{tikzpicture}[remember picture, overlay]
	\node [
	fill=none,  % Farbe des Randstreifens
	text=cdorange,  % Textfarbe
	font=\fosfamily\bfseries\large,  % Einstellungen für die Schrift
	inner xsep=0,       % Abstand des Textes von unten
	% maximale Textbreite = Papierhöhe - 2*Abstand des Textes von unten:
	%			text width={\dimexpr\paperheight-2\footskip\relax},
	align=center,
	%			minimum height=8mm,% Breite des Randstreifens
	anchor=south west,
	rotate=90
	] at ($(current page.north west)+(+8mm,-27mm)$)
	{Lösung};
	\draw[draw, line width=2pt, color=cdorange] ($(current page.north west)+(+1mm,0)$) -- ($(current page.south west)+(+1mm,0)$);
	\end{tikzpicture}%
}


%%%%%%%%%%%%%%%%%%%%%%%%%%%%%%%%%%%%%%%%%%%%%%%%%%%%%%%%%%%%%%%%%%%%%%%
\newcommand{\ghost}[1]{\raisebox{0pt}[0pt][0pt]{\makebox[0pt][l]{#1}}}
\newcommand{\blue}[1]{\textcolor{darkblue}{#1}}
\newcommand{\purple}[1]{\textcolor{cdpurple}{#1}}
\newcommand{\green}[1]{\textcolor{cddarkgreen}{#1}}
\newcommand{\orange}[1]{\textcolor{cdorange}{#1}}
\newcommand{\indigo}[1]{\textcolor{cdindigo}{#1}}

\usepackage{ragged2e}

\newcommand{\cellblue}{\cellcolor{cdblue!20}}
\newcommand{\cellred}{\cellcolor{cdpurple!20}}


\begin{document}	
	\title{Formale Systeme}
	\subtitle{Übung 12}
	\author{Eric Kunze}
	\email{eric.kunze@tu-dresden.de}
	\city{TU Dresden}
	\date{\formatdate{21}{1}{2022}}
%	\institute{Lehrstuhl für Grundlagen der Programmierung}
	\titlegraphic{\includegraphics[width=2cm]{../TUD-white.pdf}}

	\maketitle


	\begin{frame} \frametitle{Übungsblatt 12}
		\tableofcontents
	\end{frame}

	\section{Logische Äquivalenzen}

	\begin{frame}\frametitle{Logische Äquivalenz}
		\small
		Formeln sind äquivalent, wenn sie die gleiche Semantik haben:
		
		\defbox{Zwei Formeln $F$ und $G$ sind \redalert{semantisch äquivalent}, in Symbolen \redalert{$F\equiv G$},
			wenn sie genau die selben Modelle haben, d.h. wenn\\[1ex]
			
			\narrowcentering{für alle Wertzuweisungen $w$ gilt: $w(F)=w(G)$}}
	\end{frame}

	\begin{frame}\frametitle{Junktoren äquivalent ausdrücken}
		\footnotesize
		\begin{align*}
			F\to G & \equiv \neg F\vee G \equiv \neg(F\wedge\neg G) \\
			F\leftrightarrow G & \equiv (F\to G)\wedge(G\to F)\equiv (F\wedge G)\vee(\neg F\wedge \neg G) \\
			F\wedge G & \equiv \neg(\neg F\vee\neg G) \quad \text{\textcolor{devilscss}{(De Morgansches Gesetz)}} \\
			F\vee G & \equiv \neg(\neg F\wedge\neg G) \quad \text{\textcolor{devilscss}{(De Morgansches Gesetz)}}
		\end{align*}\pause
		
		\theobox{Satz: Sei $F$ eine beliebige aussagenlogische Formel.
			\begin{itemize}
				\item Es gibt eine zu $F$ äquivalente Formel, die nur die Junktoren $\wedge$ und $\neg$ enthält.
				\item Es gibt eine zu $F$ äquivalente Formel, die nur die Junktoren $\vee$ und $\neg$ enthält.
		\end{itemize}}		
	\end{frame}

	
	\begin{frame}\frametitle{Nützliche Äquivalenzen}
		\small
		\begin{align*}
			\begin{split}
				F\wedge G & \equiv G\wedge F\\
				F\vee G & \equiv G\vee F
			\end{split}
			& \text{\textcolor{devilscss}{Kommutativität}} \\[1ex]
			%
			\begin{split}
				(F\wedge G)\wedge H & \equiv F\wedge (G\wedge H)\\
				(F\vee G)\vee H & \equiv F\vee (G\vee H)
			\end{split}
			& \text{\textcolor{devilscss}{Assoziativität}} \\[1ex]
			%
			\begin{split}
				F\wedge (G\vee H) & \equiv (F\wedge G) \vee (F\wedge H)\\
				F\vee (G\wedge H) & \equiv (F\vee G)\wedge (F\vee H)
			\end{split}
			& \text{\textcolor{devilscss}{Distributivität}} \\[1ex]
			%
			\begin{split}
				F\wedge F & \equiv F\\
				F\vee F & \equiv F
			\end{split}
			& \text{\textcolor{devilscss}{Idempotenz}} \\[1ex]
			%
			\begin{split}
				F\wedge (F\vee G) & \equiv F\\
				F\vee (F\wedge G) & \equiv F
			\end{split}
			& \text{\textcolor{devilscss}{Absorption}}
		\end{align*}
		
	\end{frame}
	
	\begin{frame}\frametitle{Nützliche Äquivalenzen}
		\small
		\begin{align*}
			\neg\neg F &\equiv F
			& \text{\textcolor{devilscss}{doppelte Negation}}\\[1ex]
			%
			\begin{split}
				\neg(F\wedge G) & \equiv (\neg F\vee \neg G)\\
				\neg(F\vee G) & \equiv (\neg F\wedge \neg G)
			\end{split}
			& \text{\textcolor{devilscss}{De Morgansche Gesetze}} \\[1ex]
			%
			\begin{split}
				F\wedge \top & \equiv F\\
				F\vee \top & \equiv \top
			\end{split}
			& \text{\textcolor{devilscss}{Gesetze mit $\top$}}\\[1ex]
			%
			\begin{split}
				F\wedge \bot & \equiv \bot\\
				F\vee \bot & \equiv F
			\end{split}
			& \text{\textcolor{devilscss}{Gesetze mit $\bot$}}\\[1ex]
			%
			\begin{split}
				\neg\top & \equiv \bot\\
				\neg\bot & \equiv \top
			\end{split}
			& 
		\end{align*}
		
		Alle diese Äquivalenzen können leicht mit Wahrheitswertetabellen überprüft werden.
	\end{frame}


	\section{Aufgabe 1 \\ \itshape weitere Äquivalenzen}
	
	\begin{frame} \frametitle{Aufgabe 1}
		\small
		Zeigen Sie die Gültigkeit der folgenden Äquivalenzen:
		\begin{enumerate}[a)]
			\item Distributivitätsregel:
			\begin{equation*}
				(\phi \lor (\psi \land \pi)) \equiv ((\phi \lor \psi) \land (\phi \lor \pi))
			\end{equation*}
			\item Absorptionsregel:
			\begin{equation*}
				(\phi \land (\phi \lor \psi)) \equiv \phi
			\end{equation*}
		\end{enumerate}
	\end{frame}

	\begin{frame}
		\small
		\textbf{(a)} Beweis mittels Wahrheitswertetabelle:
		\begin{center}
			\begin{tabular}{c|c|c||c|c||c|c|c}
				\hline
				$\phi$ & $\psi$ & $\pi$ & $\psi \land \pi$ & $\phi \lor (\psi \land \pi)$ & $\phi \lor \psi$ & $\phi \lor \pi$ & $(\phi \lor \psi) \land (\phi \lor \pi)$ \\
				\hline
				0 & 0 & 0 & 0 & \cellblue 0 & 0 & 0 & \cellblue 0 \\
				0 & 0 & 1 & 0 & \cellblue 0 & 0 & 1 & \cellblue 0 \\
				0 & 1 & 0 & 0 & \cellblue 0 & 1 & 0 & \cellblue 0 \\
				0 & 1 & 1 & 1 & \cellblue 1 & 1 & 1 & \cellblue 1 \\
				1 & 0 & 0 & 0 & \cellblue 1 & 1 & 1 & \cellblue 1 \\
				1 & 0 & 1 & 0 & \cellblue 1 & 1 & 1 & \cellblue 1 \\
				1 & 1 & 0 & 0 & \cellblue 1 & 1 & 1 & \cellblue 1 \\
				1 & 1 & 1 & 1 & \cellblue 1 & 1 & 1 & \cellblue 1 \\ \hline
			\end{tabular}
		\end{center}
		Damit ist also $w(\phi \lor (\psi \land \pi)) = w((\phi \lor \psi) \land (\phi \lor \pi))$ für alle Wertzuweisungen $w$. Per Definition gilt also $(\phi \lor (\psi \land \pi)) \equiv ((\phi \lor \psi) \land (\phi \lor \pi))$.
		
		\solutionmarkB
	\end{frame}

	\begin{frame}
		\small
		\textbf{(b)} Beweis mittels Wahrheitswertetabelle:
		\begin{center}
			\begin{tabular}{c|c||c|c}
				\hline
				$\phi$ & $\psi$ & $\phi \lor \psi$ & $\phi \land (\phi \lor \psi)$ \\
				\hline
				\cellblue 0 & 0 & 0 & \cellblue 0 \\
				\cellblue 0 & 1 & 1 & \cellblue 0 \\
				\cellblue 1 & 0 & 1 & \cellblue 1 \\
				\cellblue 1 & 1 & 1 & \cellblue 1 \\ \hline
			\end{tabular}
		\end{center}
		Damit ist also $w(\phi \land (\phi \lor \psi))) = w(\phi)$ für alle Wertzuweisungen $w$. Per Definition gilt also $(\phi \land (\phi \lor \psi)) \equiv \phi$.
		\solutionmarkB
	\end{frame}
	

	\section{Aufgabe 2 \\ \itshape mehr Äquivalenzen}

	\begin{frame} \frametitle{Aufgabe 2}
		\small
		Prüfen Sie, ob die folgenden Äquivalenzen gelten.
		\begin{enumerate}[(a)]
			\item $(((a \to \lnot b) \land (\lnot a \to (b \land c))) \land ((\lnot b \lor c) \to d))$ \\
			$\equiv$ \\
			$((\lnot (a \leftrightarrow b) \land (a \lor c)) \land  \lnot ((b \lor d) \to (c \land \lnot d)))$ 
			\medskip
			\item $\Big(  \big( a \to b) \to a \big) \to a \Big)$ \\
			$\equiv$ \\
			$\Big( \big( (a \to b) \land (b \to c) \big) \to \lnot (\lnot c \land a) \Big)$
			\medskip
			\item $(((b \land \ell) \to m) \land ((a \land b) \to \ell) \land a \land b)$ \\
			$\equiv$ \\
			$((\lnot b \land \ell \land \lnot a \land b) \lor (\lnot \ell \land \ell \land \lnot a \land b) \lor (m \land \ell \land a \land b))$
		\end{enumerate}
	\end{frame}

	\section{Aufgabe 3 \\ \itshape Normalformen}
	
	\begin{frame}\frametitle{Negationsnormalform}
		\small
		\defbox{Eine Formel $F$ ist in \redalert{Negationsnormalform} (\alert{NNF}) wenn
			\begin{enumerate}[(a)]
				\item sie nur die Junktoren $\wedge$, $\vee$ und $\neg$ enthält und
				\item der Junktor $\neg$ nur direkt vor Atomen vorkommt (d.h. nur in Teilformeln der Form $\neg p$ mit $p\in\Slang{P}$).
			\end{enumerate}
		}\medskip
		
		Formeln, die negierte oder nichtnegierte Atome sind, nennt man \redalert{Literale}.
		In NNF darf Negation also nur in Literalen auftauchen.
		\pause
		
		\examplebox{Beispiele:
			\begin{itemize}
				\item $(\neg p\wedge q)\vee (p\wedge\neg q)$ ist in NNF
				\item $(b\wedge b)\vee\neg(b\wedge b)$ ist nicht in NNF
				\item $q\vee \neg\neg p$ ist nicht in NNF
				\item $p\leftrightarrow p$ ist nicht in NNF
		\end{itemize}}
	\end{frame}

	\begin{frame}\frametitle{Konjunktive und Disjunktive Normalform}
		\footnotesize
		
		\defbox{Eine Formel $F$ ist in \redalert{konjunktiver Normalform} (\alert{KNF}) wenn sie 
			eine Konjunktion von Disjunktionen von Literalen ist, d.h. wenn sie die Form hat:\\[1ex]
			%
			\narrowcentering{$(L_{1,1}\vee \ldots\vee L_{1,m_1})\wedge(L_{2,1}\vee \ldots\vee L_{2,m_2})\wedge\ldots
				\wedge(L_{n,1}\vee \ldots\vee L_{n,m_n}) $}\\[1ex]
			% \narrowcentering{$\bigwedge_{i=1}^n \bigvee_{j=1}^{m_i} L_{i,j}$}\\[1ex]
			%
			wobei die Formeln $L_{i,j}$ Literale sind. Eine Disjunktion von Literalen heißt \redalert{Klausel}.
		}\medskip
		
		\defbox{Eine Formel $F$ ist in \redalert{disjunktiver Normalform} (\alert{DNF}) wenn sie 
			eine Disjunktion von Konjunktionen von Literalen ist, d.h. wenn sie die Form hat:\\[1ex]
			%
			\narrowcentering{$(L_{1,1}\wedge \ldots\wedge L_{1,m_1})\vee(L_{2,1}\wedge \ldots\wedge L_{2,m_2})\vee\ldots
				\vee(L_{n,1}\wedge \ldots\wedge L_{n,m_n}) $}\\[1ex]
			% \narrowcentering{$\bigwedge_{i=1}^n \bigvee_{j=1}^{m_i} L_{i,j}$}\\[1ex]
			%
			wobei die Formeln $L_{i,j}$ Literale sind. Eine Konjunktion von Literalen heißt \redalert{Monom}.
		}
	\end{frame}

	\newcommand{\colA}[2]{\textcolor<#2>{darkred}{#1}}
	\newcommand{\colB}[2]{\textcolor<#2>{darkblue}{#1}}
	\newcommand{\colC}[2]{\textcolor<#2>{darkgreen}{#1}}
	\begin{frame}\frametitle{KNF und DNF bilden}
		\footnotesize
		Man kann KNF und DNF bilden, indem man die NNF erzeugt und anschließend Distributivgesetze anwendet
		$\leadsto$ oft direkter\bigskip
		
		\textbf{Konjunktive Normalform}
		\medskip
		
		Distributivgesetz: $\colA{F}{2-}\vee(\colB{G}{2-}\wedge \colC{H}{2-}) \equiv (\colA{F}{2-}\vee \colB{G}{2-})\wedge (\colA{F}{2-}\vee \colC{H}{2-})$\pause
		\medskip
		
		\examplebox{Beispiel:\\
			$\begin{array}{r@{{}\equiv{}}l}
				\colA{(p\wedge\neg q)}{3}\vee(\colB{\neg p}{3}\wedge\colC{q}{3}) \pause& (\colA{(\colB{p}{4}\wedge\colC{\neg q}{4})}{3}\vee\colA{\colB{\neg p}{3}}{4})\wedge (\colA{(p\wedge\neg q)}{3}\vee \colC{q}{3})) \pause\\
				& (\colB{p}{4}\vee\colA{\neg p}{4})\wedge(\colC{\neg q}{4}\vee\colA{\neg p}{4})\wedge ((\colB{p}{5}\wedge\colC{\neg q}{5})\vee \colA{q}{5}))\pause\\
				& (p\vee\neg p)\wedge(\neg q\vee\neg p)\wedge (\colB{p}{5}\vee \colA{q}{5})\wedge(\colC{\neg q}{5}\vee \colA{q}{5})\pause
			\end{array}$\\[1ex]
			(Man könnte die wahren Klauseln $(p\vee\neg p)$ und $(\neg q\vee q)$ streichen.)
		}\smallskip
		
		\visible<6->{
			\textbf{Disjunktive Normalform}
			\medskip
			
			Distributivgesetz: $F\wedge(G\vee H) \equiv (F\wedge G)\vee (F\wedge H)$ ~~~(analog)
		}
	\end{frame}
	
	\begin{frame} \frametitle{Aufgabe 3}
		\small
		Transformieren Sie die Formel
		\begin{equation*}
			\phi = \Big( \big( \lnot (a \leftrightarrow b) \lor \lnot (c \land a) \big) \lor \lnot \big( c \to b \big) \Big)
		\end{equation*}
		in
		\begin{enumerate}[a)]
			\item Negationsnormalform
			\item konkjunktive Normalform
			\item disjunktive Normalform
		\end{enumerate}
	\end{frame}

	\begin{frame} 
		\small
		\textbf{Vorbereitung:} Eliminiere $\leftrightarrow$ und $\to$
		\begin{equation*}
			\phi \equiv \Big( \big( \lnot ((a \land b) \lor (\lnot a \land \lnot b)) \lor \lnot (c \land a) \big) \lor \lnot (\lnot c \lor b) =: \phi_1
		\end{equation*}
		\textbf{(a) Negationsnormalform:}
		\begin{align*}
			\phi_1 &= \Big( \big( \lnot ((a \land b) \lor (\lnot a \land \lnot b)) \lor \lnot (c \land a) \big) \lor \blue{\lnot (\lnot c \lor b)} \Big) \\
			 &\equiv \Big( \big( \lnot ((a \land b) \lor (\lnot a \land \lnot b)) \lor \green{\lnot (c \land a)} \big) \lor \blue{(c \land \lnot b)} \Big) \\
			 &\equiv \Big( \big( \purple{\lnot ((a \land b) \lor (\lnot a \land \lnot b))} \lor \green{(\lnot c \lor \lnot a)} \big) \lor \blue{(c \land \lnot b)} \Big) \\
			 &\equiv \Big( \big( \purple{(\orange{\lnot (a \land b)} \land \orange{\lnot (\lnot a \land \lnot b)})} \lor (\lnot c \lor \lnot a) \big) \lor (c \land \lnot b) \Big) \\
			 &\equiv \Big( \big( (\orange{(\lnot a \lor \lnot b)} \land \orange{(a \lor b)}) \lor (\lnot c \lor \lnot a) \big) \lor (c \land \lnot b) \Big) =: \phi_2 \\
		\end{align*}
		\solutionmarkB
	\end{frame}

	\begin{frame} 
		\small
		\textbf{(b) konjunktive Normalform:}
		\begin{align*}
		\phi_2 &= \Big( \big( ((\lnot a \lor \lnot b) \orange{\land} (a \lor b)) \orange{\lor} (\lnot c \lor \lnot a) \big) \lor (c \land \lnot b) \Big) \\
		&\equiv \Big( \big( ((\lnot a \lor \lnot b) \lor (\lnot c \lor \lnot a)) \land ((a \lor b) \lor (\lnot c \lor \lnot a))  \big) \lor (c \land \lnot b) \Big) \\
		&\equiv \Big( \big( (\lnot a \lor \lnot b \lor \lnot c \lor \lnot a) \land \underbrace{(a \lor b \lor \lnot c \lor \lnot a)}_{\equiv \top} \big) \lor (c \land \lnot b) \Big) \\
		&\equiv \Big( (\lnot a \lor \lnot b \lor \lnot c) \orange{\lor} (c \orange{\land} \lnot b) \Big) \\
		&\equiv \Big( \underbrace{\big( (\lnot a \lor \lnot b \lor \lnot c) \lor c \big)}_{\equiv \top}  \land \big( ( \lnot a \lor \lnot b \lor \lnot c) \lor \lnot b \big) \Big) \\
		&\equiv \Big( \lnot a \lor \lnot b \lor \lnot c \Big) =: \phi_3
		\end{align*}
		
		\textbf{(c) disjunktive Normalform:} $\phi_3$ ist bereits auch in disjunktiver Normalform. Mit den Distributivgesetzen erhält man
		\begin{equation*}
			\phi_4 = \Big( (\lnot a \land b) \lor (a \land \lnot b) \lor (\lnot c) \lor (\lnot a) \lor (\lnot b \land c) \Big)
		\end{equation*}
		in disjunktiver Normalform.
		\solutionmarkB
	\end{frame}

	
	\section{Aufgabe 4 \\ \itshape Resolution}
	
	\begin{frame} \frametitle{Klauselform \& Resolution}
		\small
		\begin{itemize}
			\item Eine Klausel $L_1\vee\ldots\vee L_n$ wir dargestellt als Menge $\{L_1,\ldots,L_n\}$
			\item Eine Konjunktion von Klauseln $K_1\wedge\ldots\wedge K_\ell$ wird dargestellt als Menge $\{K_1,\ldots,K_\ell\}$
		\end{itemize}
		Eine Menge von Mengen von Literalen unter dieser Interpretation heißt \redalert{Klauselform}.
		\medskip\pause
		
		\defbox{
			Gegeben seien zwei Klauseln $K_1$ und $K_2$ für die es ein Atom $p \in \Slang{P}$ gibt mit $p \in K_1$ und $\neg p \in K_2$. \\
			Die \redalert{Resolvente von $K_1$ und $K_2$ bezüglich $p$} ist die Klausel 
			\begin{equation*}
				(K_1\setminus\{p\})\cup(K_2\setminus\{\neg p\})
			\end{equation*}
			
			Eine Klausel $R$ ist eine \redalert{Resolvente einer Klauselmenge $\mathcal{K}$} wenn $R$ Resolvente zweier Klauseln  $K_1,K_2\in\mathcal{K}$ ist.
		}	
	\end{frame}

	
	\begin{frame}\frametitle{Bedeutung von Resolutionsschritten}
		\small
		Wir betrachten Klauseln $\{L_1,\ldots,L_n,p\}$ und $\{\neg p,M_1,\ldots,M_\ell\}$:\pause
		\begin{itemize}
			\item $\{L_1,\ldots,L_n,p\}\equiv (L_1\vee\ldots\vee L_n\vee p) \equiv \blue{(\neg L_1\wedge\ldots\wedge\neg L_n)\to p}$ \pause
			\item $\{\neg p,M_1,\ldots,M_\ell\}\equiv (\neg p\vee M_1\vee\ldots\vee M_\ell)\equiv \blue{p\to (M_1\vee\ldots\vee M_\ell)}$ \pause
		\end{itemize}
		\medskip
		
		Daraus folgt unmittelbar \blue{$(\neg L_1\wedge\ldots\wedge\neg L_n)\to (M_1\vee\ldots\vee M_\ell)$}\\
		$\leadsto$ dies entspricht der Klausel $\{L_1,\ldots,L_n,M_1,\ldots,M_\ell\}$ \pause
		\bigskip
		
		\theobox{\textbf{Satz:} Wenn $R$ Resolvente der Klauseln $K_1$ und $K_2$ ist, dann gilt $\{K_1,K_2\}\models R$.
		}		
		
		\begin{itemize}
			\item Die leere Klausel ist eine Disjunktionen von $0$ Literalen, also gerade $\bot$ (neutrales Element von $\lor$)
			\item $\bot$ in einer Konjunktion macht die gesamte Formel falsch (unerfüllbar).
			\item Lässt sich $\bot$ ableiten, so ist die Formel unerfüllbar.
		\end{itemize}
	\end{frame}
%	
	\begin{frame}\frametitle{Das Resolutionskalkül}
		\small
		\codebox{\textbf{Resolution}\\
			Gegeben: Formel $\mathcal{F}$ in Klauselform\\
			Gesucht: Ist $\mathcal{F}$ erfüllbar oder unerfüllbar?\\[1ex]
			
			\begin{enumerate}[(1)]
				\item Finde ein Klauselpaar $K_1,K_2\in \mathcal{F}$ mit Resolvente $R\notin \mathcal{F}$
				\item Setze $\mathcal{F}\defeq \mathcal{F}\cup\{R\}$
				\item Wiederhole Schritt (1) und (2) bis keine neuen Resolventen gefunden werden können
				\item Falls $\bot\in \mathcal{F}$, dann gib "`unerfüllbar"' aus;\\andernfalls gib "`erfüllbar"' aus
			\end{enumerate}
		}\pause
		
		\emph{Beobachtung:} Unerfüllbarkeit steht fest, sobald $\bot$ abgeleitet \ghost{wurde}\\
		$\leadsto$ dann kann man das Verfahren frühzeitig abbrechen
		\medskip
		
		Erfüllbarkeit kann dagegen erst erkannt werden, wenn alle Resolventen erschöpfend gebildet worden sind
		
	\end{frame}
	
	\begin{frame} \frametitle{Aufgabe 4}
		\small
		Prüfen Sie die folgende Formel mittels Resolutionsverfahren auf Erfüllbarkeit:
		\begin{enumerate}[a)]
			\item $b \land (a \lor b) \land (\lnot b \lor c) \land (\lnot b \lor \lnot c) \land (\lnot a \lor c)$
			\item $\lnot \Big( c \to \big( (\lnot a \land b \land c) \lor (a \land \lnot b) \big) \Big)$
		\end{enumerate}
	\end{frame}

	\begin{frame} 
		\footnotesize
		\textbf{(a)} Die Formel ist bereits in konjunktiver Normalform. Die zugehörige Klauselform lautet
		\begin{equation*}
			\menge{\menge{b}, \menge{a, b}, \menge{\lnot b, c}, \menge{\lnot b, \lnot c}, {\lnot a, c}}.
		\end{equation*}
		Durchnummerieren liefert:
		
		\begin{enumerate}[(1)]
			\item $\menge{b}$
			\item $\menge{a,b}$
			\item $\menge{\lnot b, c}$
			\item $\menge{\lnot b, \lnot c}$
			\item $\menge{\lnot a, c}$
		\end{enumerate}
		
		Als Resolutionsschritte kann man beispielsweise folgende wählen:
		
		\begin{enumerate}[(1)]			
			\setcounter{enumi}{5}%
			\item \makebox[4cm][l]{$\menge{c}$} (1) + (3)
			\item \makebox[4cm][l]{$\menge{\lnot b}$} (4) + (6)
			\item \makebox[4cm][l]{$\bot$} (1) + (7)
		\end{enumerate}
	
		Da die leere Klausel ableitbar ist, ist die Formel nicht erfüllbar.
		\solutionmarkB
	\end{frame}
	
	\begin{frame} 
		\footnotesize
		\textbf{(b)} Transformation in konjunktive Normalform liefert
		\begin{equation*}
			\big( c \land (a \lor \lnot b \lor \lnot c) \land (\lnot a \lor b) \big)
		\end{equation*}
		Klauselform: $\menge{\menge{c}, \menge{a, \lnot b, \lnot c}, \menge{\lnot a, b}}$
		
		\begin{enumerate}[(1)]
			\item $\menge{c}$
			\item $\menge{a, \lnot b, \lnot c}$
			\item $\menge{\lnot a, b}$
		\end{enumerate}
	
		Als Resolutionsschritte kann man beispielsweise folgende wählen:
	
		\begin{enumerate}[(1)]
			\setcounter{enumi}{5}%
			\item \makebox[4cm][l]{$\menge{a, \lnot b}$} (1) + (2)
			\item \makebox[4cm][l]{$\menge{b, \lnot b, \lnot c}$} (2) + (3)
			\item \makebox[4cm][l]{$\menge{a, \lnot a, \lnot c}$} (2) + (3)
			\item \makebox[4cm][l]{$\menge{b, \lnot b}$} (1) + (5)
			\item \makebox[4cm][l]{$\menge{a, \lnot a}$} (1) + (6)
			\item \makebox[4cm][l]{$\menge{\lnot a, b, \lnot c}$} (3) + (5)						
		\end{enumerate}
		\justifying Man kann diesen Prozess nun noch weiter durchführen, allerdings wird man nie die leere Klausel ableiten können. Die Formel ist daher erfüllbar.
		\solutionmarkB
	\end{frame}

\end{document}