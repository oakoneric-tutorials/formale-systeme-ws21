\documentclass{beamer}
\usepackage{../tut-slides}
\usepackage{../mathoperators}
\usepackage{../fs}

\usepackage{csquotes}

\usepackage{amsmath,amssymb}
%\usepackage{enumerate}
\usepackage[normalem]{ulem}
\newcommand{\labelitemi}{\raisebox{1pt}{\scalebox{.9}{$\blacktriangleright$}}}
\newcommand{\labelitemii}{$\vartriangleright$}
\newcommand{\labelitemiii}{--}

\usepackage{booktabs}
\usepackage{tabularx}

\newcommand{\tuple}[1]{\langle{#1}\rangle}
\newcommand{\simquot}[1]{#1/_{\!\!{\sim}}}

\usepackage{pifont}

\newcommand{\cmark}{\textcolor{cddarkgreen}{\ding{51}}}%
\newcommand{\xmark}{\textcolor{darkred}{\ding{55}}}%

\newcommand{\solutionmarkB}{%
	\begin{tikzpicture}[remember picture, overlay]
		\node [
			fill=none,  % Farbe des Randstreifens
			text=cdorange,  % Textfarbe
			font=\fosfamily\bfseries\large,  % Einstellungen für die Schrift
			inner xsep=0,       % Abstand des Textes von unten
			% maximale Textbreite = Papierhöhe - 2*Abstand des Textes von unten:
			%			text width={\dimexpr\paperheight-2\footskip\relax},
			align=center,
			%			minimum height=8mm,% Breite des Randstreifens
			anchor=south west,
			rotate=90
		] at ($(current page.north west)+(+8mm,-20mm)$)
		{Lösung};
		\draw[draw, line width=2pt, color=cdorange] ($(current page.north west)+(+1mm,0)$) -- ($(current page.south west)+(+1mm,0)$);
	\end{tikzpicture}%
}
\newcommand{\solutionmarkFT}{\begin{tikzpicture}[remember picture, overlay]
	\node [
	fill=none,  % Farbe des Randstreifens
	text=cdorange,  % Textfarbe
	font=\fosfamily\bfseries\large,  % Einstellungen für die Schrift
	inner xsep=0,       % Abstand des Textes von unten
	% maximale Textbreite = Papierhöhe - 2*Abstand des Textes von unten:
	%			text width={\dimexpr\paperheight-2\footskip\relax},
	align=center,
	%			minimum height=8mm,% Breite des Randstreifens
	anchor=south west,
	rotate=90
	] at ($(current page.north west)+(+8mm,-27mm)$)
	{Lösung};
	\draw[draw, line width=2pt, color=cdorange] ($(current page.north west)+(+1mm,0)$) -- ($(current page.south west)+(+1mm,0)$);
	\end{tikzpicture}%
}


%%%%%%%%%%%%%%%%%%%%%%%%%%%%%%%%%%%%%%%%%%%%%%%%%%%%%%%%%%%%%%%%%%%%%%%
\newcommand{\ghost}[1]{\raisebox{0pt}[0pt][0pt]{\makebox[0pt][l]{#1}}}
\newcommand{\blue}[1]{\textcolor{darkblue}{#1}}
\newcommand{\purple}[1]{\textcolor{cdpurple}{#1}}
\newcommand{\green}[1]{\textcolor{cddarkgreen}{#1}}

\usepackage{ragged2e}
\undef\Var
\DeclareMathOperator{\Var}{Var}
\DeclareMathOperator{\Sub}{Sub}


\begin{document}	
	\title{Formale Systeme}
	\subtitle{Übung 11}
	\author{Eric Kunze}
	\email{eric.kunze@tu-dresden.de}
	\city{TU Dresden}
	\date{\formatdate{14}{1}{2022}}
%	\institute{Lehrstuhl für Grundlagen der Programmierung}
	\titlegraphic{\includegraphics[width=2cm]{../TUD-white.pdf}}

	\maketitle


	\begin{frame} \frametitle{Übungsblatt 11}
		\tableofcontents
	\end{frame}

	\section{Aussagenlogik}
	
	\begin{frame}\frametitle{Aussagenlogik: Syntax}
		\small
		Wir betrachten eine abzählbar unendliche Menge $\Slang{P}$ von
		\redalert{atomaren Aussagen} (auch: aussagenlogische Variablen oder Atome)
		\medskip\pause
		
		\defbox{Die Menge der \redalert{aussagenlogischen Formeln} ist induktiv\footnote{Das bedeutet: Die Definition ist selbstbezüglich und soll die kleinste \\ Menge an Formeln beschreiben, die alle Bedingungen erfüllen.} definiert: 
			\begin{itemize}
				\item Jedes Atom $p\in\Slang{P}$ ist eine aussagenlogische Formel \pause
				\item Wenn $F$ und $G$ aussagenlogische Formeln sind, so auch:
				\begin{itemize}
					\item \makebox[2.5cm][l]{Negation:} $\neg F$ \hfill "`nicht $F$"' \pause
					\item \makebox[2.5cm][l]{Konjunktion:} $(F \land G)$ \hfill "`$F$ und $G$"'
					\item \makebox[2.5cm][l]{Disjunktion:} $(F \lor G)$ \hfill "`$F$ oder $G$"'
					\item \makebox[2.5cm][l]{Implikation:} $(F\to G)$ \hfill "`$F$ impliziert $G$"'
					\item \makebox[2.5cm][l]{Äquivalenz:} $(F\leftrightarrow G)$ \hfill "`$F$ ist äquivalent zu $G$"'
				\end{itemize}
			\end{itemize}
		}
	\end{frame}

	\begin{frame}\frametitle{Teilformeln}
		\small
		Wir können Formeln als Wörter über (endlichen Teilmengen aus) dem unendlichen Alphabet
		$\Slang{P}\cup\{\Sterm{\neg},\Sterm{\wedge},\Sterm{\vee},\Sterm{\to},\Sterm{\leftrightarrow},\Sterm{(},\Sterm{)}\}$ sehen:
		\begin{equation*}
		\Snterm{F}\to \Slang{P}\mid \Sterm{\neg}\Snterm{F}\mid\Sterm{(}\Snterm{F}\Sterm{\wedge}\Snterm{F}\Sterm{)} \mid\Sterm{(}\Snterm{F}\Sterm{\vee}\Snterm{F}\Sterm{)} \mid\Sterm{(}\Snterm{F}\Sterm{\to}\Snterm{F}\Sterm{)} \mid\Sterm{(}\Snterm{F}\Sterm{\leftrightarrow}\Snterm{F}\Sterm{)}
		\end{equation*}
		Eine \redalert{Teilformel} ist ein Teilwort (Infix) einer Formel, welches selbst eine Formel ist.
		\pause
		\medskip
		
		Alternativ kann man Teilformeln auch rekursiv definieren:
		
		\defbox{Die \redalert{Menge $\operatorname{Sub}(F)$ der Teilformeln} einer Formel $F$ ist definiert als:
			\vspace{-2ex}
			\[ \operatorname{Sub}(F) = \left\{\begin{array}{l@{~~}l@{}}
			\{F\} & \text{falls }F\in\Slang{P} \\
			\{\neg G\}\cup\operatorname{Sub}(G) & \text{falls }F=\neg G \\
			\{(G_1\wedge G_2)\}\cup\operatorname{Sub}(G_1)\cup\operatorname{Sub}(G_2) & \text{falls }F=(G_1\wedge G_2) \\
			\{(G_1\vee G_2)\}\cup\operatorname{Sub}(G_1)\cup\operatorname{Sub}(G_2) & \text{falls }F=(G_1\vee G_2) \\
			\{(G_1\to G_2)\}\cup\operatorname{Sub}(G_1)\cup\operatorname{Sub}(G_2) & \text{falls }F=(G_1\to G_2) \\
			\{(G_1\leftrightarrow G_2)\}\cup\operatorname{Sub}(G_1)\cup\operatorname{Sub}(G_2) & \text{falls }F=(G_1\leftrightarrow G_2)
			\end{array}\right.\]\vspace{-2ex}
		}
	\end{frame}


	
	\begin{frame}\frametitle{Aussagenlogik: Semantik}
		\footnotesize
		Eine \redalert{Wertzuweisung} ist eine Funktion $w:\Slang{P}\to\{\mytrue,\myfalse\}$
		\bigskip\pause
		
		% Die Wahrheit einer Formel kann rekursiv definiert werden:
		
		\defbox{Eine Wertzuweisung $w$ \redalert{erfüllt} eine Formel $F$, in Symbolen \redalert{$w\models F$}, wenn
			eine der folgenden rekursiven Bedingungen gilt:\smallskip
			
			\narrowcentering{\begin{tabular}{rll}
					\rowcolor{darkred!70!gray}
					\textcolor{white}{Form von $F$} & \textcolor{white}{$w\models F$ wenn:} & \textcolor{white}{$w\not\models F$ wenn:}\\
					$F\in \Slang{P}$: & $w(F)=\mytrue$ & $w(F)=\myfalse$\\
					\rowcolor{lightred!20}
					$F=\neg G$ & $w\not\models G$ & $w\models G$\\
					$F=(G_1\wedge G_2)$ & $w\models G_1$ und $w\models G_2$ & $w\not\models G_1$ oder $w\not\models G_2$\\
					\rowcolor{lightred!20}
					$F=(G_1\vee G_2)$ & $w\models G_1$ oder $w\models G_2$ & $w\not\models G_1$ und $w\not\models G_2$\\
					$F=(G_1\to G_2)$ & $w\not\models G_1$ oder $w\models G_2$ & $w\models G_1$ und $w\not\models G_2$\\
					\rowcolor{lightred!20}
					$F=(G_1\leftrightarrow G_2)$ & $w\models G_1$ und $w\models G_2$ & $w\models G_1$ und $w\not\models G_2$\\%[-1ex]
					\rowcolor{lightred!20}
					& \multicolumn{1}{c}{\raisebox{1ex}{oder}} & \multicolumn{1}{c}{\raisebox{1ex}{oder}} \\[-2ex]
					\rowcolor{lightred!20}
					& $w\not\models G_1$ und $w\not\models G_2$ & $w\not\models G_1$ und $w\models G_2$\\[-1ex]
					\\[-1.5ex]
			\end{tabular}}
		}
			Dabei bedeutet "`A oder B"' immer "`A oder B \textit{oder beides}"'.	
	\end{frame}
	
	\begin{frame}\frametitle{Aussagenlogik: Semantik}
		\footnotesize
		Wir können Wertzuweisungen von Atomen auf Formeln erweitern:
		\begin{equation*}
			w(F)=\left\{\begin{array}{rl}\mytrue & \text{falls } w\models F\\
			\myfalse & \text{falls } w\not\models F
			\end{array}\right.
		\end{equation*}
		\pause
		
		\redalert{Wahrheitswertetabellen} illustrieren die Semantik der Junktoren:\medskip
		
		\narrowcentering{%
			\begin{tabular}{cc}
				\rowcolor{lightblue!20}
				$w(F)$ & $w(\neg F)$\\
				\myfalse & \mytrue \\
				\rowcolor{gray!10}
				\mytrue & \myfalse \\
		\end{tabular}}\medskip
		
		\narrowcentering{\begin{tabular}{cccccc}
				\rowcolor{lightblue!20}
				$w(F)$ & $w(G)$ & $w(F\wedge G)$ & $w(F\vee G)$ & $w(F\to G)$ & $w(F\leftrightarrow G)$\\
				\myfalse & \myfalse & \myfalse & \myfalse& \mytrue & \mytrue\\
				\rowcolor{gray!10}
				\mytrue & \myfalse & \myfalse & \mytrue& \myfalse & \myfalse\\
				\myfalse & \mytrue & \myfalse & \mytrue& \mytrue & \myfalse\\
				\rowcolor{gray!10}
				\mytrue & \mytrue & \mytrue & \mytrue& \mytrue & \mytrue\\
		\end{tabular}}
	\end{frame}
	
	

	\section{Aufgabe 1: \\ \itshape Logisches Rätsel}

	\begin{frame} \frametitle{Aufgabe 1}
		\footnotesize\justifying
		Die Menschen sagen stets die Wahrheit, die Vampire lügen stets.  
		
		Außerdem ist ein Teil der Bevölkerung Transsilvaniens verrückt: Alles, was wahr ist, glauben sie, sei falsch, und umgekehrt. Nicht verrückte Transsilvanier hingegen halten genau das für wahr, was wahr ist. 
		
		Insbesondere sagt ein verrückter Vampir (wie auch ein nicht-verrückter Mensch) stets das Richtige: Eine Aussage, die wahr ist, glaubt er, sei falsch, da er aber stets lügt, gibt er dennoch eine richtige Antwort.
		
		Craig verhört die zwei Beschuldigten Lucy und Minna, von denen bekannt ist,	dass eine ein Vampir ist und die andere nicht. Das Verhör geht wie folgt	vonstatten:
		
		\begin{quote}
			\emph{Craig} (zu Lucy): Erzählen Sie mir von Ihnen.\\
			\emph{Lucy}: Wir sind beide verrückt.\\
			\emph{Craig} (zu Minna): Ist das richtig?\\
			\emph{Minna}: Natürlich nicht!
		\end{quote}
	
		Formalisieren Sie dieses Szenario mithilfe aussagenlogischer Formeln und
		finden Sie heraus, wer der Vampir ist!
	\end{frame}

	\begin{frame}
		\footnotesize \setlength\tabcolsep{2pt}
		Wir führen folgende aussagenlogische Variablen ein:
		\begin{tabular}{ccl c ccl}
			$V_L$ & $\dots$ & Lucy ist ein Vampir  & & $T_L$ & $\dots$ & Lucy sagt immer die Wahrheit \\
			$V_M$ & $\dots$ & Minna ist ein Vampir & & $T_M$ & $\dots$ & Minna sagt immer die Wahrheit \\
			$C_L$ & $\dots$ & Lucy ist verrückt    & & $B_L$ & $\dots$ & Lucys Behauptung ist wahr \\
			$C_M$ & $\dots$ & Lucy ist verrückt    & & $B_M$ & $\dots$ & Minnas Behauptung ist wahr
		\end{tabular}
		\pause
		\makebox[7cm][l]{Entweder Lucy oder Minna ist ein Vampir:}  $V_L \leftrightarrow \lnot V_M$ \hfill (1)
				
		\makebox[7cm][l]{Wir sind beide verrückt:} $B_L \leftrightarrow C_L \land C_M$ \hfill (2)
		
		\makebox[7cm][l]{Natürlich nicht:} $B_M \leftrightarrow \lnot B_L$ \hfill (3)
		
		Lucy bzw. Minna sagt die Wahrheit:
		\begin{align*}
			T_L &\leftrightarrow \Big( (V_L \land C_L) \lor (\lnot V_L \land \lnot C_L) \Big) \tag{4} \\ 
			T_M &\leftrightarrow \Big( (V_M \land C_M) \lor (\lnot V_M \land \lnot C_M) \Big) \tag{5}
		\end{align*}
		
		Interessant sind nun folgende Aussagen:
		\begin{itemize}
			\item Lucy sagt genau dann die Wahrheit, wenn beide verrückt sind: \\ \hfill 
			$F_1 \defeq T_L \leftrightarrow B_L$ \hfill (6)
			\item Minna sagt genau dann die Wahrheit, wenn eine von beiden nicht verrückt ist: \\[-\baselineskip] \hfill
			$F_2 \defeq T_M \leftrightarrow B_M$ \hfill (7)
		\end{itemize}
		\solutionmarkB
	\end{frame}
	
	\begin{frame}
		\footnotesize
		Wir wollen untersuchen, ob es eine Möglichkeit gibt, in der $F_1$ und $F_2$ beide wahr werden sich daraus schlussfolgern lässt, ob $V_L$ oder $V_M$ wahr sein muss -- damit ist die Vampirfrage entschieden:
		\begin{center}
			\begin{tabular}{cc|cc|cc|cc|cc}
				\hline
				$V_L$ & $V_M$ & $C_L$ & $C_M$ & $T_L$ & $T_M$ & $B_L$ & $B_M$ & $F_1$ & $F_2$ \\
				\hline
				 0 & 1 & 0 & 0 & 1 & 0 & 0 & 1 & 0 & 0 \\
				 0 & 1 & 0 & 1 & 1 & 1 & 0 & 1 & 0 & 1 \\
				 0 & 1 & 1 & 0 & 0 & 0 & 0 & 1 & 1 & 0 \\
				 0 & 1 & 1 & 1 & 0 & 1 & 1 & 0 & 0 & 0 \\
				 \alert{1} & 0 & 0 & 0 & 0 & 1 & 0 & 1 & \alert{1} & \alert{1} \\
				 1 & 0 & 0 & 1 & 0 & 0 & 0 & 1 & 1 & 0 \\
				 1 & 0 & 1 & 0 & 1 & 1 & 0 & 1 & 0 & 1 \\
				 \alert{1} & 0 & 1 & 1 & 1 & 0 & 1 & 0 & \alert{1} & \alert{1} \\
				 & $\uparrow$ & & & $\uparrow$ & $\uparrow$ & $\uparrow$ & $\uparrow$ & $\uparrow$ & $\uparrow$  \\
				 & (1)        & & & (4)        & (5)        & (2)        & (3)        & (6)        & (7)
			\end{tabular}
		\end{center}
	
		Damit ist Lucy der Vampir.
		\solutionmarkB
	\end{frame}

	\section{Aufgabe 2 \\ \itshape Logische Konsequenzen}
	
	\begin{frame}\frametitle{Logische Konsequenzen}
		\footnotesize
		Sei $F$ eine Formel, $\mathcal{F}$ eine \textit{Menge von Formeln} und $w$ eine Wertzuweisung.
		
		\defbox{ \vspace{-.6\baselineskip}
			\begin{itemize}
				\item $w$ ist Modell von $F$, wenn $w \models F$ gilt.
				\item $w$ ist Modell von $\mathcal{F}$, wenn $w \models F$ für alle $F \in \mathcal{F}$. \\
				Wir schreiben dann auch hier $w \models \mathcal{F}$.
			\end{itemize}
		}
		\pause
		
		Die logischen Schlussfolgerungen aus einer Formel(menge) ergeben sich aus ihren Modellen:
		
		\defbox{Sei $\mathcal{F}$ eine Menge von Formeln.
			Eine Formel $G$ ist eine \redalert{logische Konsequenz} aus $\mathcal{F}$ wenn jedes Modell
			von $\mathcal{F}$ auch ein Modell von $G$ ist.\\ In diesem Fall schreiben wir $\mathcal{F}\models G$.
		}
	\end{frame}

	\begin{frame} \frametitle{Aufgabe 2}
		\small
		Zeigen Sie, welche der folgenden Aussagen gültig sind und welche nicht:
		\begin{enumerate}[(a)]
			\item $\{ (\neg a \lor b), (\neg b \lor c), (b \land c) \}
			\models ((a \leftrightarrow b) \lor c)$
			\item $\{ (a \rightarrow b), (c \lor a), (a \rightarrow \neg b),
			\neg c \} \models a$
			\item $\{ (a \land \neg b) \lor (\neg a \land b), (\neg c
			\land b), \neg (\neg a \lor b) \} \models \neg(a \lor b)$
		\end{enumerate}
	\end{frame}

	\begin{frame}
		\newcommand{\cellblue}{\cellcolor{cdblue!20}}
		\newcommand{\cellred}{\cellcolor{cdpurple!20}}
		\small \vspace{1em}
		\textbf{(a)} $\Big\{ (\neg a \lor b), (\neg b \lor c), (b \land c) \Big\} \models ((a \leftrightarrow b) \lor c)$
		\begin{center}
			\begin{tabular}{c|c|c||c|c|c||c|c}
				\hline
				$a$ & $b$ & $c$ & $\lnot a \lor b$ & $\lnot b \lor c$ & $b \land c$ & $a \leftrightarrow b$ & $(a \leftrightarrow b) \lor c$ \\
				\hline
				0 & 0 & 0 & 1 & 1 & 0 & 1 & \cellred 1 \\
				0 & 0 & 1 & 1 & 1 & 0 & 1 &  \cellred1 \\
				0 & 1 & 0 & 1 & 0 & 0 & 0 & 0 \\
				0 & 1 & 1 & \cellblue 1 & \cellblue 1 & \cellblue 1 & 0 & \cellred 1 \\
				1 & 0 & 0 & 0 & 1 & 0 & 0 & 0 \\
				1 & 0 & 1 & 0 & 1 & 0 & 0 & \cellred 1 \\
				1 & 1 & 0 & 1 & 0 & 0 & 1 & \cellred 1 \\
				1 & 1 & 1 & \cellblue 1 & \cellblue 1 & \cellblue 1 & 1 & \cellred 1 \\ \hline
			\end{tabular}
		\end{center}
	
		Jedes Modell von \colorbox{cdblue!20}{$\Big\{ (\neg a \lor b), (\neg b \lor c), (b \land c) \Big\}$} ist auch Modell von \colorbox{cdpurple!20}{$((a \leftrightarrow b) \lor c)$}. Damit gilt die Aussage.
		\solutionmarkB
	\end{frame}

	\begin{frame}
		\small \vspace{1em}
		\textbf{(b)} $\Big\{ (a \rightarrow b), (c \lor a), (a \rightarrow \neg b), \neg c \Big\} \models a$
		\begin{center}
			\begin{tabular}{c|c|c||c|c|c|c}
				\hline
				$a$ & $b$ & $c$ & $a \rightarrow b$ & $c \lor a$ & $a \rightarrow \lnot b$ & $\lnot c$ \\
				\hline
				0 & 0 & 0 & 1 & 0 & 1 & 1 \\
				0 & 0 & 1 & 1 & 1 & 1 & 0 \\
				0 & 1 & 0 & 1 & 0 & 1 & 1 \\
				0 & 1 & 1 & 1 & 1 & 1 & 0 \\
				1 & 0 & 0 & 0 & 1 & 1 & 1 \\
				1 & 0 & 1 & 0 & 1 & 1 & 0 \\
				1 & 1 & 0 & 1 & 1 & 0 & 1 \\
				1 & 1 & 1 & 1 & 1 & 0 & 0 \\ \hline
			\end{tabular}
		\end{center}
	
		\justifying
		Die Menge $\Big\{ (a \rightarrow b), (c \lor a), (a \rightarrow \neg b), \neg c \Big\}$ hat keine Modelle, d.h. es folgt Beliebiges. Insbesondere ist also jedes (nicht existente) Modell von $\Big\{ (a \rightarrow b), (c \lor a), (a \rightarrow \neg b), \neg c \Big\}$ auch Modell der Formel $a$. Damit gilt die Aussage.
		
		\solutionmarkB
	\end{frame}

	\begin{frame}
		\small \vspace{1em}
		\textbf{(c)} $\Big\{ (a \land \neg b) \lor (\neg a \land b), (\neg c \land b), \neg (\neg a \lor b) \Big\} \models \neg(a \lor b)$
		\begin{center}
			\begin{tabular}{c|c|c||c|c|c||c}
				\hline
				$a$ & $b$ & $c$ & $(a \land \lnot b) \lor (\lnot a \land b)$ & $\lnot c \land b$ & $\lnot (\lnot a \lor b)$ & $\lnot (a \lor b)$ \\
				\hline
				0 & 0 & 0 & 0 & 0 & 0 & 1 \\
				0 & 0 & 1 & 0 & 0 & 0 & 1 \\
				0 & 1 & 0 & 1 & 1 & 0 & 0 \\
				0 & 1 & 1 & 1 & 0 & 0 & 0 \\
				1 & 0 & 0 & 1 & 0 & 1 & 0 \\
				1 & 0 & 1 & 1 & 0 & 1 & 0 \\
				1 & 1 & 0 & 0 & 1 & 0 & 0 \\
				1 & 1 & 1 & 0 & 0 & 0 & 0 \\ \hline
			\end{tabular}
		\end{center}
		
		\justifying
		Die Menge $\Big\{ (a \land \neg b) \lor (\neg a \land b), (\neg c \land b), \neg (\neg a \lor b) \Big\}$ hat keine Modelle, d.h. es folgt Beliebiges. Insbesondere ist also jedes (nicht existente) Modell von $\Big\{ (a \land \neg b) \lor (\neg a \land b), (\neg c \land b), \neg (\neg a \lor b) \Big\}$ auch Modell der Formel $\neg(a \lor b)$. Damit gilt die Aussage.
		
		\solutionmarkB
	\end{frame}

	\section{Aufgabe 3 \\ \itshape Beschränktheit von Variablen und Teilformeln}
	
	\begin{frame}\frametitle{Teilformeln}
		\small
		Wir können Formeln als Wörter über (endlichen Teilmengen aus) dem unendlichen Alphabet
		$\Slang{P}\cup\{\Sterm{\neg},\Sterm{\wedge},\Sterm{\vee},\Sterm{\to},\Sterm{\leftrightarrow},\Sterm{(},\Sterm{)}\}$ sehen:
		\begin{equation*}
			\Snterm{F}\to \Slang{P}\mid \Sterm{\neg}\Snterm{F}\mid\Sterm{(}\Snterm{F}\Sterm{\wedge}\Snterm{F}\Sterm{)} \mid\Sterm{(}\Snterm{F}\Sterm{\vee}\Snterm{F}\Sterm{)} \mid\Sterm{(}\Snterm{F}\Sterm{\to}\Snterm{F}\Sterm{)} \mid\Sterm{(}\Snterm{F}\Sterm{\leftrightarrow}\Snterm{F}\Sterm{)}
		\end{equation*}
		Eine \redalert{Teilformel} ist ein Teilwort (Infix) einer Formel, welches selbst eine Formel ist.
		\pause
		\medskip
		
		Alternativ kann man Teilformeln auch rekursiv definieren:
		
		\defbox{Die \redalert{Menge $\operatorname{Sub}(F)$ der Teilformeln} einer Formel $F$ ist definiert als:
			\vspace{-2ex}
			\[ \operatorname{Sub}(F) = \left\{\begin{array}{l@{~~}l@{}}
			\{F\} & \text{falls }F\in\Slang{P} \\
			\{\neg G\}\cup\operatorname{Sub}(G) & \text{falls }F=\neg G \\
			\{(G_1\wedge G_2)\}\cup\operatorname{Sub}(G_1)\cup\operatorname{Sub}(G_2) & \text{falls }F=(G_1\wedge G_2) \\
			\{(G_1\vee G_2)\}\cup\operatorname{Sub}(G_1)\cup\operatorname{Sub}(G_2) & \text{falls }F=(G_1\vee G_2) \\
			\{(G_1\to G_2)\}\cup\operatorname{Sub}(G_1)\cup\operatorname{Sub}(G_2) & \text{falls }F=(G_1\to G_2) \\
			\{(G_1\leftrightarrow G_2)\}\cup\operatorname{Sub}(G_1)\cup\operatorname{Sub}(G_2) & \text{falls }F=(G_1\leftrightarrow G_2)
			\end{array}\right.\]\vspace{-2ex}
		}
	\end{frame}


	\newcommand{\size}[1]{\abs{#1}}
	
	\begin{frame} \frametitle{Aufgabe 3}
		\small
		Für eine Formel $F$ ist die Größe $\size{F}$ definiert durch:
		\begin{align*}
			\size{p} &:= 1 \\
			\size{\lnot G} &:= \size{G} + 1 \\
			\size{(G_1 \lor G_2)} &:= \size{G_1} + \size{G_2} + 1 \\
			\size{(G_1 \land G_2)} &:= \size{G_1} + \size{G_2} + 1 \\
			\size{(G_1 \rightarrow G_2)} & := \size{G_1} + \size{G_2} + 1 \\
			\size{(G_1 \leftrightarrow G_2)} & := \size{G_1} + \size{G_2} + 1,
		\end{align*}
		wobei $G_1$ und $G_2$ Formeln sind und $p \in \Slang{P}$ ist.  Zeigen Sie
		die folgenden Aussagen:
		\begin{enumerate}[(a)]
			\item Die Anzahl der Variablen in $F$ ist beschränkt durch $\size{F}$.
			\item Die Anzahl der Unterformeln in $F$ ist beschränkt durch $\size{F}$.
		\end{enumerate}
	\end{frame}

	
	\begin{frame}
		\footnotesize
		Wir definieren zunächst die Variablen $\Var(F)$ einer Formel $F$. Sei $p \in \Slang{P}$ eine Variable, $F$ und $G$ seien Formel und $\circ \in \menge{\land, \lor, \rightarrow, \leftrightarrow}$ ein binärer Junktor.
		\begin{align*}
			\Var(p) &\defeq \menge{p}  \tag{Atome} \\
			\Var(\lnot F) &\defeq \Var(F) \tag{Negation} \\
			\Var(F \circ G)  &\defeq \Var(F) \cup \Var(G) \tag{binäre Junktoren}
		\end{align*}
		
		Aus der Vorlesung ist die Menge der Teilformeln schon bekannt:
		\begin{align*}
			\Sub(p) &\defeq \menge{p}  \tag{Atome} \\
			\Sub(\lnot F) &\defeq \menge{\lnot F} \cup \Sub(F) \tag{Negation} \\
			\Sub(F \circ G)  &\defeq \menge{F \circ G} \cup \Sub(F) \cup \Sub(G) \tag{binäre Junktoren}
		\end{align*}
		\solutionmarkB
	\end{frame}

	\begin{frame}
		\footnotesize
		\textbf{(a)} zu zeigen: $\abs{\Var(F)} \le \abs{F}$
		
		Strukturelle Induktion über den Aufbau aussagenlogischer Formeln:
		\begin{itemize}
			\item[IA:] Sei $F = p \in \Slang{P}$ eine Variable. Dann gilt
			\begin{equation*}
			\abs{\Var(F)} = \abs{\Var(p)} = \abs{\menge{p}} = 1 \le 1 = \abs{p} = \abs{F} .
			\end{equation*}
			\item[IV:] Für Formeln $F_1$ und $F_2$ gelte $\abs{\Var(F_1)} \le \abs{F_1}$ und $\abs{\Var(F_2)} \le \abs{F_2}$.
			\item[IS:] Falls $F = \lnot F_1$, dann gilt
			\begin{equation*}
			\abs{\Var(F)} = \abs{\Var(\lnot F_1)} = \abs{\Var(F_1)} \overset{IV}{\le} \abs{F_1} \le \abs{F_1} + 1 = \abs{\lnot F_1} = \abs{F} .
			\end{equation*}
			Falls $F = F_1 \circ F_2$ für $\circ \in \menge{\land, \lor, \rightarrow, \leftrightarrow}$, dann gilt
			\begin{align*}
			\abs{\Var(F)} = \abs{\Var(F_1 \circ F_2)} &= \abs{\Var(F_1) \cup \Var(F_2)} \\
			&\le \abs{\Var(F_1)} + \abs{\Var(F_2)} \\
			\overset{IV}&{\le} \abs{F_1} + \abs{F_2} \\
			&\le \abs{F_1} + \abs{F_2} + 1 \\
			&= \abs{F_1 \circ F_2} \\
			&= \abs{F} .
			\end{align*}
		\end{itemize}
	\solutionmarkB
	\end{frame}

	\begin{frame}
		\footnotesize
		\textbf{(b)} zu zeigen: $\abs{\Sub(F)} \le \abs{F}$
		
		Strukturelle Induktion über den Aufbau aussagenlogischer Formeln:
		\begin{itemize}
			\item[IA:] Sei $F = p \in \Slang{P}$ eine Variable. Dann gilt
			\begin{equation*}
			\abs{\Sub(F)} = \abs{\Sub(p)} = \abs{\menge{p}} = 1 \le 1 = \abs{p} = \abs{F} .
			\end{equation*}
			\item[IV:] Für Formeln $F_1$ und $F_2$ gelte $\abs{\Sub(F_1)} \le \abs{F_1}$ und $\abs{\Sub(F_2)} \le \abs{F_2}$.
			\item[IS:] Falls $F = \lnot F_1$, dann gilt
			\begin{align*}
			\abs{\Sub(F)} = \abs{\Sub(\lnot F_1)} &= \abs{\menge{\lnot F_1} \cup \Sub(F_1)} \\
			&\le \abs{\menge{F_1}} + \abs{\Sub(F_1)} \\
			\overset{IV}&{\le} 1 + \abs{F_1} = \abs{\lnot F_1} = \abs{F} .
			\end{align*}
			Falls $F = F_1 \circ F_2$ für $\circ \in \menge{\land, \lor, \rightarrow, \leftrightarrow}$, dann gilt
			\begin{align*}
			\abs{\Sub(F)} = \abs{\Sub(F_1 \circ F_2)} &= \abs{\menge{F_1 \circ F_2} \cup \Sub(F_1) \cup \Sub(F_2)} \\
			&\le 1 + \abs{\Sub(F_1)} + \abs{\Sub(F_2)} \\
			\overset{IV}&{\le} 1 + \abs{F_1} + \abs{F_2} \\
			&= \abs{F_1 \circ F_2} \\
			&= \abs{F} .
			\end{align*}
		\end{itemize}
		\solutionmarkB
	\end{frame}
	
	

\end{document}