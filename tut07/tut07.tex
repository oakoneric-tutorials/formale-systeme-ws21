\documentclass{beamer}
\usepackage{../tut-slides}
\usepackage{../mathoperators}
\usepackage{../fs}

\usepackage{csquotes}

\usepackage{amsmath,amssymb}
%\usepackage{enumerate}
\usepackage[normalem]{ulem}
\newcommand{\labelitemi}{\raisebox{1pt}{\scalebox{.9}{$\blacktriangleright$}}}
\newcommand{\labelitemii}{$\vartriangleright$}
\newcommand{\labelitemiii}{--}

\usepackage{booktabs}
\usepackage{tabularx}

\newcommand{\tuple}[1]{\langle{#1}\rangle}
\newcommand{\simquot}[1]{#1/_{\!\!{\sim}}}

\usepackage{pifont}

\newcommand{\cmark}{\textcolor{cddarkgreen}{\ding{51}}}%
\newcommand{\xmark}{\textcolor{darkred}{\ding{55}}}%

\newcommand{\solutionmarkB}{%
	\begin{tikzpicture}[remember picture, overlay]
		\node [
			fill=none,  % Farbe des Randstreifens
			text=cdorange,  % Textfarbe
			font=\fosfamily\bfseries\large,  % Einstellungen für die Schrift
			inner xsep=0,       % Abstand des Textes von unten
			% maximale Textbreite = Papierhöhe - 2*Abstand des Textes von unten:
			%			text width={\dimexpr\paperheight-2\footskip\relax},
			align=center,
			%			minimum height=8mm,% Breite des Randstreifens
			anchor=south west,
			rotate=90
		] at ($(current page.north west)+(+8mm,-20mm)$)
		{Lösung};
		\draw[draw, line width=2pt, color=cdorange] ($(current page.north west)+(+1mm,0)$) -- ($(current page.south west)+(+1mm,0)$);
	\end{tikzpicture}%
}
\newcommand{\solutionmarkFT}{\begin{tikzpicture}[remember picture, overlay]
	\node [
	fill=none,  % Farbe des Randstreifens
	text=cdorange,  % Textfarbe
	font=\fosfamily\bfseries\large,  % Einstellungen für die Schrift
	inner xsep=0,       % Abstand des Textes von unten
	% maximale Textbreite = Papierhöhe - 2*Abstand des Textes von unten:
	%			text width={\dimexpr\paperheight-2\footskip\relax},
	align=center,
	%			minimum height=8mm,% Breite des Randstreifens
	anchor=south west,
	rotate=90
	] at ($(current page.north west)+(+8mm,-27mm)$)
	{Lösung};
	\draw[draw, line width=2pt, color=cdorange] ($(current page.north west)+(+1mm,0)$) -- ($(current page.south west)+(+1mm,0)$);
	\end{tikzpicture}%
}

\usepackage{tikzducks}
\newcommand{\duckA}{%
	\scalebox{0.2}{
		\reflectbox{
	\begin{tikzpicture}[baseline=2ex]
	\duck[mask=teal,cape=teal, body=orange!60!yellow]
	\end{tikzpicture}%
	}
	}
}
\newcommand{\duckB}{%
	\scalebox{0.2}{
	\begin{tikzpicture}[baseline=2ex]
	\duck[body=pink,
	unicorn=magenta!60!violet,
	longhair=magenta!60!violet]
	\end{tikzpicture}%
	}
}
\newcommand{\duckC}{%
	\scalebox{0.2}{
	\begin{tikzpicture}[baseline=2ex]
	\duck[santa=red!80!black,
	beard=white!80!brown]
	\end{tikzpicture}%
	}
}
\newcommand{\duckD}{%
	\scalebox{0.2}{
	\begin{tikzpicture}[baseline=2ex]
	\duck[signpost=420, sunglasses=black!90, body=brown!60!white]
	\end{tikzpicture}%
	}
}

%%%%%%%%%%%%%%%%%%%%%%%%%%%%%%%%%%%%%%%%%%%%%%%%%%%%%%%%%%%%%%%%%%%%%%%
\newcommand{\ghost}[1]{\raisebox{0pt}[0pt][0pt]{\makebox[0pt][l]{#1}}}
\newcommand{\blue}[1]{\textcolor{darkblue}{#1}}
\newcommand{\purple}[1]{\textcolor{cdpurple}{#1}}
\newcommand{\green}[1]{\textcolor{cddarkgreen}{#1}}



\begin{document}	
	\title{Formale Systeme}
	\subtitle{Übung 7}
	\author{Eric Kunze}
	\email{eric.kunze@tu-dresden.de}
	\city{TU Dresden}
	\date{\formatdate{3}{12}{2021}}
%	\institute{Lehrstuhl für Grundlagen der Programmierung}
	\titlegraphic{\includegraphics[width=2cm]{../TUD-white.pdf}}

	\maketitle

	\begin{frame} \frametitle{Corona-Umfrage}
		\textbf{Wie stehst du zu Online-Übungen?}
		\begin{itemize}
			\item Ich bin für Online-Übungen. \\
			\textit{nur online}
			\item Ich kann mit Online-Übungen leben. \\ 
			\textit{beides okay, bevorzugt online}
			\item Ich kann mit Präsenzübungen leben. \\
			\textit{beides okay, bevorzugt Präsenz}
			\item Ich bin für Präsenzübungen. \\
			\textit{nur Präsenz}
		\end{itemize}
	
		\begin{center}
			\Large
			\fbox{\url{https://tudvote.tu-dresden.de/88314}}
		\end{center}
	\end{frame}

	\section{Aufgabe 1: \\ \itshape Kontextfrei? Pumping-Lemma?}

	\newcommand{\colstackrel}[3]{\,{\stackrel{\textcolor{#3}{#1}}{\textcolor{#3}{#2}}}\,}
	\newcommand{\gstackrel}[2]{\colstackrel{#1}{#2}{darkgreen}}
	\newcommand{\bstackrel}[2]{\colstackrel{#1}{#2}{darkblue}}
	\newcommand{\rstackrel}[2]{\colstackrel{#1}{#2}{darkred}}
	
	\begin{frame}\frametitle{Pumpen für kontextfreie Sprachen}
		\footnotesize	
		\textbf{Grundidee beim regulären Pumping-Lemma:} \\
		endliche viele Zustände $\leadsto$ Schleifen für lange Wörter notwendig $\leadsto$ Schleifen mehrfach benutzbar
		\pause
		
		\textbf{Grundidee beim kontextfreien Pumping-Lemma:}
		\begin{itemize}
			\item Grammatiken haben nur endlich viele Variablen.
			\item Beim Generieren langer Wörter muss eine Variable zu etwas expandiert werden, das diese Variable nochmal enthält\ghost{:}\\
			\narrowcentering{$\Snterm{S}\Rightarrow\ldots\Rightarrow u\,\underline{\Snterm{A}}\,y \Rightarrow u\,\underline{z}\,y \Rightarrow\ldots\Rightarrow u\,\underline{v\Snterm{A}x}\,y \Rightarrow \ldots \Rightarrow u\,\underline{vwx}\,y$}
			\item Diese Schleife kann beliebig oft durchlaufen werden.
		\end{itemize}
	
		\theobox{\textbf{Satz (Pumping Lemma):}
			Für jede kontextfreie Sprache $\Slang{L}$\\
			gibt es eine Zahl $n\geq 0$, so dass gilt:\\
			~~für jedes Wort $z\in\Slang{L}$ mit $|z|\geq n$\\
			~~gibt es eine Zerlegung $z=uvwxy$ mit $|vx|\geq 1$ und $|vwx|\leq n$, so dass:\\
			~~~~für jede Zahl $k\geq 0$ gilt: $u v^k w x^k y\in\Slang{L}$
		}		
	\end{frame}
	

	\begin{frame} \frametitle{Aufgabe 1}
		\small
		Welche der folgenden Sprachen $L_i$ ist kontextfrei? Zur Begr"undung
		Ihrer Antwort sollten Sie das Pumping-Lemma f"ur kontextfreie
		Sprachen verwenden oder eine entsprechende kontextfreie Grammatik
		angeben.
		\begin{enumerate}[a)]
			\item $L_1 =\{ a^n b^n c^n d^n \in \{a,b,c,d\}^\ast  \mid n\ge 1 \}$
			\item $L_2 =\{ a^mb^n c^pd^q \in \{a,b,c,d\}^\ast  \mid m,n,p,q \ge 1
			\text{ und } m+n = p + q \}$
		\end{enumerate}
	\end{frame}

	\begin{frame}
		\small
		\begin{enumerate}[a)]
			\item $L_1$ ist nicht kontextfrei. Angenommen $L_1$ wäre kontexfrei, dann gilt das Pumping-Lemma mit einer Zahl $n \ge 0$. Wir betrachten das Wort $w = a^n b^n c^n d^n \in L_1$ und eine Zerlegung dessen in $w = uvwxy$ mit $\abs{vx} \ge 1$ und $\abs{vwx} \le n$. 
			\begin{itemize}
				\item Fall 1: $vwx \in \menge{a,b}^+$, d.h. $vwx$ ist ein Teilwort von $a^n b^n$. Dann enthält $u v^2 w x^2 y$ mehr $a$ oder $b$ als $c$ oder $d$. Somit ist $u v^2 w x^2 y \notin L_1$ im Widerspruch zum Pumping-Lemma.
				\item Fall 2: $vwx \in \menge{b,c}^+$, d.h. $vwx$ ist ein Teilwort von $b^n c^n$. Analog zu Fall 1.
				\item Fall 3: $vwx \in \menge{c,d}^+$, d.h. $vwx$ ist ein Teilwort von $c^n d^n$. Analog zu Fall 1.
			\end{itemize}
			\item $L_2$ ist kontextfrei. Betrachte dazu die Grammatik $G = \tuple{V, \Sigma, P, S}$ mit $V = \menge{S, S_1, S_2, S_3, S_4}$ und 
			\begin{equation*}
				P = \left\{
				\begin{array}{lcl}
				S &\to& a S_1 d \\
				S_1 &\to& a S_1 d \mid a S_2 c \mid b S_3 d \mid b S_4 c \\
				S_2 &\to& a S_2 c \mid b S_4 c \\
				S_3 &\to& b S_3 c \mid b S_4 c \\
				S_4 &\to& b S_4 c \mid \epsilon
				\end{array}
				\right\}
			\end{equation*}
		\end{enumerate}
	\solutionmarkB
	\end{frame}

	\section{Aufgabe 2: \\ \itshape Abschlusseigenschaften}
	
	\newcommand{\mytabnote}[2]{\ghost{#1}\hspace{2cm}{\textcolor{cdgray}{(#2)}}}

	
	\begin{frame}\frametitle{Abschluss für kontextfreie Sprachen}
		\small
		\theobox{
			\textbf{Satz:} Wenn $\Slang{L}$, $\Slangsub{L}{1}$ und $\Slangsub{L}{2}$ kontextfreie Sprachen sind, dann beschreiben auch die folgenden Ausdrücke kontextfreie Sprachen:
			
			\begin{enumerate}[(1)]
				\item \mytabnote{$\Slangsub{L}{1}\cup\Slangsub{L}{2}$}{Abschluss unter Vereinigung}
				\item \mytabnote{$\Slangsub{L}{1}\circ\Slangsub{L}{2}$}{Abschluss unter Konkatenation}
				\item \mytabnote{$\Slang{L}^\ast $}{Abschluss unter Kleene-Stern}
			\end{enumerate}
		}
		
		Aber: 
		
		\theobox{
			\textbf{Satz:} Es gibt kontextfreie Sprachen $\Slang{L}$, $\Slangsub{L}{1}$ und $\Slangsub{L}{2}$, so dass die folgenden Ausdrücke keine kontextfreien Sprachen sind:
			
			\begin{enumerate}[(1)]
				\item \mytabnote{$\Slangsub{L}{1}\cap\Slangsub{L}{2}$}{Nichtabschluss unter Schnitt}
				\item \mytabnote{$\overline{\Slang{L}}$}{Nichtabschluss unter Komplement}
			\end{enumerate}
		}
		
	\end{frame}

	\begin{frame} \frametitle{Aufgabe 2}
		\small
		Beweisen Sie mithilfe der Abschlu\ss{}eigenschaften f\"ur kontextfreie Sprachen,
		dass die nachfolgende Sprache $$L_0=\{a^ib^jc^k \mid i=j \;\text{oder}\; j=k\;\text{mit}\;i,j,k\ge 1\}$$
		kontextfrei ist und geben dann eine kontextfreie Grammatik $G_0$ für $L_0$ an mit $L_0=L(G_0)$. Geben Sie Ableitungen für die Wörter $abc$ und $abbcc$ an.
	\end{frame}

	\begin{frame}
		\small
		\onslide<1->{%
			\begin{align*} 
				L_0 
				&= \menge{ a^i b^j c^k : i=j \text{ oder } j=k \text{ mit } i,j,k \ge 1 } \\
				&= \menge{ a^i b^j c^k : i=j \text{ mit } i,j,k \ge 1} \cup \menge{ a^i b^j c^k : j=k \text{ mit } i,j,k \ge 1 } \\
				&= \menge{ a^i b^i c^k : i,k \ge 1} \cup \menge{ a^i b^j c^j : i,j \ge 1 } \\
				&= \underbrace{\menge{ a^i b^i : i \ge 1}}_{\text{kontextfrei}} \underbrace{\menge{c}^\ast}_{\text{regulär}} \cup \underbrace{\menge{a}^\ast}_{\text{regulär}} \underbrace{\menge{b^j c^j : j \ge 1}}_{\text{kontextfrei}} 
			\end{align*}%
		}
		
		\onslide<2->{%
			\textbf{Grammatik} $G_0 = \tuple{V, \Sigma, P, S}$ mit $V = \menge{S, S_a, S_c, S_{ab}, S_{bc}}$ und
			\begin{equation*}
				P = \left\{
				\begin{array}{lcl c lcl}
					S &\to& S_{ab} S_c \mid S_a S_{bc} \\
					S_a &\to& a S_a \mid a &\quad& S_{ab} &\to& a S_{ab} b \mid ab \\
					S_c &\to& c S_c \mid c &\quad& S_{bc} &\to& b S_{bc} c \mid bc \\
				\end{array}
				\right\}
			\end{equation*}%
		}
		
		\onslide<3->{%
			\textbf{Ableitungen}:
			\vspace{-.5\baselineskip}
			\begin{itemize}
				\item $w = abc$:
				\begin{enumerate}[(i)]
					\item $S \to S_{ab} \; S_c \to ab \; S_c \to ab \; c$
					\item $S \to S_a \; S_{bc} \to a \; S_{bc} \to a \; bc$
				\end{enumerate}
				\item $w = abbcc$: $S \to S_a \;¸ S_{bc} \to a \; S_{bc} \to a \; b S_{bc} c \to a \; b \, bc \, c$
			\end{itemize}%
		}
		
		\onslide<1->{\solutionmarkB}
	\end{frame}


	
	\section{Aufgabe 3 \& 4: \\ \itshape CNF-Design und CYK-Algorithmus}
	
	\begin{frame}\frametitle{Chomsky-Normalform}	
		\small	
		\defbox{Eine kontextfreie Grammatik $G=\tuple{V,\Sigma,P,S}$ ist in \redalert{Chomsky-Normalform} \redalert{(CNF)}, wenn alle ihre Produktionsregeln eine der beiden folgenden Formen haben:%
			\[ \Snterm{A}\to \Snterm{BC} \quad\text{ (mit $\Snterm{B},\Snterm{C}\in V$)} \qquad\text{oder}\qquad \Snterm{A}\to \Sterm{c}\quad\text{ (mit $\Sterm{c}\in \Sigma$)}\]
		}
		
		\textbf{Umwandlung in CNF}:
		\begin{enumerate}[(1)]
			\item Eliminierung von $\epsilon$-Regeln
			\item Eliminierung von Kettenregeln
			\item Extrahieren von Terminalsymbolen in Regeln $\Sntermsub{V}{c} \to \Sterm{c}$
			\item Reduzieren von Regeln der Form $\Snterm{A}\to\Sntermsub{B}{1}\cdots\Sntermsub{B}{n}$ auf $n = 2$
		\end{enumerate}
	\end{frame}
	
	\newcommand{\Rightarrowstarquest}{\mathrel{{\stackrel{?}{\Rightarrow}}{}^\ast}}
	
	\begin{frame}\frametitle{CYK: Grundidee}
		\small
		\textbf{gegeben}: kontextfreie Grammatik $G$ in CNF \\
		\textbf{Frage:} $w = \Sterm{a_1}\cdots\Sterm{a_n} \in \Slang{L}(G)$?
		
		\begin{itemize}
			\item Falls $|w|=1$, dann ist $w\in\Sigma$ und es gilt:\\
			$w\in\Slang{L}(G)$ genau dann wenn es eine Regel $\Snterm{S}\to w$ in $G$ gibt
			\item Falls $|w|>1$, dann ist:\\
			$w\in\Slang{L}(G)$ genau dann wenn es eine Regel
			$\Snterm{S}\to\Snterm{A}\Snterm{B}$ und eine Zahl $i$ gibt, so dass gilt
			\[\Snterm{A}\Rightarrow^\ast \Sterm{a_1}\cdots\Sterm{a_i}\qquad\text{und}\qquad \Snterm{B}\Rightarrow^\ast \Sterm{a_{i+1}}\cdots\Sterm{a_n}\]
		\end{itemize}
		
		\textbf{Idee}: Fall 2 reduziert das Problem $\Snterm{S}\Rightarrowstarquest w$ auf zwei einfachere Probleme
		$\Snterm{A}\Rightarrowstarquest \Sterm{a_1}\cdots\Sterm{a_i}$ und $\Snterm{B}\Rightarrowstarquest \Sterm{a_{i+1}}\cdots\Sterm{a_n}$, die man allerdings für alle Regeln $\Snterm{S}\to\Snterm{A}\Snterm{B}$ und Indizes $i$ lösen muss
		
	\end{frame}
	
	\newcommand{\MEC}[1]{\multicolumn{1}{c}{#1}}
	
	\begin{frame}\frametitle{CYK: Praktische Umsetzung}
		\small
		\textbf{Vorgehen}: $V[i,j] = $ Menge aller $\Snterm{A}$ mit $\Snterm{A}\Rightarrow^\ast w_{i,j}$ 
		
		\begin{itemize}
			\item Diagonale = Fall 1: existiert Terminalsymbolregel $\duckA \to \Sterm{a}$
			\item Fixiere Element $\duckC$: sei $\duckA$ in der gleichen Zeile ganz links und $\duckB$ direkt unten drunter 
			\begin{itemize}
				\item wenn eine Regel $\duckD \to \duckA \duckB$, dann füge $\duckD$ zu $\duckC$ hinzu
				\item schiebe $\duckA$ nach rechts und $\duckB$ nach unten und wiederhole
			\end{itemize}
		\end{itemize}
		Ist am Ende das Startsymbol $S \in \alert{V[1,|w|]}$, dann $w \in \Slang{L}$.
		
		\textbf{Beispiel}: Wir betrachten das Wort $w=\Sterm{a+b*c}$ der Länge $|w|=5$.		
		\begin{center}
			\scalebox{0.7}{%
				\begin{tabular}{c|c|c|c|c|c|}
					%  \multicolumn{1}{c}{}  & \multicolumn{1}{c}{A} & \multicolumn{1}{c}{B} & \multicolumn{1}{c}{C} & \multicolumn{1}{c}{D}  \\\cline{2-5}
					\cline{2-6}
					\Sterm{a}
					& $V[1,1]$ & $V[1,2]$ & $V[1,3]$ & $V[1,4]$ & \alert{$V[1,5]$} \\\cline{2-6}
					\MEC{\Sterm{+}} &  
					& $\duckA$ & $V[2,3]$ & $\duckC \cup \duckD$ & $V[2,5]$ \\\cline{3-6}
					\MEC{\Sterm{b}} &\MEC{}   &
					& $V[3,3]$ & $\duckB$ & $V[3,5]$ \\\cline{4-6}
					\MEC{\Sterm{*}} &\MEC{}   &\MEC{}   &   
					& $V[4,4]$ & $V[4,5]$ \\\cline{5-6}
					\MEC{\Sterm{c}} &\MEC{}   &\MEC{}   &\MEC{}   &
					& $V[5,5]$ \\\cline{6-6}
					\MEC{}          &\MEC{\Sterm{a}} & \MEC{\Sterm{+}} & \MEC{\Sterm{b}} & \MEC{\Sterm{*}} & \MEC{\Sterm{c}}
			\end{tabular}}
		\end{center}
	\end{frame}

	\begin{frame} \frametitle{Aufgabe 3 \& 4}
		\footnotesize
		\textbf{Aufgabe 3}:
		Geben Sie für die nachfolgenden Sprachen $L_i$ jeweils
		eine (kontextfreie) Grammatik $G_i$ in CNF mit $L_i=L(G_i)$ an:
		
		\begin{enumerate}[(a)]
			\item Es sei $L_1$ genau die Menge der Palindrome "uber $\Sigma = \{ a, b \} $.
			(Palindrome sind W"orter, die vorw"arts und r"uckw"arts gelesen
			gleich sind, z.B. $ aba, \; abba, \; a, \; \varepsilon , \; bb$)
			\item Es sei $L_2$ die Sprache aller $w \in \{ a, b \}^{ * } $ mit
			gleicher Anzahl an $a$'s und $b$'s.
			\item Es sei $L_3 = \{ (ab)^{n}  (ba)^{n} \mid n \geq 0 \}$
			"uber dem Alphabet $ \Sigma = \{ a, b \} $
		\end{enumerate}
	
		\textbf{Aufgabe 4}:
		Gegeben seien $L_1,L_2,L_3$ wie in Aufgabe 3.
		Wenden Sie den CYK-Algorithmus für folgende Instanzen an
		um zu prüfen ob das jeweilige Wort zur entsprechenden 
		Sprache geh"ort:	
		
		\begin{enumerate}[(a)]
			\item $L_1$ mit $w = abba$ sowie $w = aba$
			\item $L_2$ mit $w = aababb$
			\item $L_3$ mit $w = ababbaba$
		\end{enumerate}
	\end{frame}

\end{document}