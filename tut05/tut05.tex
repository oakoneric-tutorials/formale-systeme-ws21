\documentclass{beamer}
\usepackage{../tut-slides}
\usepackage{../mathoperators}
\usepackage{../fs}

\usepackage{csquotes}

\usepackage{amsmath,amssymb}
%\usepackage{enumerate}
\usepackage[normalem]{ulem}
\newcommand{\labelitemi}{\raisebox{1pt}{\scalebox{.9}{$\blacktriangleright$}}}
\newcommand{\labelitemii}{$\vartriangleright$}
\newcommand{\labelitemiii}{--}

\usepackage{booktabs}
\usepackage{tabularx}

\newcommand{\tuple}[1]{\langle{#1}\rangle}
\newcommand{\simquot}[1]{#1/_{\!\!{\sim}}}

%%%%%%%%%%%%%%%%%%%%%%%%%%%%%%%%%%%%%%%%%%%%%%%%%%%%%%%%%%%%%%%%%%%%%%%


\begin{document}	
	\title{Formale Systeme}
	\subtitle{Übung 5}
	\author{Eric Kunze}
	\email{eric.kunze@tu-dresden.de}
	\city{TU Dresden}
	\date{\today}
%	\institute{Lehrstuhl für Grundlagen der Programmierung}
	\titlegraphic{\includegraphics[width=2cm]{../TUD-white.pdf}}

	\maketitle

	

	\section{Aufgabe 1: \\ \itshape NFA $\to$ RegExp}


	\begin{frame}\frametitle{NFA $\to$ RegExp: Ersetzungsmethode}
		\small
		\textbf{Gegeben}: NFA  $\Smach{M}=\tuple{Q,\Sigma,\delta,Q_0,F}$
		
		\textbf{Gesucht}: regulärer Ausdruck $\alpha$ mit $\Slang{L}(\alpha)=\Slang{L}(\Smach{M})$
		
		\textbf{Idee}:
		
		\codebox{Für jeden Zustand $q\in Q$, berechne einen regulären Ausdruck $\alpha_q$ für die Sprache $\Slang{L}(\alpha_q) = \Slang{L}(\Smach{M}_q)$ mit $\Smach{M}_q=\tuple{Q,\Sigma,\delta,\{q\},F}$}
		
		Für Startzustände $Q_0 = \menge{q_1,q_2,\dots, q_n}$ gilt dann
		\begin{equation*}
			\Slang{L}(\Smach{M})
			=\bigcup_{q\in Q_0} \Slang{L}(\alpha_q)
			=  \Slang{L}(\alpha_{q_1}\mid\alpha_{q_2}\mid\dots\mid \alpha_{q_n})
		\end{equation*}
	\end{frame}

	\begin{frame}		
		\small
		\begin{enumerate}[(1)]
			\item \textbf{Vereinfache den Automaten} (entferne offensichtlich unnötige Zustände)	
			
			\item \textbf{Bestimme das Gleichungssystem} 
			
			\textit{Intuition}: Beschreibe $\alpha_q$ in Abhängigkeit von Folgezuständen
			
			\defbox{ \\[-0.9\baselineskip]
				\begin{itemize}
					\item Für jeden Zustand $q\in Q \setminus F$:
					$\alpha_q \equiv \sum_{\Sterm{a}\in\Sigma}\; \sum_{p\in\delta(q,\Sterm{a})} \Sterm{a}\alpha_p$
					\item Für jeden Zustand $q\in F$:
					$\alpha_q \equiv \bluealert{\epsilon} \mid \sum_{\Sterm{a}\in\Sigma}\; \sum_{p\in\delta(q,\Sterm{a})} \Sterm{a}\alpha_p$
				\end{itemize}
			}
			%
			\item \textbf{Löse das Gleichungssystem} durch Einsetzen und
			
			\theobox{Regel von Arden: Aus $\alpha\equiv \beta\alpha\mid\gamma$ mit $\epsilon\notin\Slang{L}(\beta)$ folgt $\alpha\equiv\beta^\ast\gamma$.}
			%
			\item \textbf{Gib den Ausdruck für die Sprache des NFA an}
			
			Für $Q_0 = \menge{q_1,q_2,\dots, q_n}$ gilt dann
			\begin{equation*}
				\Slang{L}(\Smach{M})
				=\bigcup_{q\in Q_0} \Slang{L}(\alpha_q)
				=  \Slang{L}(\alpha_{q_1}\mid\alpha_{q_2}\mid\dots\mid \alpha_{q_n})
			\end{equation*}
		\end{enumerate}
	\end{frame}

	\begin{frame} \frametitle{Aufgabe 1}
		\small
		Gegeben ist der DFA $\mathcal{M} = \tuple{\menge{q_0, q_1, q_2}, \menge{a,b}, \delta, q_0, \menge{q_0}}$ mit $\delta$:
		\begin{center}
			\begin{tikzpicture}[%
				->,
				>=stealth',
				semithick,
				initial text=,
				shorten <=2pt,   
				shorten >=2pt,
				auto, 
				on grid,
				node distance=7ex and 8em,
				every state/.style={minimum size=0pt,inner sep=2pt,text height=1.5ex,text depth=.25ex},
				bend angle=20]
				\node[state,initial,accepting] (q_0) {$q_0$};
				\node[state] (q_1) [above right=of q_0] {$q_1$};
				\node[state] (q_2) [below right=of q_0] {$q_2$};
				\path[->]
					(q_0) edge[loop above] node {$a$} (q_0)
					(q_0) edge[bend left] node {$b$} (q_1)
					(q_1) edge[bend left] node {$a$} (q_0)
					(q_1) edge[bend left] node {$b$} (q_2)
					(q_2) edge[loop right] node {$a$} (q_2)
					(q_2) edge[bend left] node {$b$} (q_0);
			\end{tikzpicture}
		\end{center}
		Geben Sie einen regulären Ausdruck $\alpha$ an, der die von $\mathcal{M}$ akzeptierte Sprache repräsentiert, d.h. es gilt $L(\alpha) = L(\mathcal{M})$.
		
		\scriptsize
		Hinweis: Zur Lösung können Sie die Ersetzungsmethode verwenden: geben Sie hierzu für jeden Zustand $q_i$ des Automaten eine Gleichung $\alpha_i = \dots$ an. Lösen Sie anschließend das Gleichungssystem mithilfe des \textit{Arden-Lemmas}.
	\end{frame}

	\section{Aufgabe 2: \\ \itshape Minimierung von Automaten}

	\begin{frame}\frametitle{Äquivalenz von Zuständen \& Quotientenautomat}
		\small
		DFA $\Smach{M}=\tuple{Q,\Sigma,\delta,q_0,F} \leadsto$ DFA $\Smach{M}_q=\tuple{Q,\Sigma,\delta,q,F}$
		
		\defbox{
			\makebox[5.5cm][l]{\textbf{Äquivalenz von Zuständen}:} $p\sim_{\Smach{M}}q 
			\quad \Leftrightarrow \quad
			\Slang{L}(\Smach{M}_{p}) = \Slang{L}(\Smach{M}_{q})$ \\
			\makebox[5.5cm][l]{\textbf{Äquivalenzklasse}:} $[q]_{\sim}=\{p\in Q\mid q\sim p\}$ \\
			\makebox[5.5cm][l]{\textbf{Quotient} von $P \subseteq Q$:} $\simquot{P\!} = \menge{[p]_{\sim}\mid p \in P}$
		}
	
	
		\textbf{Quotientenautomat}:  Verschmelzen von äquivalenten Zuständen
		
		\defbox{Für einen DFA $\Smach{M} = \tuple{Q,\Sigma,\delta,q_0,F}$ mit totaler Übergangsfunktion ist der Quotientenautomat $\simquot{\Smach{M}} =\tuple{\simquot{Q},\Sigma,\delta_{\sim},[q_0]_{\sim_{\Smach{M}}},\simquot{F\!}}$ gegeben durch
		\begin{itemize}
			\item $\simquot{Q} = \{[q]_{\sim}\mid q\in Q\}$
			\item $\delta_{\sim}([q]_{\sim},\Sterm{a}) = [\delta(q,\Sterm{a})]_{\sim}$
			\item $\simquot{F\!} = \{[q]_{\sim}\mid q\in F\}$
		\end{itemize}
		}
	\end{frame}

	\begin{frame}
		\small
		
		\textbf{Bestimmung von $\sim$}:
		\vspace{-.7\baselineskip}
		\begin{itemize}
			\item Initialisiere ${\not\sim}\defeq\emptyset$
			\item \textbf{Regel 1}: Für jedes Paar von Zuständen $\tuple{q,p}\in Q\times Q$:\\
			falls $q\in F$ und $p\notin F$, dann "`speichere $q\not\sim p$"'
			% ${\not\sim}\defeq{\not\sim}\cup\{\tuple{q,p},\tuple{p,q}\}$
			\item \textbf{Regel 2}: Für jedes Paar $\tuple{q,p}\in Q\times Q\setminus {\not\sim}$ und jedes $\Sterm{a}\in\Sigma$:\\
			falls $\delta(q,\Sterm{a})\not\sim\delta(p,\Sterm{a})$ dann "`speichere $q\not\sim p$"'
			% 	${\not\sim}\defeq{\not\sim}\cup\{\tuple{q,p},\tuple{p,q}\}$
			\item Wiederhole Regel 2 bis keine Änderungen mehr auftreten
			\item Das Ergebnis ist $(Q\times Q)\setminus {\not\sim}$
		\end{itemize}
		
		\examplebox{
			\begin{minipage}{6cm}
				\begin{flushleft}
					\textbf{Beispiel}: Für einen DFA mit Zuständen $Q=\{A,B,C,D,E\}$ genügt eine Tabelle mit zehn Feldern (statt $5^2=25$).\\[1ex]
					
					{\tiny (dazu reihen wir Zustände vertikal in umgekehrter Reihenfolge)}
				\end{flushleft}
			\end{minipage}
			\narrowcentering{\scalebox{0.8}{%
					\begin{tabular}{c|c|c|c|c|}
						\multicolumn{1}{c}{}  & \multicolumn{1}{c}{A} & \multicolumn{1}{c}{B} & \multicolumn{1}{c}{C} & \multicolumn{1}{c}{D}  \\\cline{2-5}
						E &   &   &   &    \\\cline{2-5}
						D &   &   &        \\\cline{2-4} 
						C &   &            \\\cline{2-3}
						B &                \\\cline{2-2}
			\end{tabular}}}
		}
	\end{frame}
	
	\begin{frame} \frametitle{Aufgabe 2}
		\small
		Gegeben ist der $\varepsilon$-NFA $\mathcal{M} = (\{q_0,q_1,q_2,q_3,q_4\}, \{a, b\},
		\Delta, \{q_{0}\}, \{q_{2}\})$ mit $\Delta$:
		\begin{center}
			\begin{tikzpicture}[%
				->,
				>=stealth',
				semithick,
				initial text=,
				shorten <=2pt,   
				shorten >=2pt,
				auto, 
				on grid,
				node distance=7ex and 8em,
				every state/.style={minimum size=0pt,inner sep=2pt,text height=1.5ex,text depth=.25ex},
				bend angle=20]
				\node[state,initial] (q_0) {$q_0$};
				\node[state] (q_1) [above right=of q_0] {$q_1$};
				\node[state,accepting] (q_2) [right=of q_1] {$q_2$};
				\node[state] (q_3) [below right=of q_0] {$q_3$};
				\node[state] (q_4) [right=of q_3] {$q_4$};
				\path[->]
				(q_0) edge node {$\varepsilon$} (q_1)
				(q_0) edge node {$b$} (q_3)
				(q_3) edge node {$a, b$} (q_1)
				(q_1) edge node {$\varepsilon$} (q_2)
				(q_2) edge node[above left] {$a$} (q_3)
				(q_2) edge[bend left] node {$a$} (q_4)
				(q_3) edge node {$b$} (q_4)
				(q_4) edge [bend left] node {$a$} (q_2) 
				(q_4) edge [loop right, white] node [white] {$a,b$} (q_4) ;
			\end{tikzpicture}
		\end{center}
		
		\begin{enumerate}[a)]
			\item Konstruieren Sie einen zu $\mathcal M$ äquivalenten DFA $\mathcal M'$.
			\item Geben Sie den zu $\mathcal M'$ reduzierten DFA $\mathcal M'_r$ an.
		\end{enumerate}
	\end{frame}
	
	\section{Aufgabe 3: \\ \itshape Nerode-Rechtskongruenz}
	
	\begin{frame} \frametitle{Nerode-Rechtskongruenz}
		\small		
		\defbox{%
			\textbf{Nerode-Rechtskongruenz} ${\simeq_{\Slang{L}}}\subseteq \Sigma^\ast\times \Sigma^\ast$:\\ 
			$u\simeq_{\Slang{L}}v$ $\qquad$ falls$\qquad$
			$uw\in\Slang{L} \enskip \Leftrightarrow \enskip vw\in\Slang{L} \qquad \forall w\in\Sigma^\ast$
		}%
		
		\theobox{\textbf{Satz (Myhill \& Nerode):} Eine Sprache $\Slang{L}$ ist genau dann regulär, wenn $\simeq_{\Slang{L}}$ endlich viele Äquivalenzklassen hat.}
		
		\codebox{\textbf{Myhill-Nerode-Minimalautomat} $\Smach{M}_{\Slang{L}}=\tuple{Q,\Sigma,\delta,q_0,F}$ mit:
			\begin{itemize}
				\item $Q=\{[w]_\simeq\mid w\in\Sigma^\ast\}$
				\item $q_0=[\epsilon]_\simeq$
				\item $F=\{[w]_\simeq\mid w\in\Slang{L}\}$
				\item $\delta([w]_\simeq,\Sterm{a})=[w\Sterm{a}]_\simeq$
			\end{itemize}
		}
	\end{frame}
	
	\begin{frame} \frametitle{Aufgabe 3}
		Gegeben ist der regul\"are Ausdruck $\alpha = (bb)^\ast a$.
		\begin{enumerate}
			\item Geben Sie f\"ur $\alpha$ die \textit{Nerode}-Rechtskongruenz $\;\simeq_{L(\alpha)}\;$ an.
			\item Geben Sie einen minimalen DFA ${\mathcal M}$ an mit $L({\mathcal M})= L(\alpha)$.
		\end{enumerate} 
	\end{frame}

	\section{Aufgabe 4 \\ \itshape Beweis von Nichtregularität}
	
	\begin{frame} \frametitle{Beweis von Nichtregularität}
		\small
		\theobox{\textbf{Satz (Myhill \& Nerode):} Eine Sprache $\Slang{L}$ ist genau dann regulär, wenn $\simeq_{\Slang{L}}$ endlich viele Äquivalenzklassen hat.}
		\pause
		
		\theobox{\textbf{Satz:} Wenn $\Slang{L}_1$ und $\Slang{L}_2$ regulär sind, dann auch $\Slang{L}_1\cap \Slang{L}_2$,
		$\Slang{L}_1\cup \Slang{L}_2$, $\Slang{L}_1^\ast$ und $\overline{\Slang{L}}_1$.}
		\pause
		
		\theobox{\textbf{Satz (Pumping-Lemma):}
			Für jede reguläre Sprache $\Slang{L}$\\
			gibt es eine Zahl $n\geq 0$, so dass gilt:\\
			~~für jedes Wort $x\in\Slang{L}$ mit $|x|\geq n$\\
			~~gibt es eine Zerlegung $x=uvw$ mit $|v|\geq 1$ und $|uv|\leq n$, so dass:\\
			~~~~für jede Zahl $k\geq 0$ gilt: $u v^k w\in\Slang{L}$
		}
	\end{frame}
	
	\begin{frame} \frametitle{Aufgabe 4}
		Welche der folgenden Sprachen sind regulär? Begründen Sie Ihre Antwort.
		\begin{enumerate}[a)]
			\item $L_a = \menge{ww^R : w \in \menge{\Sterm{0},\Sterm{1}}^\ast}$
			\item $L_b = \menge{\Sterm{a}^n \Sterm{c}^m \Sterm{b}^n : n,m \ge 0}$
			\item $L_c = \menge{w \in \menge{\Sterm{0},1}^\ast : \abs{w}_{\Sterm{0}} \text{ gerade}, \abs{w}_{\Sterm{1}} \text{ durch 3 teilbar}}$
			\item $L_d = \menge{w \in \menge{\Sterm{0},\Sterm{1}}^\ast : \text{ auf jede \Sterm{0} folgt eine \Sterm{1}}}$
			\item $L_e = \menge{\Sterm{0}^{n^2} : n \ge 0}$
			\item $L_f = \menge{\Sterm{0}^m \Sterm{1}^n \Sterm{0}^{n+m} : n,m \ge 1}$
		\end{enumerate}
	\end{frame}	
\end{document}