\documentclass{beamer}
\usepackage{../tut-slides}
\usepackage{../mathoperators}
\usepackage{../fs}

\usepackage{csquotes}
\usepackage{ragged2e}

\usepackage{amsmath,amssymb}
%\usepackage{enumerate}
\usepackage[normalem]{ulem}
\newcommand{\labelitemi}{\raisebox{1pt}{\scalebox{.9}{$\blacktriangleright$}}}
\newcommand{\labelitemii}{$\vartriangleright$}
\newcommand{\labelitemiii}{--}

\usepackage{booktabs}
\usepackage{tabularx}

\newcommand{\tuple}[1]{\langle{#1}\rangle}
\newcommand{\simquot}[1]{#1/_{\!\!{\sim}}}

\usepackage{pifont}

\newcommand{\cmark}{\textcolor{cddarkgreen}{\ding{51}}}%
\newcommand{\xmark}{\textcolor{darkred}{\ding{55}}}%

\newcommand{\solutionmarkB}{%
	\begin{tikzpicture}[remember picture, overlay]
		\node [
			fill=none,  % Farbe des Randstreifens
			text=cdorange,  % Textfarbe
			font=\fosfamily\bfseries\large,  % Einstellungen für die Schrift
			inner xsep=0,       % Abstand des Textes von unten
			% maximale Textbreite = Papierhöhe - 2*Abstand des Textes von unten:
			%			text width={\dimexpr\paperheight-2\footskip\relax},
			align=center,
			%			minimum height=8mm,% Breite des Randstreifens
			anchor=south west,
			rotate=90
		] at ($(current page.north west)+(+8mm,-20mm)$)
		{Lösung};
		\draw[draw, line width=2pt, color=cdorange] ($(current page.north west)+(+1mm,0)$) -- ($(current page.south west)+(+1mm,0)$);
	\end{tikzpicture}%
}
\newcommand{\solutionmarkFT}{\begin{tikzpicture}[remember picture, overlay]
	\node [
	fill=none,  % Farbe des Randstreifens
	text=cdorange,  % Textfarbe
	font=\fosfamily\bfseries\large,  % Einstellungen für die Schrift
	inner xsep=0,       % Abstand des Textes von unten
	% maximale Textbreite = Papierhöhe - 2*Abstand des Textes von unten:
	%			text width={\dimexpr\paperheight-2\footskip\relax},
	align=center,
	%			minimum height=8mm,% Breite des Randstreifens
	anchor=south west,
	rotate=90
	] at ($(current page.north west)+(+8mm,-27mm)$)
	{Lösung};
	\draw[draw, line width=2pt, color=cdorange] ($(current page.north west)+(+1mm,0)$) -- ($(current page.south west)+(+1mm,0)$);
	\end{tikzpicture}%
}

\usepackage{tikzducks}
\newcommand{\duckA}{%
	\scalebox{0.2}{
		\reflectbox{
	\begin{tikzpicture}[baseline=2ex]
	\duck[mask=teal,cape=teal, body=orange!60!yellow]
	\end{tikzpicture}%
	}
	}
}
\newcommand{\duckB}{%
	\scalebox{0.2}{
	\begin{tikzpicture}[baseline=2ex]
	\duck[body=pink,
	unicorn=magenta!60!violet,
	longhair=magenta!60!violet]
	\end{tikzpicture}%
	}
}
\newcommand{\duckC}{%
	\scalebox{0.2}{
	\begin{tikzpicture}[baseline=2ex]
	\duck[santa=red!80!black,
	beard=white!80!brown]
	\end{tikzpicture}%
	}
}
\newcommand{\duckD}{%
	\scalebox{0.2}{
	\begin{tikzpicture}[baseline=2ex]
	\duck[signpost=420, sunglasses=black!90, body=brown!60!white]
	\end{tikzpicture}%
	}
}


%%%%%%%%%%%%%%%%%%%%%%%%%%%%%%%%%%%%%%%%%%%%%%%%%%%%%%%%%%%%%%%%%%%%%%%
\newcommand{\ghost}[1]{\raisebox{0pt}[0pt][0pt]{\makebox[0pt][l]{#1}}}
\newcommand{\blue}[1]{\textcolor{darkblue}{#1}}
\newcommand{\purple}[1]{\textcolor{cdpurple}{#1}}
\newcommand{\green}[1]{\textcolor{cddarkgreen}{#1}}



\begin{document}	
	\title{Formale Systeme}
	\subtitle{Übung 10}
	\author{Eric Kunze}
	\email{eric.kunze@tu-dresden.de}
	\city{TU Dresden}
	\date{\formatdate{7}{1}{2022}}
%	\institute{Lehrstuhl für Grundlagen der Programmierung}
	\titlegraphic{\includegraphics[width=2cm]{../TUD-white.pdf}}

	\maketitle


	\begin{frame} \frametitle{Übungsblatt 10}
		\tableofcontents
	\end{frame}

	\section{Aufgabe 1: \\ \itshape Design von Turingmaschinen}

	

	\begin{frame} \frametitle{Aufgabe 1}
		\small		
		Geben Sie Turingmaschinen an, die folgende Funktionen 
		berechnen. Dabei wird eine Eingabe $n \in \mathbb N$ als $\emptyset^n$ mit $\emptyset \in \Sigma$ dargestellt. Es kann vorausgesetzt werden, dass die Eingabe wohlgeformt auf dem Band vorliegt. Am Ende der Berechnung hält die Turingmaschine in einem Finalzustand und das Band enthält nur das Berechnungsergebnis. 
		
		\begin{enumerate}[(a)]
			\item Die Turingmaschine $\mathcal M_{0}$ berechnet die
			Funktion $f: \mathbb N \rightarrow \mathbb N,\,n\mapsto 0$, d.\,h. das Eingabewort auf dem
			Band wird gelöscht.
			\item Die Turingmaschine $\mathcal M_{\operatorname{succ}}$ berechnet die
			Funktion $f:\mathbb N\rightarrow \mathbb N,\,n\mapsto n+1$.
			\item Die Turingmaschine $\mathcal M_{\times 3}$ berechnet die
			Funktion $f:\mathbb N \rightarrow \mathbb N,\,n\mapsto 3*n$.
		\end{enumerate} 
	\end{frame}

	

	\section{Aufgabe 2 \\ \itshape Typ 1 und LBAs}
	

	\begin{frame} \frametitle{Aufgabe 2}
		\small
		Wir betrachten die Sprache 
		\begin{equation*}
			L = \menge{a^n b^m c^k : n,m,k \ge 1, n = 2m \text{ oder } m = k} .
		\end{equation*}
		Zeigen Sie, dass $L$ von Typ 1 ist, indem Sie einen LBA skizzieren, der $L$ entscheidet.
	
		
	\end{frame}

	

	\section{Aufgabe 3 \\ \itshape Quadratische Laufzeit entscheidbar?}

	
	\begin{frame} \frametitle{Aufgabe 3}
		\small
		\justifying
		Im Folgenden bezeichne $\mathcal{M}_w$ eine deterministische 
		Turingmaschine mit einem Band und dem Eingabe\-alphabet $\Sigma=\{0,1,\#\}$, deren Codewort $\operatorname{enc}(\mathcal{M}_w)$ 
		gleich $w$ ist, falls es ein solches gibt (vgl. Vorlesung 19, Folie 15).
		Andernfalls ist $\mathcal{M}_w = \mathcal{M}_{\bot}$, eine fest 
		gew\"ahlte deterministische Turingmaschine 
		mit dem Eingabealphabet $\Sigma=\{0,1\}$, die f\"ur alle 
		Eingabewörter endlos läuft.
		
		Ist die nachfolgende Sprache entscheidbar?
		\begin{equation*}
			\setlength\arraycolsep{2pt}
			L = \left\{ \, w \in \{0,1,\#\}^\ast  \, \left\vert \
			\begin{array}{l}
			\text{es gibt ein Wort } z \in \{0,1,\#\}^\ast \text{ mit} \\
			|z| \leqslant |w|^2, \text{ so dass } \mathcal{M}_w \text{ das Eingabewort } \\
			z \text{ in höchstens } |z| \text{ Schritten akzeptiert}
			\end{array}
			\right.\right\}
		\end{equation*}
		Begründen Sie Ihre Antwort.
	\end{frame}

	\section{Aufgabe 4 \\ \itshape Exponentielle Laufzeit entscheidbar?}
	
	\begin{frame} \frametitle{Aufgabe 4}
		\small
		Sei $\mathcal{M}_w$ wie in Aufgabe~3 und 
		\begin{equation*}
		t_{\mathcal{M}_w} (x)
		\defeq
		\text{Anzahl der Schritte, die $\mathcal{M}_w$ bei
			Eingabe $x$ durchf\"uhrt.}
		\end{equation*}
		
		Ist die Sprache $L = \{ w \in \{0,1\}^\ast  \, \mid t_{\mathcal{M}_w}(w) > 2^{|w|} \}$ entscheidbar?
		Begründen Sie Ihre Antwort.
	\end{frame}

	\begin{frame} \frametitle{Aufgabe 5 \\ \itshape Quadratzahlen erkennen}
		\begin{enumerate}[(a)]
			\item Zeigen Sie, dass die Summer der ersten $n$ ungeraden Zahlen $n^2$ ergibt; also dass
			\begin{equation*}
				\sum_{i=1}^n (2i - 1) = n^2 .
			\end{equation*}
			\item Geben Sie eine Grammatik für die Sprache 
			\begin{equation*}
				L = \menge{0^n : n \text{ ist eine Quadratzahl}}
			\end{equation*}
			an.
			\item Geben Sie eine deterministische Zwei-Band-Turingmaschine an, die $L$ akzeptiert.
		\end{enumerate}
	\end{frame}

	
	

\end{document}