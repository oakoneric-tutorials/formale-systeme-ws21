\documentclass{beamer}
\usepackage{../tut-slides}
\usepackage{../mathoperators}
\usepackage{../fs}

\usepackage{csquotes}

\usepackage{amsmath,amssymb}
%\usepackage{enumerate}
\usepackage[normalem]{ulem}
\newcommand{\labelitemi}{\raisebox{1pt}{\scalebox{.9}{$\blacktriangleright$}}}
\newcommand{\labelitemii}{$\vartriangleright$}
\newcommand{\labelitemiii}{--}

\usepackage{booktabs}
\usepackage{tabularx}

\newcommand{\tuple}[1]{\langle{#1}\rangle}
\newcommand{\simquot}[1]{#1/_{\!\!{\sim}}}

\usepackage{pifont}

\newcommand{\cmark}{\textcolor{cddarkgreen}{\ding{51}}}%
\newcommand{\xmark}{\textcolor{darkred}{\ding{55}}}%

\newcommand{\solutionmarkB}{%
	\begin{tikzpicture}[remember picture, overlay]
		\node [
			fill=none,  % Farbe des Randstreifens
			text=cdorange,  % Textfarbe
			font=\fosfamily\bfseries\large,  % Einstellungen für die Schrift
			inner xsep=0,       % Abstand des Textes von unten
			% maximale Textbreite = Papierhöhe - 2*Abstand des Textes von unten:
			%			text width={\dimexpr\paperheight-2\footskip\relax},
			align=center,
			%			minimum height=8mm,% Breite des Randstreifens
			anchor=south west,
			rotate=90
		] at ($(current page.north west)+(+8mm,-20mm)$)
		{Lösung};
		\draw[draw, line width=2pt, color=cdorange] ($(current page.north west)+(+1mm,0)$) -- ($(current page.south west)+(+1mm,0)$);
	\end{tikzpicture}%
}
\newcommand{\solutionmarkFT}{\begin{tikzpicture}[remember picture, overlay]
	\node [
	fill=none,  % Farbe des Randstreifens
	text=cdorange,  % Textfarbe
	font=\fosfamily\bfseries\large,  % Einstellungen für die Schrift
	inner xsep=0,       % Abstand des Textes von unten
	% maximale Textbreite = Papierhöhe - 2*Abstand des Textes von unten:
	%			text width={\dimexpr\paperheight-2\footskip\relax},
	align=center,
	%			minimum height=8mm,% Breite des Randstreifens
	anchor=south west,
	rotate=90
	] at ($(current page.north west)+(+8mm,-27mm)$)
	{Lösung};
	\draw[draw, line width=2pt, color=cdorange] ($(current page.north west)+(+1mm,0)$) -- ($(current page.south west)+(+1mm,0)$);
	\end{tikzpicture}%
}


%%%%%%%%%%%%%%%%%%%%%%%%%%%%%%%%%%%%%%%%%%%%%%%%%%%%%%%%%%%%%%%%%%%%%%%
\newcommand{\ghost}[1]{\raisebox{0pt}[0pt][0pt]{\makebox[0pt][l]{#1}}}
\newcommand{\blue}[1]{\textcolor{darkblue}{#1}}
\newcommand{\purple}[1]{\textcolor{cdpurple}{#1}}
\newcommand{\green}[1]{\textcolor{cddarkgreen}{#1}}
\newcommand{\orange}[1]{\textcolor{cdorange}{#1}}
\newcommand{\indigo}[1]{\textcolor{cdindigo}{#1}}

\usepackage{ragged2e}

\newcommand{\cellblue}{\cellcolor{cdblue!20}}
\newcommand{\cellred}{\cellcolor{cdpurple!20}}

\undef\Var
\DeclareMathOperator{\Var}{Var}


\begin{document}	
	\title{Formale Systeme}
	\subtitle{Übung 13}
	\author{Eric Kunze}
	\email{eric.kunze@tu-dresden.de}
	\city{TU Dresden}
	\date{\formatdate{28}{1}{2022}}
%	\institute{Lehrstuhl für Grundlagen der Programmierung}
	\titlegraphic{\includegraphics[width=2cm]{../TUD-white.pdf}}

	\maketitle


	\begin{frame} \frametitle{Übungsblatt 13}
		\tableofcontents
	\end{frame}

	
	\section{Aufgabe 1 \\ \itshape Resolution}
	
	\begin{frame} \frametitle{Klauselform \& Resolution}
		\small
		\begin{itemize}
			\item Eine Klausel $L_1\vee\ldots\vee L_n$ wir dargestellt als Menge $\{L_1,\ldots,L_n\}$
			\item Eine Konjunktion von Klauseln $K_1\wedge\ldots\wedge K_\ell$ wird dargestellt als Menge $\{K_1,\ldots,K_\ell\}$
		\end{itemize}
		Eine Menge von Mengen von Literalen unter dieser Interpretation heißt \redalert{Klauselform}.
		\medskip\pause
		
		\defbox{
			Gegeben seien zwei Klauseln $K_1$ und $K_2$ für die es ein Atom $p \in \Slang{P}$ gibt mit $p \in K_1$ und $\neg p \in K_2$. \\
			Die \redalert{Resolvente von $K_1$ und $K_2$ bezüglich $p$} ist die Klausel 
			\begin{equation*}
				(K_1\setminus\{p\})\cup(K_2\setminus\{\neg p\})
			\end{equation*}
			
			Eine Klausel $R$ ist eine \redalert{Resolvente einer Klauselmenge $\mathcal{K}$} wenn $R$ Resolvente zweier Klauseln  $K_1,K_2\in\mathcal{K}$ ist.
		}	
	\end{frame}

	
	\begin{frame}\frametitle{Bedeutung von Resolutionsschritten}
		\small
		Wir betrachten Klauseln $\{L_1,\ldots,L_n,p\}$ und $\{\neg p,M_1,\ldots,M_\ell\}$:\pause
		\begin{itemize}
			\item $\{L_1,\ldots,L_n,p\}\equiv (L_1\vee\ldots\vee L_n\vee p) \equiv \blue{(\neg L_1\wedge\ldots\wedge\neg L_n)\to p}$ \pause
			\item $\{\neg p,M_1,\ldots,M_\ell\}\equiv (\neg p\vee M_1\vee\ldots\vee M_\ell)\equiv \blue{p\to (M_1\vee\ldots\vee M_\ell)}$ \pause
		\end{itemize}
		\medskip
		
		Daraus folgt unmittelbar \blue{$(\neg L_1\wedge\ldots\wedge\neg L_n)\to (M_1\vee\ldots\vee M_\ell)$}\\
		$\leadsto$ dies entspricht der Klausel $\{L_1,\ldots,L_n,M_1,\ldots,M_\ell\}$ \pause
		\bigskip
		
		\theobox{\textbf{Satz:} Wenn $R$ Resolvente der Klauseln $K_1$ und $K_2$ ist, dann gilt $\{K_1,K_2\}\models R$.
		}		
		
		\begin{itemize}
			\item Die leere Klausel ist eine Disjunktionen von $0$ Literalen, also gerade $\bot$ (neutrales Element von $\lor$)
			\item $\bot$ in einer Konjunktion macht die gesamte Formel falsch (unerfüllbar).
			\item Lässt sich $\bot$ ableiten, so ist die Formel unerfüllbar.
		\end{itemize}
	\end{frame}
%	
	\begin{frame}\frametitle{Das Resolutionskalkül}
		\small
		\codebox{\textbf{Resolution}\\
			Gegeben: Formel $\mathcal{F}$ in Klauselform\\
			Gesucht: Ist $\mathcal{F}$ erfüllbar oder unerfüllbar?\\[1ex]
			
			\begin{enumerate}[(1)]
				\item Finde ein Klauselpaar $K_1,K_2\in \mathcal{F}$ mit Resolvente $R\notin \mathcal{F}$
				\item Setze $\mathcal{F}\defeq \mathcal{F}\cup\{R\}$
				\item Wiederhole Schritt (1) und (2) bis keine neuen Resolventen gefunden werden können
				\item Falls $\bot\in \mathcal{F}$, dann gib "`unerfüllbar"' aus;\\andernfalls gib "`erfüllbar"' aus
			\end{enumerate}
		}\pause
		
		\emph{Beobachtung:} Unerfüllbarkeit steht fest, sobald $\bot$ abgeleitet \ghost{wurde}\\
		$\leadsto$ dann kann man das Verfahren frühzeitig abbrechen
		\medskip
		
		Erfüllbarkeit kann dagegen erst erkannt werden, wenn alle Resolventen erschöpfend gebildet worden sind
		
	\end{frame}
	
	\begin{frame} \frametitle{Aufgabe 4}
		\small
		Prüfen Sie die folgende Formel mittels Resolutionsverfahren auf Erfüllbarkeit:
		\begin{enumerate}[a)]
			\item $b \land (a \lor b) \land (\lnot b \lor c) \land (\lnot b \lor \lnot c) \land (\lnot a \lor c)$
			\item $\lnot \Big( c \to \big( (\lnot a \land b \land c) \lor (a \land \lnot b) \big) \Big)$
		\end{enumerate}
	\end{frame}

	\begin{frame} 
		\footnotesize
		\textbf{(a)} Die Formel ist bereits in konjunktiver Normalform. Die zugehörige Klauselform lautet
		\begin{equation*}
			\menge{\menge{b}, \menge{a, b}, \menge{\lnot b, c}, \menge{\lnot b, \lnot c}, {\lnot a, c}}.
		\end{equation*}
		Durchnummerieren liefert:
		
		\begin{enumerate}[(1)]
			\item $\menge{b}$
			\item $\menge{a,b}$
			\item $\menge{\lnot b, c}$
			\item $\menge{\lnot b, \lnot c}$
			\item $\menge{\lnot a, c}$
		\end{enumerate}
		
		Als Resolutionsschritte kann man beispielsweise folgende wählen:
		
		\begin{enumerate}[(1)]			
			\setcounter{enumi}{5}%
			\item \makebox[4cm][l]{$\menge{c}$} (1) + (3)
			\item \makebox[4cm][l]{$\menge{\lnot b}$} (4) + (6)
			\item \makebox[4cm][l]{$\bot$} (1) + (7)
		\end{enumerate}
	
		Da die leere Klausel ableitbar ist, ist die Formel nicht erfüllbar.
		\solutionmarkB
	\end{frame}
	
	\begin{frame} 
		\footnotesize
		\textbf{(b)} Transformation in konjunktive Normalform liefert
		\begin{equation*}
			\big( c \land (a \lor \lnot b \lor \lnot c) \land (\lnot a \lor b) \big)
		\end{equation*}
		Klauselform: $\menge{\menge{c}, \menge{a, \lnot b, \lnot c}, \menge{\lnot a, b}}$
		
		\begin{enumerate}[(1)]
			\item $\menge{c}$
			\item $\menge{a, \lnot b, \lnot c}$
			\item $\menge{\lnot a, b}$
		\end{enumerate}
	
		Als Resolutionsschritte kann man beispielsweise folgende wählen:
	
		\begin{enumerate}[(1)]
			\setcounter{enumi}{5}%
			\item \makebox[4cm][l]{$\menge{a, \lnot b}$} (1) + (2)
			\item \makebox[4cm][l]{$\menge{b, \lnot b, \lnot c}$} (2) + (3)
			\item \makebox[4cm][l]{$\menge{a, \lnot a, \lnot c}$} (2) + (3)
			\item \makebox[4cm][l]{$\menge{b, \lnot b}$} (1) + (5)
			\item \makebox[4cm][l]{$\menge{a, \lnot a}$} (1) + (6)
			\item \makebox[4cm][l]{$\menge{\lnot a, b, \lnot c}$} (3) + (5)						
		\end{enumerate}
		\justifying Man kann diesen Prozess nun noch weiter durchführen, allerdings wird man nie die leere Klausel ableiten können. Die Formel ist daher erfüllbar.
		\solutionmarkB
	\end{frame}

	\section{Aufgabe 1 \\ \itshape Resolution}
	
	\begin{frame} \frametitle{Aufgabe 1}
		Prüfen Sie mittels Resolution die folgenden Formeln jeweils auf Erfüllbarkeit:
		\begin{enumerate}[a)]
			\item $a \land \Big(  (c \land b) \land \big( (\lnot c \lor \lnot b) \lor (a \land (c \land b)) \big) \Big)$
			\item $\begin{alignedat}{5}
				&(\lnot a \lor \lnot b) &\land& (a \lor b) &\land& (\lnot a \lor c) &\land& (a \lor \lnot c) \\ 
				\land \enskip &(b \lor \lnot d) &\land& (\lnot b \lor d) &\land& (\lnot c \lor d) &\land& (c \lor \lnot d)
				\end{alignedat}$
		\end{enumerate}
%	  a \wedge \Bigl(( c \wedge b) \wedge \bigl((\neg c \vee \neg b ) \vee
%	(a \wedge (c \wedge b))\bigr)\Bigr) \\[1ex]
%	(\neg a \vee \neg b) \wedge
%	(     a \vee      b) \wedge
%	(\neg a \vee  c ) \wedge
%	( a \vee \neg c) \wedge {}
%	( b \vee \neg d) \wedge
%	(\neg b  \vee d) \wedge
%	(\neg c \vee d) \wedge
%	(c \vee \neg d)
	\end{frame}

	\section{Aufgabe 2 \\ \itshape $k$-Färbbarkeit}
	
	\begin{frame} \frametitle{Aufgabe 2}
		\justifying
		Eine $k$-Färbung für einen endlichen Graphen $G$ ist eine Zuordnung der Knoten von $G$ zu Werten (\enquote{Farben}) in $\{\,1, \dots, k\,\}$, so dass Knoten, die in $G$ durch eine Kante verbunden sind, nicht denselben Wert zugeordnet bekommen.
		
		Geben Sie für einen endlichen Graphen $G = (V,E)$ mit $n$ Knoten und einen Wert $k$ eine aussagenlogische Formel $\phi_{G,k}$ an, so dass $\phi_{G,k}$ genau dann erfüllbar ist, wenn es eine $k$-Färbung von $G$ gibt.
	\end{frame}

	\begin{frame}
		\footnotesize \justifying
		Wir bezeichnen die Menge aller Farben mit $C_k = \menge{1, \dots, k}$.
		Wir verwenden die aussagenlogischen Variablen $p_{v, c}$ für $v \in V$ und $c \in C_k$ um auszudrücken, dass der Knoten $v$ mit der Farbe $c$ gefärbt wird. 
		
		\begin{itemize}
			\item Jeder Knoten hat mindestens eine Farbe:
			\begin{equation*}
				\phi_{\ge 1} \defeq \bigwedge_{v \in V} \bigvee_{c \in C_k} p_{v, c}
			\end{equation*}
			\item Jeder Knoten hat höchstens eine Farbe:
			\begin{equation*}
			\phi_{\le 1} \defeq \bigwedge_{v \in V} \bigvee_{\substack{c_1, c_2 \in C_k \\ c_1 \neq c_2}} \lnot \Big( p_{v, c_1} \land p_{v, c_2} \Big)
			\end{equation*}
			\item Benachbarte Knoten haben unterschiedliche Farben
			\begin{equation*}
			\phi_{\neq} \defeq \bigwedge_{\substack{v_1, v_2 \in V \\ (v_1, v_2) \in E}} \bigvee_{c \in C_k} \lnot \Big( p_{v_1, c} \land p_{v_2, c} \Big)
			\end{equation*}
		\end{itemize} 
		
		Die Formel $\phi_{G, k}$ ergibt sich dann als
		\begin{equation*}
			\phi_{G, k} \defeq \phi_{\ge 1} \land \phi_{\le 1} \land \phi_{\neq}.
		\end{equation*}		
		\solutionmarkB
	\end{frame}

	\section{Aufgabe 3 \\ \itshape Aussagenlogik nur mit $\to$?}
	
	\begin{frame} \frametitle{Aufgabe 3}
		\justifying
		\begin{enumerate}[a)]
			\item Es sei $\varphi$ eine Formel, die ausschließlich den Junktor
			$\rightarrow$ verwendet. Zeigen Sie: Wenn $w(p_i) = 1$ für alle $p_i \in
			\Var(\varphi)$ ist, dann ist auch $w(\varphi) = 1$.
			\item Es sei $\varphi$ eine allgemeine aussagenlogische Formel.  Beweisen oder
			widerlegen Sie: $\varphi$ ist äquivalent zu einer Formel, die ausschließlich
			den Junktor $\rightarrow$ verwendet.
		\end{enumerate}
	\end{frame}

	\begin{frame}
		\footnotesize 
		\begin{enumerate}[\bfseries (a)] \justifying
			\item Induktion über den Aufbau von Formeln.
			\begin{itemize} \footnotesize \justifying
				\item[(IA)] Sei $\phi = p \in \Slang P$ ein Atom. Dann gilt $w(p) = 1 \implies w(\phi) = 1$.
				\item[(IV)] Seien $\phi_1, \phi_2$ Formeln, die nur den Junktor $\to$ verwenden und Folgendes erfüllen: Gilt $w(p_1) = 1$ für alle $p_1 \in \Var(\phi_1)$, dann gilt $w(\phi_1) = 1$; sowie gilt $w(p_2) = 1$ für alle $p_2 \in \Var(\phi_2)$, dann gilt $w(\phi_2) =  1$.
				\item[(IS)] Sei $\phi \defeq (\phi_1 \to \phi_2)$. Es gilt $\Var(\phi) = \Var(\phi_1) \cup \Var(\phi_2)$. Ist $w(p) = 1$ für alle $p \in \Var(\phi)$, so auch $w(p) = 1$ für alle $p \in \Var(\phi_1)$ und $w(p) = 1$ für alle $p \in \Var(\phi_2)$. Per Induktionsvoraussetzung gilt dann auch $w(\phi_1) = 1$ und $w(\phi_2) = 1$; und somit schließlich $w(\phi) = w(\phi_1 \to \phi_2) = 1$. 
			\end{itemize}
			\item Die Aussage ist falsch. Sei $p \in \Slang P$ ein Atom mit $w(p) = 1$. Dann gibt es keine zu $\lnot p$ äquivalente Formel mit nur dem Junktor $\to$. 
			
			Achtung: auch die \enquote{Abkürzungen} $\top$ und $\bot$ dürfen nicht verwendet werden, d.h. $p \to \bot$ ist keine Formel, die nur den Junktor $\to$ verwendet.
		\end{enumerate}
	\end{frame}

	\section{Aufgabe 4 \\ \itshape Polynomielle Abschlüsse}
	
	\begin{frame} \frametitle{Aufgabe 4}
		\small
		Seien $\Sigma$ ein Alphabet und $L, L_1, L_2 \subseteq \Sigma^\ast$ Sprachen mit $L, L_1, L_2 \in \Slang{P}$.
		
		Zeigen Sie:
		\begin{enumerate}[a)]
			\item		
			$L_1 \cup L_2, L_1 \circ L_2$ und $\bar{L}$ sind in polynomieller Zeit entscheidbar.
			
			\item $L^R \, = \, \{w^R :\, w \, \in \, L\}$ ist in polynomieller Zeit entscheidbar,
			wobei f\"ur $w \, = \, a_1 \dots a_n \in \Sigma^\ast$ das Rückwärtswort $w^R$ definiert ist durch 
			\[
			w^R = a_n a_{n-1} \dots a_1
			\]
			(und $\varepsilon^R = \varepsilon$, $a^R = a$ f\"ur $a \in \Sigma$).			
		\end{enumerate}
	\end{frame}

	\begin{frame}
		\footnotesize
		\begin{enumerate}[\bfseries (a)]
			\item Seien $\Smach{M}$, $\Smach{M}_1$ und $\Smach{M}_2$ polynomiell zeitbeschränkte Turingmaschinen.
			\begin{itemize} \footnotesize
				\item[$\quer{L}$:] Die folgende Turingmaschine $\quer{\Smach M}$ entscheidet $\quer L$:
				
				\begin{slshape}
					Eingabe $w$ \\
					Simuliere $\Smach M$ auf $w$. \\
					\hspace{2em} falls $\Smach M$ akzeptiert: verwerfe \\
					\hspace{2em} falls $\Smach M$ verwirft: akzeptiere
				\end{slshape}
								
				Da $\Smach M$ in polynomieller Zeit entscheidet, entscheidet auch $\quer{\Smach M}$ in polynomieller Zeit. Es gilt also $\quer{L} \in \Slang P$.
				%
				\item[$L_1 \cup L_2$:] Die folgende Turingmaschine $\Smach M_{\cup}$ entscheidet $L_1 \cup L_2$:
				
				\begin{slshape}
					Eingabe $w$ \\
					Simuliere $\Smach M_1$ auf $w$. \\
					\hspace{2em} falls $\Smach M_1$ akzeptiert: akzeptiere \\
					\hspace{2em} falls $\Smach M_1$ verwirft: \\
					\hspace{4em} Simuliere $\Smach M_2$ auf $w$. \\
					\hspace{6em} falls $\Smach M_2$ akzeptiert: akzeptiere \\
					\hspace{6em} falls $\Smach M_2$ verwirft: verwerfe
				\end{slshape}
			
				\justifying
				Sowohl $\Smach M_1$ als auch $\Smach M_2$ entscheiden ihren Teil in polynomieller Zeit und \enquote{polynomiell + polynomiell = polynomiell}, d.h. auch $\Smach M_{\cup}$ entscheidet in polynomieller Zeit. Somit gilt $L_1 \cup L_2 \in \Slang P$.
			\end{itemize}
		\end{enumerate}
		\solutionmarkB
	\end{frame}

	\begin{frame}
		\footnotesize
		\begin{enumerate}[\bfseries (a)]
			\item Seien weiterhin $\Smach{M}$, $\Smach{M}_1$ und $\Smach{M}_2$ polynomiell zeitbeschränkte Turingmaschinen.
			\begin{itemize} \footnotesize
				\item[$L_1 \circ L_2$:] Die folgende Turingmaschine $\Smach M_{\circ}$ entscheidet $L_1 \circ L_2$:
				
				\begin{slshape}
					Eingabe $w = a_1 \dots a_n$ \\
					Für alle $1 \le k \le n$: \\
					\hspace{2em} Simuliere $\Smach M_1$ auf $a_1 \dots a_k$. \\
					\hspace{4em} falls $\Smach M_1$ verwirft: wähle nächstes $k$ \\
					\hspace{4em} falls $\Smach M_1$ akzeptiert: \\
					\hspace{6em} Simuliere $\Smach M_2$ auf $a_{k+1} \dots a_n$. \\
					\hspace{8em} falls $\Smach M_2$ akzeptiert: akzeptiere \\
					\hspace{8em} falls $\Smach M_2$ verwirft: wähle nächstes $k$
				\end{slshape}
				
				\justifying
				Sowohl $\Smach M_1$ als auch $\Smach M_2$ entscheiden ihren Teil in polynomieller Zeit. Die äußere Schleife wird maximal $n$ mal durchlaufen, d.h. der Aufwand lässt sich salopp mit
				\begin{equation*}
					 n * (\text{polynomiell} + \text{polynomiell}) = \text{polynomiell}
				\end{equation*} angeben. Daher ist $L_1 \circ L_2 \in \Slang P$.
			\end{itemize}
		\end{enumerate}
		\solutionmarkB
	\end{frame}

	\begin{frame}
		\footnotesize
		\begin{enumerate}[\bfseries (a)]
			\setcounter{enumi}{1}
			\item 
			\begin{slshape}
				Eingabe: $w \in \Sigma^\ast$. \\
				Transformiere $w$ zu $w^R$. \\
				Simuliere $\Smach M$ auf $w^R$. \\
				\hspace{2em} falls $\Smach M$ akzeptiert: akzeptiere \\
				\hspace{2em} falls $\Smach M$ verwirft: verwerfe \\
			\end{slshape}
		
			Die Simulation von $\Smach M$ ist polynomiell nach Voraussetzung und die Transformation von $w$ zu $w^R$ ist offensichtlich auch polynomiell. Somit ist $L^R \in \Slang P$.
		\end{enumerate}
		\solutionmarkB
	\end{frame}

	\section{Aufgabe 5 \\ \itshape Komplexität co-endlicher Sprachen}
	
	\begin{frame} \frametitle{Aufgabe 5}
		Eine Sprache $L \subseteq \Sigma^\ast$ ist \textit{co-endlich} genau dann, wenn $\bar{L}$ (also $\Sigma^\ast \setminus L$) endlich ist.
		
		Beweisen oder widerlegen Sie:
		\begin{enumerate}[a)]
			\item Jede co-endliche Sprache ist in $\operatorname{DTIME}(n)$.
			\item Jede co-endliche Sprache ist in $\operatorname{DTIME}(1)$.
		\end{enumerate}
	\end{frame}

	\begin{frame}
		\footnotesize
		\begin{enumerate}[\bfseries (a)]
			\item Die Aussage gilt. Jede endliche Sprache ist regulär und reguläre Sprachen sind unter Komplement abgeschlossen. Daher ist jede co-endliche Sprache regulär.
			
			Da das Wortproblem für reguläre Sprachen in linearer Zeit entscheidbar ist, ist jede co-endliche Sprache in $\operatorname{DTIME}(n)$.
			
			\item Auch diese Aussage gilt. Sei $L \subseteq \Sigma^\ast$ co-endlich. Dann ist per Definition $\quer{L}$ endlich.
			\begin{itemize} \footnotesize
				\item Ist $\quer{L} = \emptyset$, so ist $L = \Sigma^\ast$ und damit offensichtlich in konstanter Zeit entscheidbar (akzeptiere alle Eingaben). 
				\item Ist $\quer{L} \neq \emptyset$, so existiert (da $\quer{L}$ endlich ist) ein längstes Wort $w_{\max}$ mit $\abs{w_{\max}} = m$. Somit muss jede Turingmaschine $\Smach{M}$ höchstens $m$ Zeichen der Eingabe lesen (unabhängig von der Eingabe) und kann dann sofort $w \in \quer{L}$ und damit $w \in L$ entscheiden.
				Damit gilt nun: die Anzahl der Schritte von $\Smach{M}$ bei Eingabe $w \in \Sigma^\ast$ sind durch $f : \N \to \R$ mit $f(n) = c_m$ charakterisiert, wobei $c_m \in \R$ eine positive Konstante ist, die nur von $m$ abhängt. Für dieses $f$ gilt $f \in \mathcal{O}(1)$, da $f(n) \le c_m * 1$ für alle $n \ge 0$ gilt. Damit folgt $L \in \operatorname{DTIME}(1)$.
			\end{itemize}
		\end{enumerate}
	\solutionmarkB
	\end{frame}
\end{document}