\documentclass{beamer}
\usepackage{../tut-slides}
\usepackage{../mathoperators}
\usepackage{../fs}

\usepackage{csquotes}

\usepackage{amsmath,amssymb}
%\usepackage{enumerate}
\usepackage[normalem]{ulem}
\newcommand{\labelitemi}{\raisebox{1pt}{\scalebox{.9}{$\blacktriangleright$}}}
\newcommand{\labelitemii}{$\vartriangleright$}
\newcommand{\labelitemiii}{--}

\usepackage{booktabs}
\usepackage{tabularx}

\newcommand{\tuple}[1]{\langle{#1}\rangle}
\newcommand{\simquot}[1]{#1/_{\!\!{\sim}}}

\usepackage{pifont}

\newcommand{\cmark}{\textcolor{cddarkgreen}{\ding{51}}}%
\newcommand{\xmark}{\textcolor{darkred}{\ding{55}}}%

\newcommand{\solutionmarkB}{%
	\begin{tikzpicture}[remember picture, overlay]
		\node [
			fill=none,  % Farbe des Randstreifens
			text=cdorange,  % Textfarbe
			font=\fosfamily\bfseries\large,  % Einstellungen für die Schrift
			inner xsep=0,       % Abstand des Textes von unten
			% maximale Textbreite = Papierhöhe - 2*Abstand des Textes von unten:
			%			text width={\dimexpr\paperheight-2\footskip\relax},
			align=center,
			%			minimum height=8mm,% Breite des Randstreifens
			anchor=south west,
			rotate=90
		] at ($(current page.north west)+(+8mm,-20mm)$)
		{Lösung};
		\draw[draw, line width=2pt, color=cdorange] ($(current page.north west)+(+1mm,0)$) -- ($(current page.south west)+(+1mm,0)$);
	\end{tikzpicture}%
}
\newcommand{\solutionmarkFT}{\begin{tikzpicture}[remember picture, overlay]
	\node [
	fill=none,  % Farbe des Randstreifens
	text=cdorange,  % Textfarbe
	font=\fosfamily\bfseries\large,  % Einstellungen für die Schrift
	inner xsep=0,       % Abstand des Textes von unten
	% maximale Textbreite = Papierhöhe - 2*Abstand des Textes von unten:
	%			text width={\dimexpr\paperheight-2\footskip\relax},
	align=center,
	%			minimum height=8mm,% Breite des Randstreifens
	anchor=south west,
	rotate=90
	] at ($(current page.north west)+(+8mm,-27mm)$)
	{Lösung};
	\draw[draw, line width=2pt, color=cdorange] ($(current page.north west)+(+1mm,0)$) -- ($(current page.south west)+(+1mm,0)$);
	\end{tikzpicture}%
}

%%%%%%%%%%%%%%%%%%%%%%%%%%%%%%%%%%%%%%%%%%%%%%%%%%%%%%%%%%%%%%%%%%%%%%%


\begin{document}	
	\title{Formale Systeme}
	\subtitle{Übung 6}
	\author{Eric Kunze}
	\email{eric.kunze@tu-dresden.de}
	\city{TU Dresden}
	\date{\formatdate{26}{11}{2021}}
%	\institute{Lehrstuhl für Grundlagen der Programmierung}
	\titlegraphic{\includegraphics[width=2cm]{../TUD-white.pdf}}

	\maketitle

	

	\section{Aufgabe 1: \\ \itshape Pumping-Lemma}

	\newcommand{\colstackrel}[3]{\,{\stackrel{\textcolor{#3}{#1}}{\textcolor{#3}{#2}}}\,}
	\newcommand{\gstackrel}[2]{\colstackrel{#1}{#2}{darkgreen}}
	\newcommand{\bstackrel}[2]{\colstackrel{#1}{#2}{darkblue}}
	\newcommand{\rstackrel}[2]{\colstackrel{#1}{#2}{darkred}}
	
	\begin{frame}\frametitle{Nichtregularität durch Pumpen}	
		\small	
		\textbf{Idee:}
		\begin{itemize}
			\item Jeder DFA hat nur endlich viele Zustände $n$
			\item Aber manche reguläre Sprachen enthalten beliebig lange Wörter
		\end{itemize}
		
		\textbf{Wie kann ein DFA Wörter mit mehr als $n$ Zeichen akzeptieren?}
		
		\begin{itemize}
			\item Dann muss der DFA beim Einlesen einen Zustand mehr als einmal besuchen
			\item Dafür muss es in den Zustandsübergangen eine Schleife geben
			\item Diese Schleife kann man aber auch mehr als einmal durchlaufen
		\end{itemize}
		
		\begin{center}
			\itshape
			Jedes akzeptierte Wort mit $\geq n$ Zeichen hat einen Teil, den man beliebig oft wiederholen -- "`aufpumpen"' -- kann.
		\end{center}
		
	\end{frame}
	
	\begin{frame} \frametitle{Das Pumping-Lemma}
		\small
		\theobox{\textbf{Satz (Pumping-Lemma):}
			Für jede reguläre Sprache $\Slang{L}$\\
			gibt es eine Zahl $n\geq 0$, so dass gilt:\\
			~~für jedes Wort $z\in\Slang{L}$ mit $|z|\geq n$\\
			~~gibt es eine Zerlegung $z=uvw$ mit $|v|\geq 1$ und $|uv|\leq n$, so dass:\\
			~~~~für jede Zahl $k\geq 0$ gilt: $u v^k w\in\Slang{L}$
		}
		
		\emph{Beweis:} Sei $\Smach{M}$ ein DFA für $\Slang{L}$ mit $|Q|$ Zuständen. 
		Wir wählen $n=|Q|+1$.\\
		Ein akzeptierender Lauf für ein beliebiges Wort $z$ mit $|z|=\ell\geq n$
		muss in den ersten $n$ Schritten einen Zustand $p$ zweimal besuchen (sagen wir: nach $i$ und $j$ Schritten),
		hat also die Form:
		% Es gibt also Zahlen mit $0\leq i<j\leq n$ für die der Lauf folgende Form hat:
		% Der Lauf hat also folgende Form:
		%
		\[ q_0 \gstackrel{z_1}{\to}q_1\gstackrel{z_2}{\to}\!\ldots\!\gstackrel{z_{i-1}}{\to}q_{i-1}\gstackrel{z_i}{\to} p\bstackrel{z_{i+1}}{\to} q_{i+1}\bstackrel{z_{i+2}}{\to}\!\ldots\!\bstackrel{z_{j-1}}{\to}q_{j-1}\bstackrel{z_j}{\to} p \rstackrel{z_{j+1}}{\to}q_{j+1}\rstackrel{z_{j+2}}{\to}\!\ldots\!\rstackrel{z_\ell}{\to}q_\ell\]
		%
		Die gesuchte Zerlegung ist $u\,{=}\,\textcolor{darkgreen}{z_1\cdots z_i}$, $v\,{=}\,\textcolor{darkblue}{z_{i+1}\cdots z_{j}}$, $w\,{=}\,\textcolor{darkred}{z_{j+1}\cdots z_\ell}$.\\ 
		Der Lauf $(q_0\ldots q_{i-1} p) (q_{i+1}\ldots q_{j-1}p)^k (q_{j+1}\ldots q_\ell)$ akzeptiert $uv^k w$.
		% 
		\qed
	\end{frame}

	\begin{frame} \frametitle{Aufgabe 1}
		\small
%		Gegeben ist der NFA $\mathcal M=(\{q_0,q_1,q_2,q_3,q_4\},\{a,b,c,d\}, \delta ,\{q_0\},\{q_2\})$ mit $\delta$:
		\begin{center}
			\scalebox{0.7}{%
			\begin{tikzpicture} [->, >=stealth', initial text=, auto, node
				distance=25mm, bend angle=20, semithick]%[node distance=2cm,auto]
				
				\node[state,initial] (q_0) {$q_0$}; 
				\node[state] (q_1) [above right of=q_0] {$q_1$}; 
				\node[state,accepting] (q_2) [right of=q_1] {$q_2$}; 
				\node[state] (q_3) [below right of=q_0] {$q_3$};
				\node[state] (q_4) [right of=q_3] {$q_4$};
				
				\path[->]
				(q_0) edge [bend left]  node {$a, b$} (q_1) 
				(q_0) edge  [bend right] node {$b, c$} (q_3) 
				(q_1) edge [loop above] node {$c$} (q_1) 
				(q_1) edge node {$a$} (q_2) 
				(q_1) edge node {$d$} (q_3) 
				(q_3) edge node {$c$} (q_2) 
				(q_3) edge [bend left] node {$c$} (q_4) 
				(q_4) edge  node {$a$} (q_2)
				(q_4) edge [bend left] node {$d$} (q_3);
			\end{tikzpicture}}
		\end{center}
		
		\begin{enumerate}[a)]
			\item Geben Sie für jedes $z\in \{bc,adc,cda,bcdc,acdc\}$ alle Zerlegungen $z=uvw$ mit
			$u,w\in \Sigma^\ast$, $v\in \Sigma^{+}$ an, sodass für alle $k\ge 0$ gilt:
			$uv^kw\in L(\mathcal M)$. Begründen Sie Ihre Antworten.
			\item Ermitteln Sie eine Zahl $n \in \mathbb{N}$, sodass für alle $z \in L(\mathcal{M})$ mit $\abs{z} \ge n$ gilt, dass eine Zerlegung $z = uvw$ mit $u, w \in \Sigma^\ast$, $v \in \Sigma^+$ und $\abs{uv} \le n$ existiert, sodass für alle $k \ge 0$ gilt: $uv^kw \in L(\mathcal{M})$.
		\end{enumerate}
	\end{frame}

	\begin{frame}
		\small
		\begin{enumerate}[a)]
			\item Idee: Aufpumpen entspricht Zyklen/Schleifen in $\mathcal{M}$ \\
			$\leadsto$ möglich durch Schleife in $q_1$ oder Zyklus $q_3 \to q_4 \to q_3$
			\begin{itemize}
				\item Für $z \in \menge{bc, adc}$ existiert keine Zerlegung (beachte, dass stets auch $k=0$ zulässig sein muss)
				\item $z = cda \notin L(\mathcal{M})$
				\item $z = bcdc \leadsto b \mid c \mid dc \text{ oder }
				                         b \mid cd \mid c \text{ oder }
				                         bc \mid dc \mid \epsilon$
				\item $z = acdc \leadsto a \mid c \mid dc$
			\end{itemize}
			\item Laut Beweis des Pumping-Lemmas ist $n = \abs{Q} + 1 = 6$ zulässig. Tatsächlich reicht auch $n = 5$, da ein Lauf mit $i$ Übergängen automatisch $i+1$ Zustände besucht. Auch $n = 4$ ist ausreichend, wenn man sich überlegt wie Wörter der Länge $4$ aussehen dürfen:
			\begin{itemize}
				\item Weg über $q_3$: nutze Zyklus $q_3 \to q_4 \to q_3$
				\item Weg über $q_1$ und $q_3$ -- Variante 1: nutze Loop in $q_1$
				\item Weg über $q_1$ und $q_3$ -- Variante 2: nutze Zyklus $q_3 \to q_4 \to q_3$ (evtl. mit Start in $q_1$, d.h. $q_1 \to q_3 \to q_4 \to q_3 \to \dots \to q_4 \to q_2$)
			\end{itemize} 
		\end{enumerate}
	\solutionmarkB
	\end{frame}

	\section{Aufgabe 2: \\ \itshape Regularität von Sprachen}
	
	\begin{frame} \frametitle{Beweis von Nichtregularität}
		\small
		\theobox{\textbf{Satz (Myhill \& Nerode):} Eine Sprache $\Slang{L}$ ist genau dann regulär, wenn $\simeq_{\Slang{L}}$ endlich viele Äquivalenzklassen hat.}
		
		\theobox{\textbf{Satz:} Wenn $\Slang{L}_1$ und $\Slang{L}_2$ regulär sind, dann auch $\Slang{L}_1\cap \Slang{L}_2$,
			$\Slang{L}_1\cup \Slang{L}_2$, $\Slang{L}_1^\ast$ und $\overline{\Slang{L}}_1$.}
		
		\theobox{\textbf{Satz (Pumping-Lemma):}
			Für jede reguläre Sprache $\Slang{L}$\\
			gibt es eine Zahl $n\geq 0$, so dass gilt:\\
			~~für jedes Wort $x\in\Slang{L}$ mit $|x|\geq n$\\
			~~gibt es eine Zerlegung $x=uvw$ mit $|v|\geq 1$ und $|uv|\leq n$, so dass:\\
			~~~~für jede Zahl $k\geq 0$ gilt: $u v^k w\in\Slang{L}$
		}
	\end{frame}

	\begin{frame} \frametitle{Aufgabe 2}
		\small
		Gegeben ist das Alphabet $\Sigma = \menge{a,b}$. Welche der folgenden Sprachen $L_j$ über $\Sigma$ mit $1 \le j \le 2$ ist regulär? Beweisen Sie Ihre jeweilige Antwort.
		\begin{enumerate}[a)]
			\item $L_1 = \menge{a^i b^i : 1 \le i \le 15}$
			\item $L_2 = \menge{a^n b^m a^{n*m} : n,m \ge 0}$
		\end{enumerate}
	\end{frame}

	\begin{frame}
		\small
		\begin{enumerate}[a)]
			\item Die Sprache $L_1$ ist endlich, da $i$ eine obere Grenze hat. Jede endliche Sprache ist regulär.
			\item $L_2$ ist nicht regulär --- Pumping-Lemma: Angenommen $L_2$ sei regulär. Dann existiert nach dem Pumping-Lemma ein $n \ge 0$, sodass jedes Wort $x \in L_2$ mit $\abs{x} \ge n$ gepumpt werden kann. Insbesondere muss dies auch für das Wort 
			$x = a^n b^1 a^{n * 1}$
			gelten. Dementsprechend muss es eine Zerlegung
			\begin{equation*}
				x = uvw \quad \text{ mit } \abs{uv} \le n \text{ und } \abs{v} \ge 1
			\end{equation*}
			geben. Wegen $\abs{uv} \le n$ muss $uv = a^\ell$ mit $1 \le \ell \le n$ gelten (d.h. $uv$ liegt in den ersten $a$'s). Ein Teil der $a$'s muss dem $v$ zugeschrieben werden, d.h. es gilt $v = a^h$ mit $1 \le h \le \ell$. 
			Damit können wir nun pumpen, insbesondere mit $k=2$:
			\begin{equation*}
				u v^2 w = a^{\ell - h} a^{2h} a^{n - \ell} b a^n = a^{\ell - h + 2h + n - \ell} b a^n= a^{n + h} b a^n \notin L_2
			\end{equation*}
			im Widerspruch zur Aussage des Pumping-Lemmas, was $u v^2 w \in L_2$ sichern würde. Damit kann die Annahme der Regularität nicht richtig gewesen sein und $L_2$ ist nicht regulär.
		\end{enumerate}
		\solutionmarkB
	\end{frame}


	
	\section{Aufgabe 3: \\ \itshape Wiederholung}
	
	\begin{frame} \frametitle{Aufgabe 3}
		\footnotesize
		Beweisen oder widerlegen Sie unter Verwendung von Resultaten aus der Vorlesung folgende Aussagen.
		\begin{enumerate}[a)]
			\item \only<2->{\xmark} F\"ur die Grammatik $G=(\{S,X,Y,Z\},\{a,b\},\{S\rightarrow Y,\;X\rightarrow b,\;Y\rightarrow aYYb,\;aY\rightarrow aZ,\;ZY\rightarrow ZX,\;Z\rightarrow a\},S)$ gilt: $abab\in L(G)$.
			\item \only<3->{\cmark} Kann eine Sprache $L$ von einem DFA erkannt werden, so gibt es auch einen
			$\varepsilon$-NFA $\mathcal M$ mit $L({\mathcal M})=L$.
			\item \only<4->{\cmark} F\"ur jeden NFA $\mathcal M$ mit Wort\"uberg\"angen gibt es einen \"aquivalenten NFA.
			\item \only<5->{\cmark} Es gibt eine regul\"are Sprache, f\"ur welche die Anzahl der \"Aquivalenzklassen der  {\emph{Nerode}}-Rechtskongruenz endlich ist.
			\item \only<6->{\cmark} Wenn es f\"ur eine Sprache $L$ ein $n\in \mathbb N$ gibt, so dass die {\emph{Nerode}}-Rechtskongruenz $\simeq_L$ höchstens $n$ Äquivalenzklassen hat, so
			kann $L$ von einem DFA erkannt werden.
			\item \only<7->{\cmark} F\"ur jede Sprache $L$ gilt: $\displaystyle L = \bigcup\limits_{u \in L} \ [u]_{\simeq_{L}}\;$.
		\end{enumerate}
		\onslide<2->{\solutionmarkFT}
	\end{frame}


	\section{Aufgabe 4 \\ \itshape Chomsky-Normalform}
	
	\begin{frame} \frametitle{Eliminieren von $\epsilon$-Regeln}
		\small
		\codebox{
			\textbf{Eingabe}: CFG $G=\tuple{V,\Sigma,P,S}$\\
			\textbf{Ausgabe}: $\epsilon$-freie CFG $G'=\tuple{V',\Sigma,P',S'}$ mit $\Slang{L}(G')=\Slang{L}(G)$

			\begin{itemize}
				\item Initialisiere $P' \defeq P$ und $V'\defeq V$
				% 
				\item Berechne $V_\epsilon = \{\Snterm{A}\in V\mid \Snterm{A}\Rightarrow^\ast \epsilon\}$
				% 
				\item Entferne alle $\epsilon$-Regeln aus $P'$
				% 
				\item Solange es in $P'$ eine Regel $\Snterm{B}\to x\Snterm{A}y$ gibt, mit\\[1ex]
				%
				\narrowcentering{ $\Snterm{A}\in V_\epsilon$\hfill $|x|+|y|\geq 1$\hfill $\Snterm{B}\to xy\notin P'$ }\\[1ex]
				%
				wähle eine solche Regel und setze $P'\defeq P'\cup \{\Snterm{B}\to xy\}$
				%
				\item Falls $S\in V_\epsilon$ dann definiere ein neues Startsymbol $S'\notin V$, setze $V'\defeq V'\cup\{S'\}$ und 
				$P'\defeq P'\cup\{S'\to S,S'\to\epsilon\}$.
				Falls $S\notin V_\epsilon$, dann verwenden wir einfach $S'\defeq S$ als Startsymbol.
			\end{itemize}
		}
	\end{frame}
	
	\begin{frame}\frametitle{Eliminierung von Kettenregeln}
		\small
		\defbox{Eine \textbf{Kettenregel} ist eine Regel der Form $\Snterm{A}\to\Snterm{B}$.}
		
		Sei $G=\tuple{V,\Sigma,P,S}$ $\epsilon$-frei. Eine äquivalente Grammatik ohne Kettenregeln ist gegeben durch $G' = \tuple{V, \Sigma, P',S}$:
		
		\codebox{%
			\textbf{Eliminieren von Kettenregeln:} 
			
			$E(A) \dots$ Menge aller $B\in V$, die man von $A \in V$ aus über Kettenregeln erreichen kann:
			\begin{enumerate}[(1)]
				\item $A\in E(A)$
				\item Falls $B\in E(A)$ und $B\to B'\in P$ mit $B'\in V$ dann $B'\in E(A)$. Wiederhole.
			\end{enumerate}
			\begin{equation*}
				\Longrightarrow
				P' = \bigcup_{A\in V} \Big\{ A\to w\mid \text{es gibt }B\to w\in P\text{ mit }w\notin V\text{ und }B\in E(A) \Big\}
			\end{equation*}
		}
	\end{frame}
	
	\begin{frame}\frametitle{Die Chomsky-Normalform}	
		\small	
		\defbox{Eine kontextfreie Grammatik $G=\tuple{V,\Sigma,P,S}$ ist in \redalert{Chomsky-Normalform} \redalert{(CNF)}, wenn alle ihre Produktionsregeln eine der beiden folgenden Formen haben:%
			\[ \Snterm{A}\to \Snterm{BC} \quad\text{ (mit $\Snterm{B},\Snterm{C}\in V$)} \qquad\text{oder}\qquad \Snterm{A}\to \Sterm{c}\quad\text{ (mit $\Sterm{c}\in \Sigma$)}\]
		}
	
		\textbf{Umwandlung in CNF}:
		\begin{enumerate}[(1)]
			\item Eliminierung von $\epsilon$-Regeln
			\item Eliminierung von Kettenregeln
		\end{enumerate}
	\end{frame}

	\begin{frame}
		\small
%		\textbf{Umwandlung in CNF}:
		\begin{enumerate}[(1)]
%			\item Eliminierung von $\epsilon$-Regeln
%			\item Eliminierung von Kettenregeln
			\setcounter{enumi}{2}
			\item Extrahiere Regeln der Form $\Snterm{A} \to \Sterm{c}$, so dass alle anderen Regeln $\Snterm{B}\to w$ keine Terminale mehr in $w$ enthalten.
			\begin{itemize}
				\item für jedes Symbol $\Sterm{a}\in\Sigma$: \\
				neue Variable $\Sntermsub{V}{a}$ mit Regel $\Sntermsub{V}{a}\to \Sterm{a}$
				\item für Regeln $\Snterm{A}\to w$ mit $|w|>1$: \\ 
				ersetze jedes Vorkommen von $\Sterm{a}\in\Sigma$ in $w$ durch $\Sntermsub{V}{a}$
			\end{itemize}
			\item Reduziere Regeln der Form $\Snterm{A}\to\Sntermsub{B}{1}\cdots\Sntermsub{B}{n}$ auf $n = 2$
			
			Für jede Produktionsregel $\Snterm{A}\to \Sntermsub{B}{1}\cdots\Sntermsub{B}{n}$ mit $n>2$:\\
			\begin{itemize}
				\item Führe $n-2$ neue Variablen $\Sntermsub{C}{1},\ldots,\Sntermsub{C}{n-2}$ ein
				\item Ersetze die Regel durch neue Regeln:
				\begin{align*}
					\Snterm{A} &\to \Sntermsub{B}{1}\Sntermsub{C}{1} \\
					\Sntermsub{C}{1} &\to \Sntermsub{B}{2}\Sntermsub{C}{2} \\
					&\vdots \\
					\Sntermsub{C}{n-3} &\to \Sntermsub{B}{n-2}\Sntermsub{C}{n-2} \\
					\Sntermsub{C}{n-2} &\to \Sntermsub{B}{n-1}\Sntermsub{B}{n}
				\end{align*}
			\end{itemize}
		\end{enumerate}
	\end{frame}

	
	\begin{frame} \frametitle{Aufgabe 4}
		\small
		Betrachten Sie die Grammatik
		$G_0 = \tuple{V, \Sigma, P, S}$ mit $V = \menge{S, T, U, V, R}$, $\Sigma = \menge{a,b}$ und
		\begin{align*}
			P = \{ \quad 
			S &\to \varepsilon, & S &\to aSb, & S &\to T, & S &\to R, \\
			T &\to bbT, & T &\to U, & U &\to aaU, & U &\to bb T, \\
			V &\to bSa, &   &       & R &\to \varepsilon, & R &\to bSa \quad \}
		\end{align*}
		\vspace{-\baselineskip}
		\begin{enumerate}[a)]
			\item Konstruieren Sie eine Grammatik $G_1$, die
			keine Regeln der Form $A \to \varepsilon$ für $A\in V$ enthält.
			Erweitern Sie dazu, wenn nötig, die Grammatik $G_0$ um ein neues
			Startsymbol $S'$ und entsprechende Regeln.
			\item Geben Sie zu $G_1$ eine äquivalente Grammatik $G_2$ an, die
			keine Kettenregeln, also Produktionen der Form $A \to B$ mit
			Nichtterminalsymbolen $A,B$, enthält. 
			\item Geben Sie eine Grammatik $G_3$ in
			Chomsky-Normalform an mit $L(G_3) = L(G_2) \setminus \{\varepsilon\}$.
		\end{enumerate}
	\end{frame}

	\begin{frame}
		\small
		Zur Vereinfachung entfernen wir nicht erreichbare Symbole ($V$) und nicht terminierende Symbole ($T$, $U$).\footnote{Diese Vereinfachung ist nicht Bestandteil des Algorithmus.}
		Damit sieht die Regelmenge in Kurznotation wie folgt aus:
		\begin{equation*}
			P = \menge{ \quad 
				\begin{array}{lcl}
					S &\to& \epsilon \mid aSb \mid R \\
					R &\to& \epsilon \mid bSa 
				\end{array}
			  \quad }
		\end{equation*}
		a) \textbf{Eliminieren der $\epsilon$-Regeln} laut Algorithmus: $V_\epsilon = \menge{S,A}$
		\begin{equation*}
			P_1 = \menge{ \quad 
				\begin{array}{lcl}
					S &\to& aSb \mid R \mid ab \\
					R &\to& bSa \mid ba \\
					S' &\to& \epsilon \mid S
				\end{array}
				\quad }
		\end{equation*}
		
		Damit erhalten wir $G_1 = \tuple{V_1, \Sigma, P_1, S'}$ mit $V_1 = \menge{S, R, S'}$ und $P_1$ von oben.
		 \solutionmarkB
	\end{frame}

	\newcommand{\blue}[1]{\textcolor{darkblue}{#1}}
	\newcommand{\purple}[1]{\textcolor{cdpurple}{#1}}
	\newcommand{\green}[1]{\textcolor{cddarkgreen}{#1}}

	\begin{frame}
		\small
		b) \textbf{Eliminieren von Kettenregeln}:
		Erreichbarkeitsmengen: 
		\begin{equation*}
			E(S) = \menge{S, R}, \qquad E(R) = \menge{R}, \qquad E(S') = \menge{S',S,R}
		\end{equation*}
		\codebox{%
			$\displaystyle P_{\text{neu}} = \bigcup_{\purple{A} \in V} \Big\{ \purple{A} \to w\mid \text{es gibt }\blue{B}\to w\in P\text{ mit }w\notin V\text{ und } \blue{B} \in E(A) \Big\}$%
		}
		\begin{equation*}
			P_2 = \menge{ \quad 
				\begin{array}{lclrl}
					S &\to& \underbrace{aSb \mid ab}_{\blue{B = S}} \mid \underbrace{bSa \mid ba}_{\blue{B = R}} &&\purple{A = S}\\
					R &\to& \underbrace{bSa \mid ba}_{\blue{B = R}} &&\purple{A = R}\\
					S' &\to& \underbrace{\epsilon}_{\blue{B = S'}} \mid \underbrace{aSb \mid ab}_{\blue{B = S}} \mid \underbrace{bSa \mid ba}_{\blue{B = R}} && \purple{A = S'}
				\end{array}
				\quad }
		\end{equation*}
		
		Damit erhalten wir $G_2 = \tuple{V_1, \Sigma, P_2, S'}$ mit $V_1 = \menge{S, R, S'}$ wie bisher und $P_2$ von oben.
		\solutionmarkB
	\end{frame}

	\begin{frame}
		\small
		c) \textbf{Chomsky-Normalform}:
		
		Extrahieren von Terminalen
			\begin{equation*}
				\phantom{P_2 = }\menge{ \quad 
					\begin{array}{lclrl}
					S &\to& V_a \purple{S V_b} \mid V_a V_b \mid V_b \blue{S V_a} \mid V_b V_a\\
					R &\to& V_b \blue{S V_a} \mid V_b V_a \\
					S' &\to& \epsilon \mid V_a \purple{S V_b} \mid V_a V_b \mid V_b \blue{S V_a} \mid V_b V_a \\
					V_a &\to& a \\
					V_b &\to& b
					\end{array}
					\quad }
			\end{equation*}
		Verkürzen der Nichtterminalsequenzen
			\begin{equation*}
			P_3' = \menge{ \quad 
				\begin{array}{lclrl}
				S &\to& V_a \purple{C_b} \mid V_a V_b \mid V_b \blue{C_a} \mid V_b V_a\\
				R &\to& V_b \blue{C_a} \mid V_b V_a \\
				S' &\to& \epsilon \mid V_a \purple{C_b} \mid V_a V_b \mid V_b \blue{C_a} \mid V_b V_a \\
				V_a &\to& a \\
				V_b &\to& b \\
				\blue{C_a} &\to& SV_a \\
				\purple{C_b} &\to& SV_b
				\end{array}
				\quad }
			\end{equation*}
			Damit ist $G_3 = \tuple{V_3, \Sigma, P_3, S'}$ mit $V_3 = \menge{S',S,R,V_a,V_b,C_a,C_b}$ und $P_3 = P_3' \setminus \menge{S' \to \epsilon}$ in Chomsky-Normalform. Es gilt jedoch $L(G_3) = L(G_2) \setminus \menge{\epsilon}$.
			\solutionmarkB
	\end{frame}

	\section{Aufgabe 5 \\ \itshape CYK-Algorithmus}
	
	\newcommand{\Rightarrowstarquest}{\mathrel{{\stackrel{?}{\Rightarrow}}{}^\ast}}

	\begin{frame}\frametitle{CYK: Grundidee}
		\small
		\textbf{gegeben}: kontextfreie Grammatik $G$ in CNF \\
		\textbf{Frage:} $w = \Sterm{a_1}\cdots\Sterm{a_n} \in \Slang{L}(G)$?
		
		\begin{itemize}
			\item Falls $|w|=1$, dann ist $w\in\Sigma$ und es gilt:\\
			$w\in\Slang{L}(G)$ genau dann wenn es eine Regel $\Snterm{S}\to w$ in $G$ gibt
			\item Falls $|w|>1$, dann ist:\\
			$w\in\Slang{L}(G)$ genau dann wenn es eine Regel
			$\Snterm{S}\to\Snterm{A}\Snterm{B}$ und eine Zahl $i$ gibt, so dass gilt
			\[\Snterm{A}\Rightarrow^\ast \Sterm{a_1}\cdots\Sterm{a_i}\qquad\text{und}\qquad \Snterm{B}\Rightarrow^\ast \Sterm{a_{i+1}}\cdots\Sterm{a_n}\]
		\end{itemize}
		
		\textbf{Idee}: Fall 2 reduziert das Problem $\Snterm{S}\Rightarrowstarquest w$ auf zwei einfachere Probleme
		$\Snterm{A}\Rightarrowstarquest \Sterm{a_1}\cdots\Sterm{a_i}$ und $\Snterm{B}\Rightarrowstarquest \Sterm{a_{i+1}}\cdots\Sterm{a_n}$, die man allerdings für alle Regeln $\Snterm{S}\to\Snterm{A}\Snterm{B}$ und Indizes $i$ lösen muss
		
	\end{frame}
	
	\newcommand{\MEC}[1]{\multicolumn{1}{c}{#1}}
	
	\begin{frame}\frametitle{CYK: Praktische Umsetzung}
		\small
		\textbf{Vorgehen}: $V[i,j] = $ Menge aller $\Snterm{A}$ mit $\Snterm{A}\Rightarrow^\ast w_{i,j}$ \\
		$\leadsto$ $V[i,j]$ können in einer Dreiecksmatrix notiert werden
		
		\begin{itemize}
			\item Diagonale = Fall 1: existiert Terminalsymbolregel
			\item Fixiere Element $\blacksquare$: sei $\blacktriangleleft$ in der gleichen Zeile ganz links und $\blacktriangledown$ direkt unten drunter 
			\begin{itemize}
				\item wenn eine Regel $\blacklozenge \to \blacktriangleleft \blacktriangledown$, dann füge $\blacklozenge$ zu $\blacksquare$ hinzu
				\item schiebe $\blacktriangleleft$ nach rechts und $\blacktriangledown$ nach unten und wiederhole
			\end{itemize}
		\end{itemize}
		Ist am Ende das Startsymbol $S \in \alert{V[1,|w|]}$, dann liegt $w$ in der Sprache
		
		\textbf{Beispiel}: Wir betrachten das Wort $w=\Sterm{a+b*c}$ der Länge $|w|=5$.		
		\begin{center}
			\scalebox{0.7}{%
			\begin{tabular}{c|c|c|c|c|c|}
				%  \multicolumn{1}{c}{}  & \multicolumn{1}{c}{A} & \multicolumn{1}{c}{B} & \multicolumn{1}{c}{C} & \multicolumn{1}{c}{D}  \\\cline{2-5}
				\cline{2-6}
				\Sterm{a}
				& $V[1,1]$ & $V[1,2]$ & $V[1,3]$ & $V[1,4]$ & \alert{$V[1,5]$} \\\cline{2-6}
				\MEC{\Sterm{+}} &  
				& $\blacktriangleleft$ & $V[2,3]$ & $\blacksquare \cup \blacklozenge$ & $V[2,5]$ \\\cline{3-6}
				\MEC{\Sterm{b}} &\MEC{}   &
				& $V[3,3]$ & $\blacktriangledown$ & $V[3,5]$ \\\cline{4-6}
				\MEC{\Sterm{*}} &\MEC{}   &\MEC{}   &   
				& $V[4,4]$ & $V[4,5]$ \\\cline{5-6}
				\MEC{\Sterm{c}} &\MEC{}   &\MEC{}   &\MEC{}   &
				& $V[5,5]$ \\\cline{6-6}
				\MEC{}          &\MEC{\Sterm{a}} & \MEC{\Sterm{+}} & \MEC{\Sterm{b}} & \MEC{\Sterm{*}} & \MEC{\Sterm{c}}
			\end{tabular}}
		\end{center}
	\end{frame}
	
	\begin{frame} \frametitle{Aufgabe 5}
		\small
		Gegeben ist folgende Grammatik $G=(V,\Sigma,P,S)$ mit 
		$V=\{S,X,M,A,B\}$, $\Sigma=\{a,b\}$ und
		\begin{align*}
			P = \{ \quad S &\to \varepsilon, & S &\to AX, & S &\to AB, \\
			X &\to MB, \\
			M &\to AB, & M &\to AX, \\
			A &\to a, \\
			B &\to a, & B &\to b \rlap{\qquad \}} 
		\end{align*}
		Verwenden Sie den CYK-Algorithmus (mit der Matrix-Notation aus der
		Vorlesung), um für die folgenden Wörter $w_i$ zu entscheiden, ob
		$w_i\in L(G)$ ist. 
		
		\begin{enumerate}[a)]
			\item $w_1 = aaabba$
			\item $w_2 = aabbaa$
		\end{enumerate}    
	\end{frame}

	\begin{frame}
		\small
		\begin{enumerate}[a)]
			\item $w_1 = aaabba$
			\begin{center}
				\scalebox{0.8}{%
					\begin{tabular}{c|c|c|c|c|c|c|}
						%  \multicolumn{1}{c}{}  & \multicolumn{1}{c}{A} & \multicolumn{1}{c}{B} & \multicolumn{1}{c}{C} & \multicolumn{1}{c}{D}  \\\cline{2-5}
						\cline{2-7}
						a
						& $A,B$ & $S,M$ & $X$ & $S,M$ & $X$ & $\alert{S}, M$ \\\cline{2-7}
						\MEC{a} &  
						& $A,B$ & $S,M$ & $X$ & $S,M$ & $X$ \\\cline{3-7}
						\MEC{a} &\MEC{}   &
						& $A,B$ & $S,M$ & $X$ & $\emptyset$ \\\cline{4-7}
						\MEC{b} &\MEC{}   &\MEC{}   &   
						& $B$ & $\emptyset$ & $\emptyset$ \\\cline{5-7}
						\MEC{b} &\MEC{}   &\MEC{}   &\MEC{}   &
						& $B$ & $\emptyset$ \\\cline{6-7}
						\MEC{a} & \MEC{} & \MEC{} & \MEC{} & \MEC{} &  & $A,B$ \\\cline{7-7}
						\MEC{}          &\MEC{a} & \MEC{a} & \MEC{a} & \MEC{b} & \MEC{b} & \MEC{a}
				\end{tabular}}
			\end{center}
			$\Rightarrow w_1 \in L(G)$
			
			\item $w_2 = aabbaa$
			\begin{center}
				\scalebox{0.8}{%
					\begin{tabular}{c|c|c|c|c|c|c|}
						%  \multicolumn{1}{c}{}  & \multicolumn{1}{c}{A} & \multicolumn{1}{c}{B} & \multicolumn{1}{c}{C} & \multicolumn{1}{c}{D}  \\\cline{2-5}
						\cline{2-7}
						a
						& $A,B$ & $S,M$ & $X$ & $S,M$ & $X$ & \alert{$\emptyset$} \\\cline{2-7}
						\MEC{a} &  
						& $A,B$ & $S,M$ & $X$ & $\emptyset$ & $\emptyset$ \\\cline{3-7}
						\MEC{b} &\MEC{}   &
						& $B$ & $\emptyset$ & $\emptyset$ & $\emptyset$ \\\cline{4-7}
						\MEC{b} &\MEC{}   &\MEC{}   &   
						& $B$ & $\emptyset$ & $\emptyset$ \\\cline{5-7}
						\MEC{a} &\MEC{}   &\MEC{}   &\MEC{}   &
						& $A,B$ & $S,M$ \\\cline{6-7}
						\MEC{a} & \MEC{} & \MEC{} & \MEC{} & \MEC{} &  & $A,B$ \\\cline{7-7}
						\MEC{}          &\MEC{a} & \MEC{a} & \MEC{b} & \MEC{b} & \MEC{a} & \MEC{a}
				\end{tabular}}
			\end{center}
		$\Rightarrow w_2 \notin L(G)$
		\end{enumerate}
		\solutionmarkB		
	\end{frame}

\end{document}