\documentclass{beamer}
\usepackage{../tut-slides}
\usepackage{../mathoperators}
\usepackage{../fs}

\usepackage{csquotes}

\usepackage{amsmath,amssymb}
%\usepackage{enumerate}
\usepackage[normalem]{ulem}
\newcommand{\labelitemi}{\raisebox{1pt}{\scalebox{.9}{$\blacktriangleright$}}}
\newcommand{\labelitemii}{$\vartriangleright$}
\newcommand{\labelitemiii}{--}

\usepackage{booktabs}
\usepackage{tabularx}

\newcommand{\tuple}[1]{\langle{#1}\rangle}
\newcommand{\simquot}[1]{#1/_{\!\!{\sim}}}

\usepackage{pifont}

\newcommand{\cmark}{\textcolor{cddarkgreen}{\ding{51}}}%
\newcommand{\xmark}{\textcolor{darkred}{\ding{55}}}%

\newcommand{\solutionmarkB}{%
	\begin{tikzpicture}[remember picture, overlay]
		\node [
			fill=none,  % Farbe des Randstreifens
			text=cdorange,  % Textfarbe
			font=\fosfamily\bfseries\large,  % Einstellungen für die Schrift
			inner xsep=0,       % Abstand des Textes von unten
			% maximale Textbreite = Papierhöhe - 2*Abstand des Textes von unten:
			%			text width={\dimexpr\paperheight-2\footskip\relax},
			align=center,
			%			minimum height=8mm,% Breite des Randstreifens
			anchor=south west,
			rotate=90
		] at ($(current page.north west)+(+8mm,-20mm)$)
		{Lösung};
		\draw[draw, line width=2pt, color=cdorange] ($(current page.north west)+(+1mm,0)$) -- ($(current page.south west)+(+1mm,0)$);
	\end{tikzpicture}%
}
\newcommand{\solutionmarkFT}{\begin{tikzpicture}[remember picture, overlay]
	\node [
	fill=none,  % Farbe des Randstreifens
	text=cdorange,  % Textfarbe
	font=\fosfamily\bfseries\large,  % Einstellungen für die Schrift
	inner xsep=0,       % Abstand des Textes von unten
	% maximale Textbreite = Papierhöhe - 2*Abstand des Textes von unten:
	%			text width={\dimexpr\paperheight-2\footskip\relax},
	align=center,
	%			minimum height=8mm,% Breite des Randstreifens
	anchor=south west,
	rotate=90
	] at ($(current page.north west)+(+8mm,-27mm)$)
	{Lösung};
	\draw[draw, line width=2pt, color=cdorange] ($(current page.north west)+(+1mm,0)$) -- ($(current page.south west)+(+1mm,0)$);
	\end{tikzpicture}%
}

\usepackage{tikzducks}
\newcommand{\duckA}{%
	\scalebox{0.2}{
		\reflectbox{
	\begin{tikzpicture}[baseline=2ex]
	\duck[mask=teal,cape=teal, body=orange!60!yellow]
	\end{tikzpicture}%
	}
	}
}
\newcommand{\duckB}{%
	\scalebox{0.2}{
	\begin{tikzpicture}[baseline=2ex]
	\duck[body=pink,
	unicorn=magenta!60!violet,
	longhair=magenta!60!violet]
	\end{tikzpicture}%
	}
}
\newcommand{\duckC}{%
	\scalebox{0.2}{
	\begin{tikzpicture}[baseline=2ex]
	\duck[santa=red!80!black,
	beard=white!80!brown]
	\end{tikzpicture}%
	}
}
\newcommand{\duckD}{%
	\scalebox{0.2}{
	\begin{tikzpicture}[baseline=2ex]
	\duck[signpost=420, sunglasses=black!90, body=brown!60!white]
	\end{tikzpicture}%
	}
}


%%%%%%%%%%%%%%%%%%%%%%%%%%%%%%%%%%%%%%%%%%%%%%%%%%%%%%%%%%%%%%%%%%%%%%%
\newcommand{\ghost}[1]{\raisebox{0pt}[0pt][0pt]{\makebox[0pt][l]{#1}}}
\newcommand{\blue}[1]{\textcolor{darkblue}{#1}}
\newcommand{\purple}[1]{\textcolor{cdpurple}{#1}}
\newcommand{\green}[1]{\textcolor{cddarkgreen}{#1}}



\begin{document}	
	\title{Formale Systeme}
	\subtitle{Übung 8}
	\author{Eric Kunze}
	\email{eric.kunze@tu-dresden.de}
	\city{TU Dresden}
	\date{\formatdate{10}{12}{2021}}
%	\institute{Lehrstuhl für Grundlagen der Programmierung}
	\titlegraphic{\includegraphics[width=2cm]{../TUD-white.pdf}}

	\maketitle

%	\begin{frame} \frametitle{Corona-Umfrage}
%		\textbf{Wie stehst du zu Online-Übungen?}
%		\begin{itemize}
%			\item Ich bin für Online-Übungen. \\
%			\textit{nur online}
%			\item Ich kann mit Online-Übungen leben. \\ 
%			\textit{beides okay, bevorzugt online}
%			\item Ich kann mit Präsenzübungen leben. \\
%			\textit{beides okay, bevorzugt Präsenz}
%			\item Ich bin für Präsenzübungen. \\
%			\textit{nur Präsenz}
%		\end{itemize}
%	
%		\begin{center}
%			\Large
%			\fbox{\url{https://tudvote.tu-dresden.de/88314}}
%		\end{center}
%	\end{frame}

	\begin{frame} \frametitle{Übungsblatt 8}
		\tableofcontents
	\end{frame}

	\section{Aufgabe 1: \\ \itshape Kellerautomaten}

	\begin{frame}\frametitle{Kellerautomaten}
		\small			
		\defbox{Ein \textbf{Kellerautomat} (PDA) 
			\Smach{M} ist ein Tupel $\Smach{M}=\tuple{Q,\Sigma,\Gamma,\delta,Q_0,F}$:
			\begin{itemize}
				\item $Q$: endliche Menge von Zuständen
				\item $\Sigma$: Eingabealphabet
				\item $\Gamma$: \textit{Kelleralphabet}
				\item $\delta$: totale Übergangsfunktion $Q\times\Sigma_\epsilon\times\Gamma_\epsilon \to 2^{Q\times\Gamma_\epsilon}$
				\item $Q_0$: Menge möglicher Startzustände $Q_0\subseteq Q$
				\item $F$: Menge von Endzuständen $F\subseteq Q$
			\end{itemize}
		} \pause
	
		$\tuple{q_2,\Snterm{D}}\in\delta(q_1,\Sterm{a},\Snterm{C})$ bedeutet:
		"`Wenn der PDA in Zustand $q_1$ das Symbol \Sterm{a} einliest und $\Snterm{C}$ oben vom Keller nimmt (\texttt{pop}),
		dann \textit{kann} er in Zustand $q_2$ wechseln und dabei $\Snterm{D}$ auf den Keller legen (\texttt{push})."' 
		
		\begin{center}
			\vspace{-1.5em}
			\scalebox{1}{%
				\begin{tikzpicture} [->, >=stealth', initial text=, auto, node
					distance=35mm, bend angle=20, semithick]%[node distance=2cm,auto]
					
					\node[state] (q_1) {$q_1$}; 
					\node[state] (q_2) [right of=q_1] {$q_2$}; 
					
					\path[->]
					(q_1) edge node {$\Sterm{a}, \Snterm{C} \mapsto \Snterm{D}$} (q_2);
			\end{tikzpicture}}
		\end{center}
		
	\end{frame}	

	\begin{frame} \frametitle{Aufgabe 1}
		\small
		Geben Sie einen Kellerautomaten $\mathcal{M}_i$ für die Sprachen
		$L_i$ ($i = 1, \dots, 4$) sowie eine akzeptierende Folge von Konfigurationsübergängen für die gegebenen Wörter $w$.
		\begin{enumerate}[(a)]
			\item $L_0 = L(\mathcal{M}_0) = \{a^ib^jc^k \mid i=j \text{ oder } j=k \text{ mit } i,j,k\ge 1\}$
			 \\ $w=aaabbcc$ 
			\item $L_1 = L(\mathcal M_1)=\{a^{n}b^m\mid  n,m\ge 0\,, n=3m\}$ \\ $w=aaab$
			\item $L_2 = L(\mathcal M_2) = \{w\in\{a,b\}^\ast \mid |w|_a = |w|_b\}$ \\ $w=aabbba$ 
			\item $L_3 = L(\mathcal{M}_3) = \{ (ab)^{n}  (ba)^{n} \mid n \geq 0 \}$ \\ $w=ababbaba$ 
		\end{enumerate}
	\end{frame}

	

	\section{Aufgabe 2 \\ \itshape Permutationssprache kontextfrei?}
	
	\newcommand{\mytabnote}[2]{\ghost{#1}\hspace{2cm}{\textcolor{cdgray}{(#2)}}}

	

	\begin{frame} \frametitle{Aufgabe 2}
		\small
		Beweisen oder widerlegen Sie folgende Aussage.
		
		Ist $L \subseteq \Sigma^\ast $ eine kontextfreie Sprache, so ist auch
		$$\pi(L) = \left\{ a_1 \ldots a_n \in \Sigma^\ast : \begin{array}{l} \text{ex. Permutation } (i_1 \ldots i_n) \text{ von } (1 \ldots n), \\ \text{sodass }  a_{i_1} \ldots a_{i_n} \in L \end{array} \right\} $$
		kontextfrei.
	\end{frame}

	\begin{frame}\frametitle{Abschluss für kontextfreie Sprachen}
		\small
		\theobox{
			\textbf{Satz:} Wenn $\Slang{L}$, $\Slangsub{L}{1}$ und $\Slangsub{L}{2}$ kontextfreie Sprachen sind, dann beschreiben auch die folgenden Ausdrücke kontextfreie Sprachen: 
			
			$\Slangsub{L}{1}\cup\Slangsub{L}{2}$, $\Slangsub{L}{1}\circ\Slangsub{L}{2}$, $\Slang{L}^\ast$
		}
		
		Aber: 
		
		\theobox{
			\textbf{Satz:} Es gibt kontextfreie Sprachen $\Slang{L}$, $\Slangsub{L}{1}$ und $\Slangsub{L}{2}$, so dass die folgenden Ausdrücke keine kontextfreien Sprachen sind:
			
			$\Slangsub{L}{1}\cap\Slangsub{L}{2}$, $\overline{\Slang{L}}$
		}
		
		Aber (nicht in VL):
		\codebox{
			\textbf{Lemma:} Es sei $\Slang{L}$ kontextfrei und $\Slang{R}$ regulär. Dann ist $\Slang{L} \cap \Slang{R}$ kontextfrei.
		}
		
		\textit{Beweisidee:} Konstruiere einen Produktautomaten wie für reguläre Sprachen. Dieser wird wieder ein PDA sein.
	\end{frame}

	\begin{frame}
		\footnotesize
		Die Aussagen ist falsch. Wir betrachten die Sprache $L = \menge{(abc)^n : n \ge 0}$, welche bekanntermaßen regulär und insbesondere also kontextfrei ist. Ihre Permutationssprache ist
		\begin{equation*}
			\pi(L) = \menge{w \in \Sigma^\ast : \abs{w}_a = \abs{w}_b = \abs{w}_c}.
		\end{equation*}
		Wir wollen nun also zeigen, dass $\pi(L)$ nicht kontextfrei ist. 
		\begin{enumerate}[(i)]
			\item Wir nutzen Abschlusseigenschaften. Es gilt
			\begin{equation*}
				\pi(L) \cap \underbrace{\menge{a^n b^m c^k : n,m,k \ge 0}}_{\text{regulär}} = \underbrace{\menge{a^n b^n c^n : n \ge 0}}_{\text{nicht kontextfrei}}.
			\end{equation*}
			Angenommen $\pi(L)$ wäre kontextfrei. Dann ist $\menge{a^n b^n c^n : n \ge 0} = \text{ kontextfrei } \cap \text{ regulär}$ auch kontextfrei --- Widerspruch.
			\item Alternativ kann man das Pumping-Lemma bedienen. Dazu sei $n \ge 0$ die Zahl aus dem Pumping-Lemma und wir betrachten das Wort $z = a^n b^n c^n \in \pi(L)$. Damit läuft der Beweis wie das Musterbeispiel der Vorlesung für die Sprache $\menge{a^n b^n c^n : n \ge 0}$ (d.h. Fallunterscheidung nach möglichen Positionen der Pump-Region und in jedem Fall verändern sich die Anzahlen von $a$, $b$ und $c$ unterschiedlich).
		\end{enumerate}
		\onslide<1->{\solutionmarkB}
	\end{frame}

	\section{Aufgabe 3 \\ \itshape Deterministische Kellerautomaten}
	
	\begin{frame}\frametitle{Deterministische Kellerautomaten}
		\small
		\defbox{Ein \textbf{deterministischer Kellerautomat} (DPDA) 
			\Smach{M} ist ein Tupel $\Smach{M}=\tuple{Q,\Sigma,\Gamma,\delta,q_0,F}$ mit
			den folgenden Bestandteilen:
			\begin{itemize}
				\item $Q$: endliche Menge von Zuständen
				\item $\Sigma$: Eingabealphabet
				\item $\Gamma$: Kelleralphabet
				\item $\delta$: \textit{partielle} Übergangsfunktion $Q\times\Sigma_\epsilon\times\Gamma_\epsilon \to Q\times\Gamma_\epsilon$, 
				so dass für alle $q\in Q$, $\Sterm{a}\in\Sigma$ und $\Snterm{A}\in\Gamma$ jeweils nur eines der folgenden definiert ist: \\[1ex]
				\narrowcentering{$\delta(q,\Sterm{a},\Snterm{A})$\hfill$\delta(q,\Sterm{a},\epsilon)$\hfill$\delta(q,\epsilon,\Snterm{A})$\hfill$\delta(q,\epsilon,\epsilon)$}
				\item $q_0$: ein Startzustand $q_0 \in Q$
				\item $F$: Menge von Endzuständen $F \subseteq Q$
			\end{itemize}
		}
	\end{frame}
	
	\begin{frame} \frametitle{Aufgabe 3}
		\small
		Gegeben sei die Sprache $L=\{w\in \Sigma^\ast \;|\;|w|_a+|w|_b=|w|_c\}$ \"uber dem Alphabet $\Sigma =\{a,b,c\}$, wobei $|w|_a$ der Anzahl der Vorkommen von $a$ in $w$ entspricht.
		\begin{enumerate}[(a)]
			\item Entwerfen Sie einen Kellerautomaten $\mathcal{M}$ mit $L(\mathcal{M})=L$, der mittels Finalzustand akzeptiert.
			\item Welcher andere Akzeptanzbegriff f\"ur Kellerautomaten ist laut Anmerkung in der Vorlesung auch m\"oglich?
			\item Wann ist eine Sprache deterministisch kontextfrei? Ist $L$ deterministisch kontextfrei?
		\end{enumerate}
	\end{frame}

	\begin{frame}
		\setlength\arraycolsep{2pt}
		\begin{enumerate}[(a)]
			\item<+-> $\mathcal{M} \defeq \tuple{Q, \Sigma, \Gamma, \delta, \menge{q_0}, \menge{q_F}}$ mit $Q = \menge{q_0, q_1, q_2}$, $\Sigma = \menge{a, b, c}$, $\Gamma = \menge{S, A, C}$ und $\delta$:
			\begin{center}
				\scalebox{0.7}{%
					\begin{tikzpicture} [->, >=stealth', initial text=, auto, node
						distance=35mm, bend angle=20, semithick]%[node distance=2cm,auto]
						
						\node[state,initial] (q_0) {$q_0$}; 
						\node[state] (q_1) [right of=q_0] {$q_1$}; 
						\node[state,accepting] (q_F) [right of=q_1] {$q_F$}; 
						
						\path[->]
						(q_0) edge node {$\epsilon, \epsilon \mapsto S$} (q_1) 
						(q_1) edge node {$\epsilon, S \mapsto \epsilon$} (q_F)
						(q_1) edge [loop above] node {$\begin{array}{rl c rl} 
								a, &\epsilon \mapsto A, &\phantom{MI}& c, &A \mapsto \epsilon, \\
								b, &\epsilon \mapsto A, && a, &C \mapsto \epsilon, \\
								c, &\epsilon \mapsto C, && b, &C \mapsto \epsilon \\
							\end{array}$} (q_1) ;
				\end{tikzpicture}}
			\end{center}
			\item<+-> Akzeptanz mittels leerem Keller
			\item<+-> Eine Sprache $L$ heißt deterministisch kontextfrei, falls ein deterministischer Kellerautomat existiert, der $L$ akzeptiert.
			\begin{center}
				\scalebox{0.7}{%
					\begin{tikzpicture} [->, >=stealth', initial text=, auto, node
						distance=35mm, bend angle=20, semithick]%[node distance=2cm,auto]
						
						\node[state, initial, accepting] (q_0) {$q_0$}; 
						\node[state] (q_1) [right of=q_0] {$q_1$}; 
						
						\path[->]
						(q_0) edge[bend left] node {$\begin{array}{rl} 
								a, &\epsilon \mapsto AS, \\
								b, &\epsilon \mapsto AS, \\
								c, &\epsilon \mapsto CS, \\
							\end{array}$} (q_1) 
						(q_1) edge[bend left] node {$\epsilon, S \mapsto \epsilon$} (q_0)
						(q_1) edge [loop right] node {$\begin{array}{rl} 
								a, &A \mapsto AA,  \\
								a, &C \mapsto \epsilon, \\
								b, &A \mapsto AA, \\
								b, &C \mapsto \epsilon, \\
								c, &A \mapsto \epsilon, \\
								c, &C \mapsto CC \\
							\end{array}$} (q_1) ;
				\end{tikzpicture}}
			\end{center}
		\end{enumerate}
		\solutionmarkB
	\end{frame}

	\section{Aufgabe 4 \\ \itshape Wiederholung}
	
	\begin{frame} \frametitle{Aufgabe 4}
		\small
		Welche der folgenden Aussagen sind wahr und welche nicht? Begr"unden Sie Ihre
		Antworten -- dabei d"urfen Sie den gesamten Stoff und alle Resultate
		der Vorlesung und "Ubung verwenden.
		\begin{enumerate}[(a)]
			\item \only<2->{\cmark} Es gibt eine Sprache, die von einem nichtdeterministischen Kellerautomaten erkannt wird, nicht aber von einem deterministischen Kellerautomaten. \\
			\only<3->{\textit{Es gilt $\text{deterministisch } \subsetneq \text{ eindeutig } \subsetneq \text{ Typ 2}$.}}
			\item \only<4->{\xmark} Mithilfe des Pumping-Lemmas f\"ur kontextfreie Sprachen kann bewiesen werden, dass eine Sprache $L$ kontextfrei ist.\\
			\only<5->{\textit{Pumping ist notwendig, aber nicht hinreichend.}}
			\item \only<6->{\xmark} Für eine beliebige Sprache $L$ gilt: $L$ ist regul\"ar, wenn es eine nat\"urliche Zahl $n_0\ge 1$ gibt, so dass sich jedes Wort $w\in L$ mit $|w|\ge n_0$ zerlegen l\"asst in $w=xyz$ mit $y\not = \varepsilon, xy^kz\in L$ f\"ur alle $k\ge 0$. \\
			\only<7->{\textit{Pumping ist notwendig, aber nicht hinreichend.}}
		\end{enumerate}
		\onslide<2->{\solutionmarkFT}
	\end{frame}

	
	

\end{document}