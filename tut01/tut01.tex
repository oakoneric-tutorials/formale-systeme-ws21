\documentclass{beamer}
\usepackage{../tut-slides}
\usepackage{../mathoperators}

\usepackage{amsmath,amssymb}
\usepackage{enumerate}
\usepackage[inline]{enumitem} 		%customize label
\usepackage[normalem]{ulem}
\newcommand{\labelitemi}{\raisebox{1pt}{\scalebox{.9}{$\blacktriangleright$}}}
\newcommand{\labelitemii}{$\vartriangleright$}
\newcommand{\labelitemiii}{--}

\usepackage{booktabs}
\usepackage{tabularx}
\usepackage{tabu}
\newcommand*\head{\rowfont{\bfseries}}
\newcommand*{\tw}{\rowfont{\ttfamily}}

\renewcommand{\tabularxcolumn}[1]{>{\hspace{0pt}}m{#1}}


\begin{document}	
	\title{Formale Systeme}
	\subtitle{Übung 1: Einleitung}
	\author{Eric Kunze}
	\email{eric.kunze@tu-dresden.de}
	\city{TU Dresden}
	\date{}
%	\institute{Lehrstuhl für Grundlagen der Programmierung}
	\titlegraphic{\includegraphics[width=2cm]{../TUD-white.pdf}}

	\maketitle
	
	\begin{frame} \frametitle{Wer bin ich?}
		\begin{minipage}{\dimexpr0.75\linewidth-\fboxrule-\fboxsep}
			\begin{itemize}
				\item \uline{Eric} [Kunze]
				\item \url{eric.kunze@mailbox.tu-dresden.de}
				\item Fragen, Wünsche, Vorschläge, \dots 
			\end{itemize}
		\end{minipage}
		\begin{minipage}{\dimexpr0.25\linewidth-\fboxrule-\fboxsep}
			\includegraphics[width=3cm]{./tut01_pic.jpg}
		\end{minipage}		
		
		\begin{itemize}
			\item \textbf{Telegram:} \texttt{@oakoneric} bzw. \url{t.me/oakoneric}
		\end{itemize}
	\end{frame}

%	\begin{frame}\frametitle{Infos \& Co}
%		\textbf{Lehrveranstaltungswebsite}: \\
%		{\small \url{https://www.orchid.inf.tu-dresden.de/teaching/2020ws/aud/} \\
%			(Google: ''aud tu dresden``)}
%		
%		\textbf{OPAL-Kurs}: \\
%		{\small \url{https://bildungsportal.sachsen.de/opal/auth/RepositoryEntry/26663256067?5}}
%		
%		\begin{itemize}
%			\item alle Informationen zur Lehrveranstaltung
%			\item aktuelles Übungsblatt
%			\item Link zu Lösungsvorschlägen
%			\item Übungsverlegungen
%		\end{itemize}
%		
%		\textbf{Skript \& Aufgabensammlung}: Copyshop ''Die Kopie``
%	\end{frame}
%
%	\begin{frame} \frametitle{Corona \& Co.}
%		\textbf{Online-Lehre}
%		\begin{itemize}
%			\item Vorlesung als Video \textbf{freitags} im OPAL-Kurs
%			\item Übungsblatt \textbf{freitags} im OPAL-Kurs (und zeitnah auf meiner Website)
%			\item digitale Übungsbesprechung: Freitag, 2.DS (ausgewählte Aufgaben)
%			\item Korrektur von Aufgaben (nur ohne Präsenzübung)
%		\end{itemize}
%	
%		\textbf{Corona-Regelungen} {\tiny \itshape ohne Garantie -- Eigenverantwortlichkeit}
%		\begin{itemize}
%			\item Präsenzübung einschreibungspflichtig
%			\item Nicht-Erscheinen ohne Ankündigung $=$ Austragung
%			\item Wartelisteplätze rutschen nach
%			\item \textbf{Abmeldung} bei Verhinderung !!!
%			\item Teilnahmeliste
%			\item Maskenpflicht abseits des Platzes, Lüften, \dots
%		\end{itemize}
%	
%		
%	\end{frame}
%
%	\begin{frame} \frametitle{Vorlesung vs. Übung}
%		\centering
%		\textbf{Was ist eine Übung?}
%		
%		''Lehrveranstaltung an der Hochschule, in der etw., bes. das Anwenden von Grundkenntnissen, von den Studierenden geübt wird`` [Duden]
%		
%		\pause
%		
%		\begin{tabularx}{\linewidth}{X|X}
%			\hline
%			\textbf{Vorlesung} & \textbf{Übung} \\ \hline \hline
%			Vermittlung von neuem Wissen & Üben und Festigen des Stoffes der VL \\ \hline
%			hohes Tempo & (selbst definierbares) langsameres Tempo \\ \hline
%			wenig Interaktion & (sehr) viel Interaktion \\ \hline
%			$\approx 50\%$ Verständnis & $>80\%$ Verständnis \\
%			\hline
%		\end{tabularx}
%	\end{frame}
%
%	\begin{frame} \frametitle{Was wird in der Übung erwartet?}
%		\begin{center}
%			\bfseries
%			Es wird keine reine Vorrechenübung werden!
%		\end{center}
%		
%		\pause
%		
%		\begin{block}{Mein Input}
%			\begin{itemize}[leftmargin=2em, nolistsep]
%				\item Zusammenfassung einiger Vorlesungsinhalte
%				\item beispielhafte Lösungsansätze und Lösungen
%				\item Fragen, Fragen, Fragen
%			\end{itemize}
%		\end{block}
%	
%		\pause
%			%
%		\begin{block}{Euer Input}
%			\begin{itemize}[leftmargin=2em, nolistsep]
%				\item Grundverständnis aus der Vorlesung
%				\item Vorbereitung der Übungsaufgaben
%				\item aktive Mitarbeit und FRAGEN
%			\end{itemize}
%		\end{block}
%	\end{frame}
%
%	\begin{frame} \frametitle{Lösungen}
%		\textbf{Slides werden mit Sourcecode auf Github zur Verfügung stehen.}
%		\begin{itemize}[leftmargin=*]
%			\item \url{https://github.com/oakoneric/algorithmen-datenstrukturen-ws20}
%			\item \url{github.com} $\to$ oakoneric $\to$ algorithmen-datenstrukturen-ws20
%		\end{itemize}
%	\textbf{\dots und auf meiner Website:} \url{https://oakoneric.github.io}
%		\begin{itemize}
%			\item evtl. zusätzliche Materialien (nach Bedarf)
%			\item \alert{\textbf{kein Anspruch auf Vollständigkeit \& Korrektheit}}
%			\item gefundene Fehler melden
%		\end{itemize}
%	\end{frame}
%
%%%%%%%%%%%%%%%%%%%%%%%%%%%%%%%%%%%%%%%%%%%%%%%%%%%%%%%%%%%%%%%%%%%%
%
%\section{Übungsblatt 1}
%
%
%
%\begin{frame} \frametitle{Aufgabe 1}
%	\renewcommand*{\arraystretch}{1.1}
%	\begin{tabularx}{\textwidth}{>{\hsize=.5\hsize}X | X}
%		\toprule
%		\textbf{Begriff} & \textbf{Erklärung} \\
%		\midrule
%		\only{Syntax & Struktur (einer Sprache), erlaubte Zeichenketten  \\
%		Semantik & Bedeutung der Zeichenketten}<2->  \\
%		\only{Objektsprache & (syntaktisch) zu beschreibende Sprache \\
%		Metasprache & Hilfssprache zur Beschreibung der Objektsprache }<3->\\
%		\only{Alphabet $\Sigma$ & nichtleere, endliche Menge von Terminalsymbolen, Zeichenvorrat}<4-> \\
%		\only{Wort & endliche Folge von Symbolen}<5-> \\
%		\only{Konkatenation & Verkettung von Wörtern}<6-> \\
%	\end{tabularx}
%\end{frame}	
%
%\begin{frame} \frametitle{Aufgabe 1 (Fortsetzung)}
%	\renewcommand*{\arraystretch}{1.3}
%	\begin{tabularx}{\textwidth}{>{\hsize=.5\hsize}X | X}
%%		\toprule
%		 Begriff & Erklärung \\
%		\midrule
%		\only{Potenzmenge $\mathcal{P}$ & Menge aller Teilmengen}<2-> \\
%		\only{$\Sigma^\ast$ & Menge aller Wörter über $\Sigma$}<3-> \\
%		\only{$\mathcal{P}(\Sigma^\ast)$ & Menge aller Sprachen über $\Sigma$}<4-> \\
%		\only{formale Sprache $L$ & Menge von Wörtern über $\Sigma$ \newline $L \in \mathcal{P}(\Sigma^\ast)$}<5-> \\
%		\only{Komplexprodukt ''$\cdot$`` & Verknüpfung von Sprachen \newline $L_1 \cdot L_2 = \{uv \mid u \in L_1, v \in L_2 \}$}<6-> \\
%		\only{$L^\ast$ & Menge aller Konkatenationen von Wörtern aus $L$ \newline $L^\ast = \bigcup_{n \ge 0} L^n$ mit $L^0 = \menge{\epsilon}$ und $L^{n+1} = L^n * L$}<7-> \\
%		\bottomrule
%	\end{tabularx}
%\end{frame}
%
%\begin{frame} \frametitle{Aufgabe 2}
%	Sei $\Sigma = \{1,2,a,b\}$. \pause
%	\begin{itemize}[leftmargin=*]
%		\item \textbf{Wörter}
%		\dots entstehen durch Konkatentation von Symbolen \\
%		z.B. $\epsilon, 1, 2, a,b,12,1a,1b,21,22,2a,2b,ab,abba,\dots$
%		\pause
%		\item Symbole $\overset{\text{''}\cdot\text{``}}{\longrightarrow}$ Wörter $\overset{\in}{\longrightarrow} \underbrace{\Sigma^\ast}_{\text{Menge 1. Ordnung}} \overset{\in}{\longrightarrow} \underbrace{\mathcal{P}(\Sigma^\ast)}_{\text{Menge 2. Ordnung}}$
%		\pause
%		\item \textbf{Sprache $L$}
%		\dots Menge von Wörtern, d.h. $L \subseteq \Sigma^\ast$ bzw. $L \in \mathcal{P}(\Sigma^\ast)$ , z.B.
%		\begin{equation*}
%		\begin{aligned}
%		L &= \{1,1a,1b,1aa,1bb,1ab,1aab, \dots \} 
%		= \{1 a^n b^m \colon n,m \ge 0 \} \\
%		&= \{1\} \cdot \{a\}^\ast \cdot \{b\}^\ast
%		\end{aligned}
%		\end{equation*}
%		
%		\textbf{Beachte:} $\emptyset \in \mathcal{P}(\Sigma^\ast)$ und $\epsilon \in \Sigma^\ast$
%	\end{itemize}
%	
%\end{frame}
%
%\begin{frame} \frametitle{Aufgabe 3}
%	Seien $L_1 = \{a\}$, $L_2 = \{b\}$, $L_3 = \{a,ba\}$.
%	
%	\begin{itemize}[leftmargin=*]
%		\item $L_1 * L_2 * L_3 = \{aba,abba\}$
%		\item $L_1^\ast = \{a\}^\ast = \{\epsilon,a,aa,aaa, \dots\} = \{a^n \colon n \ge 0\}$
%	 	\item $L_3^\ast = \{\epsilon, a,ba,aa,aba,baa,baba,\dots\} = \{a^{m_1}(ba)^{n_1} \cdots a^{m_k}(ba)^{n_k} \colon m_i, n_i \in \N, k \in \N^+, 1 \le i \le k\}$
%	 	\item $L_2^\ast * L_1 = \{a,ba,bba,bbba,\dots\} = \{b^n a \colon n \ge 0\}$
%	 	\item $\pows{L_1^\ast} = \menge{\emptyset, \menge{\epsilon}, \menge{a}, \menge{aa}, \menge{aaa}, \dots , \menge{\epsilon, a}, \menge{\epsilon,aa}, \menge{\epsilon,aaa}} = \menge{\menge{a^n \colon n \in I} : I \subseteq \N}$
%	\end{itemize}
%\end{frame}

\section{Keine Angst vor Mathe!}

\begin{frame} \frametitle{Keine Angst vor Mathe!}
	\begin{block}{Euklid: Satz 4 in Buch II der ''Elemente``}
		Wird eine Strecke in zwei geteilt, dann ist das Quadrat über der ganzen Strecke gleich den Quadraten über den Teilen und dem doppelten Rechteck, das die Teile ergeben, zusammen.
	\end{block}


	\small siehe \url{http://www.opera-platonis.de/euklid/Buch2.pdf}
\end{frame}

\begin{frame} \frametitle{Keine Angst vor Mathe!}
	\begin{block}{al-Khwarizmi in Al-jabr wa'l muqabalah'}
		\small What must be the amount of a square, which, when twenty-one
		dirhems are added to it, becomes equal to the equivalent of ten
		roots of that square?
		
		\textbf{Solution}: Halve the number of the roots; the moiety is five.
		Multiply this by itself; the product is twenty-five. Subtract from
		this the twenty-one which are connected with the square; the
		remainder is four. Extract its root; it is two. Subtract this from
		the moiety of the root, which is five; the remainder is three. This
		is the root of the square which you required, and the square is
		nine. Or you may add the root of the moiety of the roots; the
		sum is seven; this is the root of the square which you sought for,
		and the square itself is forty nine.
	\end{block}
\end{frame}

\begin{frame} \frametitle{Der Kleene-Stern}
	\begin{Definition}[Kleene-Stern]
		Für eine formale Sprache $L$ definieren wir
		\begin{equation*}
		L^\ast = \bigcup_{n \ge 0} L^n = \bigcup_{n = 0}^\infty L^n
		\end{equation*}
		wobei $L^0 = \{ \epsilon \}$ und $L^{n+1} = L^n \cdot L$. \\
		Beachte: $\{ \epsilon \}^\ast = \emptyset^\ast = \{ \epsilon \}$
	\end{Definition}
\end{frame}

\end{document}